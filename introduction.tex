%!TEX root = ./butidigress.tex
\documentclass[./butidigress.tex]{subfiles}
\begin{document}
\chapter{Introduction}\label{chap:intro}
\epigraph{I have but one lamp by which my feet are guided, and that is the lamp of experience. I know of no way of judging the future but by the past.}{\attrib{Patrick Henry}{speech at the Virginia Convention}{1775}}
\newpage

% Why am I writing this?
This document is an attempt to layout the personal philosophy that I have developed over the course of my life so far.
While realizing that I have only lived 23 years, quite a short time in the scheme of things, I believe that documenting my thoughts will serve me to reflect on from whence I came and to meditate on where I am going.
I intend this to be some manner of living document, updated over time with my new insights.

This introduction exists to define some terms and as exposition regarding the development of my philosophy.
We start our journey with a discussion of ethics vs. morals and follow that up with a (mostly) chronological account of my journey to this point.
Just before getting to the meat of my elucidation I'll mention some influences.

\unnumsection{Ethics}\label{sec:ethics}
Personal philosophy is a bit abstract, I would, to a certain degree, prefer the term \emph{ethical code}.
What does that mean exactly?
I assume that we all have a passing familiarity with the term ethics, but this usage may be slightly opaque.
I think a precise definition of the term ethical (or ethics) would be appropriate here, I am using a definition set forth in \textit{Objectivity}, by Lorraine Daston and Peter Galison (2007).\autocite{objectivity}
\begin{description}
    \item [ethics] normative codes of conduct that are bound up with a way of being in the world, an ethos in the sense of the habitual disposition of an individual or group
    \item [morals] specific normative rules that may be upheld or transgressed and to which one may be held to account
\end{description}

I really doubt that these are very many people's working definitions of these terms.
It doesn't really matter in this case, because I'm just defining these terms to mean those definitions in this specific context.
Once you're done reading this (this is a sentence which will likely only ever be read by me) you can go back to whatever normal definition you have, but for now use these.

I have no real moral code (by the above definition), besides, like, the law, I guess; I'm saying that I don't believe in God or religion, not that I don't care about morality.
I don't believe in an afterlife; my ethics are founded purely on choice, or they try to be, and I don't think there will be any reward for following them or a punishment for breaking them.

Removing incentive, I have found (along with many behavioral psychologists and those who realize the obvious), tends to make one less likely to do some thing.
The question that comes to mind is: why continue to follow?

The answer is in two parts, neither of which are likely to mollify current skeptics.
First, this list is, at least in part, descriptive, it's how I have lived my life (which has worked out pretty well for me so far).
The second part is really a rejection of the question; living this way has made my life better, more fulfilling.

\unnumsection{Development}\label{sec:development}
\epigraph{Constant and frequent questioning is the first key to wisdom\ldots\ For through doubting we a led to inquire, and by inquiry we perceive the truth.}{\attrib{Peter Abelard}{Sic et Non}{~1121}}
I started college pretty goddamn depressed; for about two years, I thought about my suicide more like a historical event than a potential future choice.
I started this document near the end of my \nth{5} year of college, I am now without suicidal thoughts (for all intents and purposes---I'm not gonna say I'm perfect).

When I first decided to write this, I figured it might be beneficial or focusing to write down how I want to ideally live my life.
After talking to people about my task, I decided to modify my purpose.
The original goal remains: a codification of a lifestyle; a new goal enters: a reflection of how I got from preordained death to a healthy(er) human being.
What this document has become for me is an accounting of how I survived, an existential \textit{Robinson Crusoe} if one wanted to be full of oneself.
The tenets enumerated below represent the codification of life changes I made in order to, in the most literal sense, save my own life.
They are a product of years of hard work; I introspected and I studied and I made friends and I sought help.

There is something, quite a lot actually, to be said for getting help if you need it.
Don't ever be too proud to get help, it's not even a matter of pride.
The stigmatization of mental health is dangerous and buying into it can be deadly.
Learning that there isn't anything wrong with me was groundbreaking and is to a large degree the reason that these guidelines\footnote{The five points? I don't know what to call them.} exist.

\unnumsection{Influences}\label{sec:influences}\marginnote{5/10/18}
The previously written introduction to this section was an exposition on the key influences and insights David Foster Wallace lent me.
Using language and tone that seem chillingly crass looking back, I derided those I felt ``misappropriated'' his philosophy, those fucking douchebags.
The pedestal upon which I stood was, apparently, at an elevation high enough to destroy any semblance of rational thought.

With the rise of the \#MeToo movement and the general social change around 2017--18, the subject of David Foster Wallace's deeds have come to the occupy some space within the public consciousness once again.
I do not in any way, shape, or form want to sound like I begrudge any of the women---especially Mary Karr---coming forward and reporting sexual assault or other vile behavior from men; the following few paragraphs are not intended to read like I'm complaining that because people are speaking out I've been inconvenienced.
Having to rewrite is literally nothing compared to such personal trauma.
Ideally my explanation will sound less like a anti-woman screed and more like bargaining for my own morality.

You may recall in the disclaimer (page \pageref{chap:disclaimer}) my mention of the fundamental connection between art and artist, that one cannot be interpreted without the other.
This follows from a personal definition of art, which is reality mediated through the human experience---or something to that effect.
What I'm trying to say is that a human who produces some work of art, independent of medium, has left in said work an imprint of their essence, a trace of their true, unconscious mind.
(Art is commonly seen as an expression of the unconscious mind.
Basically that's how unintentional symbolism can happen.)
Assuming this view on art is sound (which I'd imagine a great deal of people would have problems here) one inexorably comes to the conclusion that art has morality, reflective of the artist's.

Essentially: one can---frankly one should---judge the quality/morality of art based on its creator.
Like, I won't ever again watch \textit{American Beauty} or a countless number of other movies (\textit{American Beauty} just always comes to mind).
(This line of thought could segue into a conversation about whether or not Weinstein produced films are not okay, but that will be left for another time.\footnote{or I might get into it later on in this section\ldots})

Another influence for the items listed here is Albert Camus, specifically absurdism.
I don't embrace absurdism wholeheartedly; there were some strikingly problematic elements in \textit{The Myth of Sisyphus}, what I do take from Camus' philosophy is the desperate, futile struggle against the void.
This fight, that unwinnable, yet cosmically noble struggle against the unknowable appeals to me.

Not all of my influences are now deceased well-known acclaimed authors; the next influence I'll talk about is actually a duo: Freddie Wong and Matthew Arnold.
(The acclaimed authors bit isn't intended as a slight, it was just funny, so shut the fuck up or something.)
Sincerity is one of the core virtues I'll be exploring later on, and my fascination with the concept stems in large part from these two.
I've been watching \textit{freddiew} videos, I assume, since I first went on YouTube, a lot of his early work is pretty ubiquitous on the World Wide Web.
Wong and Arnold also produced \textit{Video Game High School}, which I enjoyed immensely and their Hulu series was a good spot of fun as well.
Their podcast, \textit{Story Break}, is almost certainly my favorite podcast, but their true contribution to my life comes in introducing me to sincerity.\footnote{I don't mean sincerity the concept, obviously I know what that is. They pointed out sincerity in media, most memorably \movie{Speed Racer}.}

I'll not get too deep into it here, as there is a whole chapter later on dedicated to it, but I, from them, realized that it was the reason I enjoyed certain works.
The first example of this is \textit{Speed Racer}, my favorite movie (well, tied for first\footnote{With \movie{Short Term 12} which---you know what, we'll get into it later.}), but I also see it in their works.

One of my first influences, another group of Internet\footnote{This is the wrong usage of the term.} pioneers, Rooster Teeth, has guided me towards a path that I hope will involve more creativity than the normal engineering life.\marginnote{5/1/2018}
I don't remember the most recent time that I \emph{wasn't} a Rooster Teeth fan.
(Checking my profile, it says I joined in 2008, I feel like this may have been after I starting watching)
I've been through a whole mess of development with Red vs. Blue and Achievement Hunter and Funhaus as a backdrop.
Burnie Burns is one of my heroes; I've learn much about not being afraid to fail (which I haven't fully assimilated) from him.

Film Crit Hulk is the pseudonym of an Internet film critic, a man who is, in my estimation, one of the sharpest literary and pop-culture minds around.

\unnumsection{The Title}\label{sec:thetitle}\marginnote{5/2/2018}
Digression is one of the main features of my personal discourse.
It comes up a little less in casual conversation (mainly because I'm trying to end the conversation ASAP), but it is \emph{for sure} a key feature of my writing.
So get ready, cause this whole document is gonna be chalk full of random detours from the stated subjects.

\unnumsection{Obligatory Chapter Rundown}\label{sec:chapterrundown}\marginnote{5/10/2018}%\footnote{I hope I have made mine more narrative than most of that genre}
As is standard in any textbook type work\footnote{I hope this book is less dry and more narrative-ish than most textbooks} I will now embark on an explanatory tour of each chapter, covering their subject in brief and providing some useful nuggets.

\end{document}