%!TEX root = ./butidigress.tex
\documentclass[./butidigress.tex]{subfiles}
\begin{document}
\chapter{Introduction}\label{chap:intro}
\epigraph{I have but one lamp by which my feet are guided, and that is the lamp of experience. I know of no way of judging the future but by the past.}{\attrib{Patrick Henry}{speech at the Virginia Convention}{1775}}
\newpage

\setcounter{footnote}{0}

% Why am I writing this?
This document is an attempt to layout the personal philosophy that I have developed over the course of my life so far.
While realizing that I have only lived 23 years, quite a short time in the scheme of things, I believe that documenting my thoughts will serve me to reflect on from whence I came and to meditate on where I am going.
I intend this to be some manner of living document, updated over time with my new insights.

This introduction exists to define some terms and as exposition regarding the development of my philosophy.
We start our journey with a discussion of ethics vs. morals and follow that up with a (mostly) chronological account of my journey to this point.
Just before getting to the meat of my elucidation I'll mention some influences.

\unnumsection{Ethics}\label{sec:ethics}
Personal philosophy is a bit abstract, I would, to a certain degree, prefer the term \emph{ethical code}.
What does that mean exactly?
I assume that we all have a passing familiarity with the term ethics, but this usage may be slightly opaque.
I think a precise definition of the term ethical (or ethics) would be appropriate here, I am using a definition set forth in \textit{Objectivity}, by Lorraine Daston and Peter Galison (2007).\autocite{objectivity}
\begin{description}
    \item [ethics] normative codes of conduct that are bound up with a way of being in the world, an ethos in the sense of the habitual disposition of an individual or group
    \item [morals] specific normative rules that may be upheld or transgressed and to which one may be held to account
\end{description}

I really doubt that these are very many people's working definitions of these terms.
It doesn't really matter in this case, because I'm just defining these terms to mean those definitions in this specific context.
Once you're done reading this (this is a sentence which will likely only ever be read by me) you can go back to whatever normal definition you have, but for now use these.

I have no real moral code (by the above definition), besides, like, the law, I guess; I'm saying that I don't believe in God or religion, not that I don't care about morality.
I don't believe in an afterlife; my ethics are founded purely on choice, or they try to be, and I don't think there will be any reward for following them or a punishment for breaking them.

Removing incentive, I have found (along with many behavioral psychologists and those who realize the obvious), tends to make one less likely to do some thing.
The question that comes to mind is: why continue to follow?

The answer is in two parts, neither of which are likely to mollify current skeptics.
First, this list is, at least in part, descriptive, it's how I have lived my life (which has worked out pretty well for me so far).
The second part is really a rejection of the question; living this way has made my life better, more fulfilling.

\unnumsection{Development}\label{sec:development}
\epigraph{Constant and frequent questioning is the first key to wisdom\ldots\ For through doubting we a led to inquire, and by inquiry we perceive the truth.}{\attrib{Peter Abelard}{Sic et Non}{~1121}}
I started college pretty goddamn depressed; for about two years, I thought about my suicide more like a historical event than a potential future choice.
Missing out on what amounts to about two years of social development in such an important time is hard.

This document got started near the end of my \nth{5} year of college, I am now without suicidal thoughts (for all intents and purposes---I'm not gonna say I'm perfect\footnote{or that I never feel depressed or shitty}).
I'm not sure when I got the idea to start writing.
For sure I got the idea sometime in the spring semester of my \nth{5} year at Northeastern, I just remember that I started writing just after getting the idea.

When I first decided to write this, I figured it might be beneficial or focusing to write down how I want to ideally live my life.
After talking to people about my task, I decided to modify my purpose.
The original goal remains: \textbf{a codification of a lifestyle}; a new goal enters: \textbf{a reflection of how I got from preordained death to a healthy(er) human being}.
What this document has become for me is an accounting of how I survived, an existential \textit{Robinson Crusoe} if one wanted to be full of oneself.
The tenets enumerated below represent an attempt at codification of life changes I made in order to, in the most literal sense, save my own life.
They are a product of years of hard work; I introspected and I studied and I made friends and I didn't do it alone.

There is something, quite a lot actually, to be said for getting help if you need it.
Don't ever be too proud to get help, it's not even a matter of pride.
The stigmatization of mental health is dangerous and buying into it can be deadly.
Learning that there isn't anything wrong with me was groundbreaking and is to a large degree the reason that these guidelines exist.

\unnumsubsection{Narrative Interlude}\label{subsec:narrative}\margindate{5/11/18}
This is the part where I'll go through some years of college and explain what happened in them.
(Also the part where I hope that my dumbass college years weren't just some derivative bullshit\footnote{also the part where I hope I don't swear too much\ldots actually probably not the only part}.)

I attended, as previously probably mentioned, Northeastern University in Boston, Massachusetts.
It was not my first choice, or really even a destination that I had considered before I got in.
My reason for applying? my friend, Henry, texted me (or talked to me at water polo practice) and said (approximately), ``Yo, apply to Northeastern, they don't need an essay.''
Northeastern was the first college that accepted me, the first of four, and they gave me some cash and seemed to actually want me.
(The others were state schools, two didn't give me a major that I wanted and I didn't like the other all that much when I visited.)

So I essentially went in completely blind and ended up staying for five years.\footnote{the majority of people at Northeastern do five years b/c co-op}
Basically, you can split that ish into two parts: the first two years and the last three; the former were frankly pretty garbage, the latter, probably the best years of my life.

So we'll tackle the first two years, this is going to be \ldots\ not that dope, but hopefully cathartic.

\vdots

Lol jk, we'll actually go over some of the people that I met while at Northeastern, just as a primer (and also for some emotional comments\footnote{if anyone I went to Costa Rica with is reading this, it may feel familiar}).
(Trinna omit last names here maybe?)
(And also, I feel like I'm gonna be pretty shitty at explaining why I love these people so much, but to be fair, I'm pretty drunk.
I do love them though.)

Damn, wrote that last paragraph when I was a lot drunker than I thought I was.\margincomm{So I'll use this spot to talk about something that occurred to me while writing this. It's that books are, I get that I'm reusing a word here, atomic. As far as recollection holds, all books I have read, read like they were written in one long ass sitting. I can't recall a book that talked about, or referenced, the time that passed during the writing process, or the different states of mind of different writing sessions.}

Sober Michael signing back on.
Expounding on one's friends without being gushy (not that there's anything wrong with that necessarily) or overly revelatory re: unexpressed feelings is obviously difficult.
I consider my core friend group to be the people with whom I went to Costa Rica in Spring 2018 plus one or two people.
The order in which I mention people is not intended to be relevant, but instead that which made narrative sense.

Freshman year of college, within the first week or two of classes, I took the elevator down a couple floors to a party in a room below me; it was there that I would meet some lifelong (I assume) friends.
This room\footnote{technically it was a suite} was the residence of the first two people I'll mention, Brock and Vishal; Brock I knew from water polo and Vishal I knew from Brock.% \margincomm{Water polo is a running theme in my life, kinda a lot of what I have is b/c water polo}
Another teammate of mine, Will, introduced me to his roommate Milan, in them I found my future housemates and another two great friends.

\vspace{1em}\textbox{Alright, so full disclosure time.
Strong memories of this party, frankly, don't really exist; to be fair, it was five years ago and there was some underage drinking, allegedly.
So there may be some conflation of different parties in the same place.
Basically, Brock and Vishal's suite is like, a metaphor for the early days of school.}

Here too, I met Averie, with whom I have an uncharacteristically vivid memory of her finding me at the party once every like fifteen minutes and making sure that remembered her name.

So this is the part where I drop the whole party metaphor thing, because it's tiresome and it feels a bit disingenuous.
Instead I'll just run through the remaining people, starting with Carly.\footnote{shoutout, b/c you're probably the one reading this}

Carly is a person about whom it is hard to say too much about; a person with whom I actually can have an interesting philosophical conversation without feeling like a goddamn charlatan\footnote{but like I still do a bit\ldots\ but that's my own thing}.

Naty, probably the nicest human being I have had the pleasure to meet, again a person with whom I can discuss so many topics, from engineering stuff, to \textit{The Office} (see page \pageref{chap:sincerity} for more on that).

Mary, I believe, but have had trouble expressing in the past, is impressive in a particular way.
There is no possible expression here that won't be reductive or condescending or patronizing.
But, her ability to like, sound and act like a normal human being while also being maybe the smartest person I know is just plain ridiculous.

The dynamic brother-sister duo of Dre and Bella are last, but a value-based ordinal position doesn't really make sense\footnote{as they say}.
All I'll say about Dre is that he's fascinating.
Talking movies with Bella is always a delight, and she is one of the few people I know (in real life) whose movie opinions and analysis I really respect.

\vspace*{3\parskip}

\ldots\ Getting back to point, starting with the first two years of school.
In reading about all those friends, one might be inclined to assume some pleasant things about early college, and yes, the first bit was good.
Got good grades, made new friends, but, the second semester, the Spring semester was not as good.

\unnumsection{Influences}\label{sec:influences}\margindate{5/10/18}
The previously written introduction to this section was an exposition on the key influences and insights David Foster Wallace lent me.
Using language and tone that seem chillingly crass looking back, I derided those I felt ``misappropriated'' his philosophy, those fucking douchebags.
The pedestal upon which I stood was, apparently, at an elevation high enough to asphyxiate any semblance of rational thought.

With the rise of the \#MeToo movement and the general social change around 2017--18, the subject of David Foster Wallace's deeds have come to the occupy some space within the public consciousness once again.
I do not in any way, shape, or form want to sound like I begrudge any of the women---especially Mary Karr---coming forward and reporting sexual assault or other vile behavior from men; the following few paragraphs are not intended to read like I'm complaining that because people are speaking out I've been inconvenienced.
Having to rewrite is literally nothing compared to such personal trauma.
Ideally my explanation will sound less like a anti-woman screed and more like bargaining for my own morality.

You may recall in the disclaimer (page \pageref{chap:disclaimer}) my mention of the fundamental connection between art and artist, that one cannot be interpreted without the other.
This follows from a personal definition of art, which is reality mediated through the human experience---or something to that effect.
What I'm trying to say is that a human who produces some work of art, independent of medium, has left in said work an imprint of their essence, a trace of their true, unconscious mind.
(Art is commonly seen as an expression of the unconscious mind.
Basically that's how unintentional symbolism can happen.)
Assuming this view on art is sound (which I'd imagine a great deal of people would have problems here) one inexorably comes to the conclusion that art has morality, reflective of the artist's.

Essentially: one can---frankly one should---judge the quality/morality of art based on its creator.
Like, I won't ever again watch \textit{American Beauty} or a countless number of other movies (\textit{American Beauty} just always comes to mind).
(This line of thought could segue into a conversation about whether or not Weinstein produced films are okay, but that will be left for another time.\footnote{or I might get into it later on in this section\ldots})
Part of my certainty stems from the atomicity of a particular piece of art.

No work of art is truly an island, collaboration is important and unavoidable, but most art has a specific, principle author.\margintodo{There should be more here on why my philosophy is okay, if only to allay my own fears}

Another influence for the items listed here is Albert Camus, specifically absurdism.
I don't embrace absurdism wholeheartedly; there were some strikingly problematic elements in \textit{The Myth of Sisyphus}, what I do take from Camus' philosophy is the desperate, futile struggle against the void.
This fight, that unwinnable, yet cosmically noble struggle against the unknowable appeals to me.

Not all of my influences are now deceased well-known acclaimed authors; the next influence I'll talk about is actually a duo: Freddie Wong and Matthew Arnold.
(The acclaimed authors bit isn't intended as a slight, it was just funny, so shut the fuck up or something.)
Sincerity is one of the core virtues I'll be exploring later on, and my fascination with the concept stems in large part from these two.
I've been watching \textit{freddiew} videos, I assume, since I first went on YouTube, a lot of his early work is pretty ubiquitous on the World Wide Web.
Wong and Arnold also produced \textit{Video Game High School}, which I enjoyed immensely and their Hulu series was a good spot of fun as well.
Their podcast, \textit{Story Break}, is almost certainly my favorite podcast\footnote{Although I rather like \textit{Pod Save America}}, but their true contribution to my life comes in introducing me to sincerity.\footnote{I don't mean sincerity the concept, obviously I know what that is. They pointed out sincerity in media, most memorably \movie{Speed Racer}.}

I'll not get too deep into it here, as there is a whole chapter later on dedicated to it, but I, from them, realized that it was the reason I enjoyed certain works.
The first example of this is \textit{Speed Racer}, my favorite movie (well, tied for first\footnote{With \movie{Short Term 12} which---you know what, we'll get into it later.}), but I also see it in their works.

One of my first influences, another group of Internet\footnote{This is the wrong usage of the term.} pioneers, Rooster Teeth, has guided me towards a path that I hope will involve more creativity than the normal engineering life.\margindate{5/1/2018}
I don't remember the most recent time that I \emph{wasn't} a Rooster Teeth fan.
(Checking my profile, it says I joined in 2008, I feel like this may have been after I starting watching)
I've been through a whole mess of development with Red vs. Blue and Achievement Hunter and Funhaus as a backdrop.
Burnie Burns is one of my heroes; I've learn much about not being afraid to fail (which I haven't fully assimilated) from him.

Film Crit Hulk is the pseudonym of an Internet film critic, a man who is, in my estimation, one of the sharpest literary and pop-culture minds around.\margintodo{expand; he's thoughtful and emotional and just plain great (also I learned what semiotics was from him)}

\unnumsection{The Title}\label{sec:thetitle}\margindate{5/2/2018}
Digression is one of the main features of my personal discourse.
It comes up a little less in casual conversation (mainly because I'm trying to end the conversation ASAP), but it is \emph{for sure} a key feature of my writing.
So get ready, cause this whole document is gonna be chalk full of random detours from the stated subjects.

\unnumsection{Obligatory Chapter Rundown}\label{sec:chapterrundown}\margindate{5/10/2018}
As is standard in any textbook type work\footnote{I hope this book is less dry and more narrative-ish than most textbooks} I will now embark on an explanatory tour of each chapter, covering their subject in brief and providing some useful nuggets.

The chapters, as I could decide no other appropriate order, are alphabetical (excluding `the').\margindate{5/14/2018}

Empathy (chapter \ref{chap:empathy}): the ability to understand the feelings of another person, is, I claim, the result of conscious choice.
A choice that must be made relentlessly, the empathetic path is not easy, but it is rewarding.

Recognizing that a situation will not be changed is an important part of maturing and, honestly, is great for saving energy to focus on what really matters.
Immutability (chapter \ref{chap:immutable}) is the next covered topic and is a more down-to-Earth, everyday idea.
I also get into a corollary: effecting change when one can.

The pursuit of knowledge (chapter \ref{chap:knowledge}) is, on a certain level, my life's work.
In that chapter, I cover ideas on certainty and the general quest for greater understanding.

Sincerity (chapter \ref{chap:sincerity}) is an important quality to me, in people, but also in the media I consume.

The Void (chapter \ref{chap:thevoid}) refers to an extremely high level component of my philosophy, which can be viewed as an offshoot of absurdism.

\end{document}