\documentclass[../butidigress.tex]{subfiles}

\begin{document}
\chapter{Old Notes}\label{chap:oldnotes}
\newpage

\lettrine{H}{ere} is where I'll store the notes I took while writing some of the sections of this book.
Mainly because I'm afraid or hesitant to get rid of them, you never know, I might need them at some point.\ftnote{Said the hoarder to their stacks of old newspapers.}

\addcontentsline{toc}{section}{\tvshow{The 100}}
\begin{somenotes}{\tvshow{The 100}}
    \item (transcribed from my notebook, to be adapted into additions to the current section)
    \item What presented itself as a pretty standard young-adult show\ftnote{As in like, the first episode was pretty vanilla.} is actually a super intense drama. For example, in the third (?) episode of the fourth season, I saw a child die in her parents arms, of radiation poisoning; she took her last breaths, her breathing slowed until it stopped, her parents went from comforting to breaking down. That's not some young-adult bs, that's for real. \par I do really hate the Jasper character. He's kinda an embodiment of postmodern, existentialism not caring about the world, that resignation to the universe's lack of regard for human life and morality. I recognize that the universe does not have a modicum of regard for us\ftnote{It can't, it's inanimate, but the point is we're insignificant.} but that does not, in my opinion, make my life meaningless. In fact, choosing to live my life by a moral code is a more noble choice, laughing in the face of the unknowable void is bravery. Succumbing to hopelessness, resigning oneself to a life free of morals or restraint is cowardly.\par I have been a fan of the show's interplay between faith and science. Clarke, generally, represents science, or pragmatism, she takes the practical solution, as hard as it may be. Jaha on the other hand, represents faith, he believes in long shots, maintaining hope, leading people to salvation through his beliefs. The show does a great job of showing that we need both.\par Being pragmatic is important, but sometimes hope is the only way out.
    \item \tvshow{The 100} contains multitudes. Jasper, while I still find him reprehensible, actually has a purpose. He is the embodiment of an unfettered Id. Obviously, that's not great in isolation. But he exists to show that there really is no point to all this survival if your humanity is lost on the way.
    \item It (\tvshow{The 100}) is, as it has ever been, a pretty brutal depiction of the conflict between humanity's base savagery and our better angels. Can we, as a species, be civil in the most trying circumstances. The bunker people (Octavia) and the prison crew (the Colonel) are the negative side of that. Clarke and the six on the Ark are the positive side.\par The fact that the \say{good} groups are small is a bit of a downer. But, the large groups have people who speak for civility (Kane and the Pilot in the bunker and with the prisoners, respectively). Obviously the season has yet to play out fully, so we don't know who wins out.
\end{somenotes}

\addcontentsline{toc}{section}{\movie{Speed Racer}}
\begin{somenotes}{\movie{Speed Racer}}
    \item \href{https://birthmoviesdeath.com/2013/11/20/hulks-favorite-movies-speed-racer-2008}{Hulk---2013}\autocite{hulkspeedracer2013}
    \begin{itemize}
        \item the movie is \emph{assured}
        \item the sincerity of family and feelings
        \item \say{What brings on Hulk's favoritism? Often it is spark, playful ingenuity, a belief in self. \ldots [T]hey make movies \emph{feel} new, even when the notion is an impossibility.}\autocite{hulkspeedracer2013}
        \item \say{\movie{Speed Racer} is really serious about its world. It's serious about the stakes. It's serious about the fun. It's serious about the characters. That's because it just \emph{believes in itself}.}
        \item \say{\ldots [T]he creativity of \movie{Speed Racer} is expressed in a much weirder way that easily throws us on just about every tonal level.}
        \item \say{But this isn't a lame case of simply trying to please everybody. This is a film where every single decision makes sens for the story and characters. Better yet, every decision hammers home the point of the movie itself.}
        \item \say{So when Hulk says you just have to \say{go with it} to enjoy \movie{Speed Racer}, Hulk isn't saying to turn your brain off. Hulk is actually asking you to turn your brain on. For this is as vibrant, swelling, and unapologetic a movie as they come.}
    \end{itemize}
    \item \href{https://birthmoviesdeath.com/2015/03/27/film-crit-hulk-smash-speed-racer-as-artist}{Hulk---2015}\autocite{hulkspeedracer2015}
    \begin{itemize}
        \item \say{The Wachowskis, perhaps more than any other filmmakers, suffer from the pains of being pure at heart}\autocite{hulkspeedracer2015}
        \item \say{They [the Wachowskis] make cinema that is so genuine and jaw-droppingly sincere that is can't help but skew right into most people's \say{this is corny} territory.}
        \item \say{\ldots [I]f Hulk had one creative standard of all filmmakers, it is the request that one's cinema be honest to what it really wants, what it really believes, and what it is really saying.}
    \end{itemize}
    \item \href{http://observer.com/2018/05/looking-back-at-the-wachowskis-2008-masterpiece-speed-racer/}{Hulk---2018}\autocite{hulkspeedracer2018}
    \begin{itemize}
        \item \say{[T]hey [the audience] had no idea what to do with this fluffy, neon-soaked bit of confection that was being sold to them.}
        \item \say{[T]his is indeed a true-blue PG kids film. Because of that, it will be unapologetically goofy, over the top, and prominently feature monkey gags.}
    \end{itemize}
\end{somenotes}

\end{document}