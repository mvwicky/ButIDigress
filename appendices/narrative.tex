\documentclass[../butidigress.tex]{subfiles}
\begin{document}

\chapter{Narrative Reconstruction}\label{chap:narrative}
\epi{Which his fair tongue---conceit's expositor---\\ Delivers in such apt and gracious words, \\ That aged ears play truant at his tales, \\ And younger hearings are quite ravished.}{\attrib{William Shakespeare}{Love's Labour's Lost}{c. 1595--1596}}
\newpage

\lettrine{I}{n} writing this book, I've experienced some frustration with the atomicity stemming from the way that I've laid this whole thing out.
Giving each subject a specific chapter makes it somewhat difficult to naturally express synthetic ideas, that is, concepts which are rooted in more than one concept.
In order to fight this problem, I've been trying to find a way to restructure in such a way that I can weave in and out of concepts elegantly.
The idea that has grabbed me is a reformulation, the construction of a narrative.

\end{document}