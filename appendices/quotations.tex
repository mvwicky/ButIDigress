%!TEX program = lualatex
%!TEX root = ../butidigress.tex
\documentclass[../butidigress.tex]{subfiles}

\begin{document}
\chapter{Quotations}
\newpage
\newlength{\episkip}
\setlength{\episkip}{0.5cm}
% \newcommand{\postepi}{\vspace{\episkip}\noindent\hfill\rule{\textwidth}{0.1em}\hfill\vspace{\episkip}}
\newcommand{\postepi}{}
% https://en.wikiquote.org/wiki/Truth
% https://en.wikiquote.org/wiki/Skepticism
% https://en.wikiquote.org/wiki/Knowledge
% https://en.wikiquote.org/wiki/Science
% https://en.wikiquote.org/wiki/History_of_science
% https://en.wikiquote.org/wiki/Atomic_theory
% https://en.wikiquote.org/wiki/Immanuel_Kant
% https://en.wikiquote.org/wiki/Malcolm_X
% https://en.wikiquote.org/wiki/Martin_Luther_King,_Jr.
Some quotations that I decided to not use as epigraphs (and also other quotes that I like).
Honestly, some of these may be upgraded to epigraph status; for example, I had a Camus quote from \textit{The Myth of Sisyphus}, which is now on page~\pageref{sec:voidexpounding}

The quotations may seem, to some to be kinda cheesy, or indicative of me not being able to come up with all of my own ideas
\par
\postepi
\epigraph{How does it happen that a properly endowed natural scientist comes to concern himself with epistemology? Is there no more valuable work in his specialty? I hear many of my colleagues saying, and I sense it from many more, that they feel this way. I cannot share this sentiment. When I think about the ablest students whom I have encountered in my teaching, that is, those who distinguish themselves by their independence of judgment and not merely their quick-wittedness, I can affirm that they had a vigorous interest in epistemology. They happily began discussions about the goals and methods of science, and they showed unequivocally, through their tenacity in defending their views, that the subject seemed important to them. Indeed, one should not be surprised by this.}{\attrib{Albert Einstein}{Physikalische Zeitschrift}{1916}}
\postepi
\epigraph{Tonight, we gather to affirm the greatness of our nation---not because of the height of our skyscrapers, or the power of our military, or the size of our economy. Our pride is based on a very simple premise, summed up in a declaration made over two hundred years ago: `We hold these truths to be self-evident, that all men are created equal, that they are endowed by their Creator with certain inalienable rights, that among these are life, liberty, and the pursuit of happiness.' That is the true genius of America---a faith in simple dreams, an insistence on small miracles.}{\attrib{Barack Obama}{DNC Keynote Speech}{2004}}
\postepi
\epigraph{The best episodes of \tvshow{The West Wing} that dealt with policy and stuff, in my opinion, were the ones where they were in the middle of a crisis, and they were trying to figure out how to solve problems.}{\scshape Michael Schur}
\postepi
\epigraph{The best shows are always the ones that are very, very low-concept and just about great characters.}{\scshape Michael Schur}
\postepi
\epigraph{You should be nice to people because it's better to be nice to people than mean to people, not because you think there's something in it for you.}{\scshape Michael Schur}
\postepi
\epigraph{There's no way to approach anything in an objective way. We're completely subjective; our view of the world is completely controlled by who we are as human beings, as men or women, by our age, our history, our profession, by the state of the world.}{\scshape Charlie Kaufman}
\postepi
\epigraph{It occurred to me that every work of art is a synecdoche, there's no way around it. Every creative work that someone does can only represent an aspect of the whole of something. I can't think of an exception to that.}{\scshape Charlie Kaufman}
\postepi
\epigraph{Happiness is not an ideal of reason but of imagination}{\attrib{Immanuel Kant}{Fundamental Principles of the Metaphysics of Ethics}{1785}}
\postepi
\epigraph{Science is the belief in the ignorance of experts.}{\scshape Richard Feynman}
\postepi
\epigraph{No union is more profound than marriage, for it embodies the highest ideals of love, fidelity, devotion, sacrifice, and family. In forming a marital union, two people become something greater than once they were. As some of the petitioners in these cases demonstrate, marriage embodies a love that may endure even past death. It would misunderstand these men and women to say they disrespect the idea of marriage. Their plea is that they do respect it, respect it so deeply that they seek to find its fulfillment for themselves. Their hope is not to be condemned to live in loneliness, excluded from one of civilization's oldest institutions. They ask for equal dignity in the eyes of the law. The Constitution grants them that right.\par\hspace*{2em} The judgment of the Court of Appeals for the Sixth Circuit is reversed. \par\hfill \textit{It is so ordered.}}{\attrib{Justice Kennedy}{Obergefell v.\ Hodges}{2015}}
\end{document}