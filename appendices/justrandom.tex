\documentclass[../butidigress.tex]{subfiles}

\begin{document}

\chapter{Just Random Bullshit\ldots}\label{chap:justrandom}
\newpage

This appendix serves two purposes.
The first is as a area for my thoughts on subjects that come to my mind.
Notes made regarding said subjects will live here until I can figure out how to integrate them into the main matter of the book.
The second is for stuff that really will never serve any purpose.
For example, the random formal definitions of the pumping lemmas.

\section{Random Thoughts on Speech}\margintodo{try to incorporate this into \say{Knowledge} (\ref{chap:knowledge})}
Our (humanity's) ability to describe is, as far as is currently known, unique amongst animals, or at least, we are by far the best at descriptive language.
Other species, most or all of them, have the ability to speak, some form of language exists.
However, they deal purely within the scope of the prescriptive or the immediate, e.g., \say{look out for that lion!} or \say{there are some ripe-ass berries over there}.
They don't recall.
They don't tell they're friends what they did the preceding day.

We (\textit{Homo sapiens}) do.

Not only do we flap our gums (and twirl our pens and poke our keyboards) regarding things that have happened, we communicate ideas and concepts that have no basis in reality (hot take, religion\ftnote{Slightly less hot take, nations, but I'll get into that a bit later.}). In \book{Sapiens}, it is asserted that this ability is our most fundamental property, what renders \emph{Homo sapiens} alone alongside all other animals\autocite{sapiens}.
It has been found, independently (I think, haven't checked) that \emph{Homo sapiens'} superlative pattern recognition (superior pattern processing) is a foundational element of language.

An anecdote, from Barthes: Karl von Frisch, sought to prove the commonly held hypothesis that bees had some form of language.
He succeeded in this pursuit, but found that their language contained no description, it was all prescriptive; they had to coordinate in order to get their food, they didn't talk about what they were doing or had done, if such information was not imminently necessary for survival.
Basically, we're the only animals that talk about shit that doesn't really matter/exist.

\margindate{6}{14}{18}
Benedict Anderson wrote at some length about this concept, albeit in an decidedly less biological way, in his seminal work: \book{Imagined Communities}.
Originally published in 1983, \book{Imagined Communities} explores nationalism, its rise and its effects.
While Anderson didn't really come up with the concept, it's generally acknowledge that he coined the term.
What it boils down to is the fact that a \emph{nation} doesn't actually exist, it's just that everybody agrees that it's a thing\autocite{imaginedcommunities}.
Other animals have packs, but those max out at like, fifty, human beings are unique in our ability to construct communities of millions.

\margindate{6}{15}{18}
Going down to a deeper level, that of the brain specifically, we can attribute this uniquely human ability to our ability to recognize and process patterns\autocite{spphumans}.
On a certain level, this superior pattern processing (SPP) is the \emph{key} trait that differentiates humans from all other animals.
Well, referring to the first paragraph, it's a pretty simple syllogism\ftnote{If not entirely logically sound.}: our ability to describe separates us from other animals; our ability to describe comes from SPP; SPP separates us from other animals.

\digsec{Google Memo Thoughts}{6}{14}{18}{googlememo}
(I don't really know where to put this in the main text, or if I should at all, so it's going here for now)\margintodo{probably can fit this into \say{Empathy} (\ref{chap:empathy})}

I fairly recently (not that recently at all) read that stupid memo by that big dumb idiot that used to work for Google.
By my tone, I'm assuming one can guess my reaction to said memo\ftnote{It was, bar none, the closest I've come to putting my fist through my computer screen; my hand was only stayed by my knowledge of the cost, and the futility of the act.}; it was pretty much a whole load of bullshit and, frankly, it was borderline insulting, not just to over 50\% of the population, but to me.

First, it's important to say that my feelings on this matter are a bit irrelevant, I'm not disadvantaged by gender-biased hiring practices.
But, I am commenting, so here we go.

As a sort-of left-brained\ftnote{I'm not actually all that logical in my life, but I understand logic and can think through problems and spot fallacies, it's pretty much my training as an engineer.} person, there's not much more infuriating than the flawed appeal to \textit{logos}.
There are so many ridiculous attempts at a \say{logical} explanation for why women don't have jobs in tech and the holes are big enough to fly through with a C-130.
For example, Damore points out, that many cultures throughout world history have had similarly gendered divisions in society.
So fucking what?
A shit-ton of societies also had slaves, was that cool too?
Like, there's a bunch of stuff like that, where he points out the fact that \say{everybody does it, so it's okay,} a point which is obviously asinine.

Okay, to be fair\ftnote{Not that this dude deserves fairness.}, the schoolyard justification isn't stated explicitly.
It's introduced as evidence for the \say{biological} difference between men and women, but that doesn't make the point any less fallacious\ftnote{Side note, horrible word, just sounds way too close to another, way worse to say around mixed company word.}.
There are, many, practices spanning across cultures that are not because of \say{genetics.}
If I've gotten anything out of all my reading on semiotics it's the power and pervasiveness of ingrained, social behavior.

\digsec{What Do I Want From Myself?}{7}{4}{18}{whatdoiwant}
\textbox{I'm writing this here specifically because I don't know where to put it besides the introduction, but I'm self conscious about the fact that my intro is longer than any chapter in the book.
This might be a failure of the general idea of this book, or I need to think of an angle from which to approach this subject.
I mean, it's pretty wide-ranging and meta, so the introduction might be the most appropriate.}

When a nurse triages patients, they must make a determination of who is worth saving, which human being is worth expending precious resources.
Generally, we think of this as basically ordering people from least to most \say{injured}\ftnote{Which is an obviously squishy/subjective metric.} and get doctors to treat those who are injured enough to need treatment, but not so injured that they probably won't survive regardless of assistance.
But there may be extenuating circumstances, i.e., people who are, by virtue of circumstance or for more personal reasons, more \say{worthy} of saving.
Where was I going with this?

Oh yeah, triage is analogous to my struggle in determining which personality traits I want to preserve.
I focus a great deal of energy on change, fighting depression being the overriding priority, but there are definite aspects of me that I like and want to maintain.

\newpage
\digsec{Toxic (Male) Nerd Culture}{6}{16}{18}{toxicnerddfa}
Here's a graph that I made that tracks the construction of Toxic Nerd Culture, which is, in its construction, inherently male.

\begin{figure*}[htp]
\centering
\begin{tikzpicture}[->,>=stealth',shorten >=1pt,auto,node distance=4cm,semithick]
\tikzstyle{every state}=[fill=white,text=black]
\node[initial,state]   (B)              {\scriptsize Barrier to Entry};
\node[state]           (H) [below of=B] {\scriptsize Homogeneity};
\node[state] (T) [below of=H] {\scriptsize Toxic Nerd Culture};
\node[state]           (P) [left of=H]  {\scriptsize Prior Exclusion};

\path (B) edge                    node {}                (H)
      (H) edge                    node {}                (T)
      (P) edge [left,bend right]  node {reactionary}     (T)
      (T) edge [right,bend right] node {structural bias} (H);
\end{tikzpicture}
\end{figure*}

\section{Formal Statements}
\subsection{The Pumping Lemma for Regular Languages}
\newcommand{\tab}{\hspace*{2em}}
\begin{align*}
&(\forall L\subseteq \Sigma^{\ast}) \\
&\tab (regular(L) \Rightarrow \\
&\tab ((\exists p\geq 1)((\forall w\in L)((|w| \geq p) \Rightarrow \\
&\tab ((\exists x,y,z \in \Sigma^{\ast}) \\
&\tab (w=xyz \wedge (|y| > 0 \wedge |xy|\leq p \wedge (\forall i \geq 0)(xy^{i}z\in L))))))))
\end{align*}

\subsection{The Pumping Lemma for Context Free Languages}
\begin{align*}
&(\forall L\subseteq \Sigma^{\ast}) \\
&\tab (context\ free(L) \Rightarrow \\
&\tab ((\exists p\geq 1)((\forall w\in L)((|w| \geq p) \Rightarrow \\
&\tab ((\exists u,v,x,y,z \in \Sigma^{\ast})(w=uvxyz \wedge (|vy| > 0 \wedge |vxy|\leq p\ \wedge \\
&\tab\tab (\forall i \geq 0)(uv^{i}xy^{i}z\in L))))))))
\end{align*}

\def\thmskiplen{1em}
% \newtheoremstyle{<name>}{<space above>}{<space below>}{<body font>}{<indent amount>}{<head font>}{<punctiation after thm. head>}{<space after thm head>}{<thm. head spec.>}
\newtheoremstyle{defn}{\thmskiplen}{\thmskiplen}{}{}{\bfseries}{:}{0.5em}{\thmname{#1}\thmnumber{ #2}\thmnote{ (#3)}}
\newtheoremstyle{thmskip}{\thmskiplen}{\thmskiplen}{}{}{\bfseries}{:}{0.25em}{\thmname{#1}\thmnumber{ #2}\thmnote{ [#3]}}

\theoremstyle{thmskip}
\newtheorem{theorem}{Theorem}[section]
\newtheorem*{corollary}{Corollary}
\newtheorem{lemma}[theorem]{Lemma}

\theoremstyle{defn}
\newtheorem{definition}{Definition}[section]

\newcommand{\vprime}[1]{{#1}^{\prime}}
\def\vp{\vprime}
\newcommand{\Z}{\mathbb{Z}}
\newcommand{\mathlist}{\itemsep2pt \parsep0pt \parskip0pt}

% \setlength{\parskip}{1.2em}
\digsec{Abstract Algebra Notes}{6}{16}{18}{abstractalgebra}
This section is basically just me copying \book{Contemporary Abstract Algebra}.

\digsubsec{Integers and Equivalence Relations}{6}{19}{18}{integersequivalence}

\digsubsubsec{Properties of Integers}{6}{18}{18}{propofintegers}

\begin{definition}{Well Ordering Principle}
The well ordering principle states that every nonempty set of positive integers contains a smallest number.
\end{definition}

A nonzero integer $t$ is a \emph{divisor} of an integer $s$ if there is an integer $u$ such that $s=tu$.
Write $t\mid s$, \say{$t$ divides $s$.}

When $t$ is not a divisor of $s$, write $t\nmid s$.

A \emph{prime} is a positive integer greater than $1$ whose only positive divisors are $1$ and itself.

An integer $s$ is a \emph{multiple} of an integer $t$ if there is an integer $u$ such that $s=tu$ or, equivalently, if $t\mid s$.

\begin{theorem}[Division Algorithm]
Let $a$ and $b$ be integers with $b>0$.
Then there exist unique integers $q$ and $r$ with the property that $a=bq+r$, where $0\leq r<b$.
\end{theorem}

\begin{proof}
We begin with the existence portion of the theorem.
Consider the set $S = \left\{a - bk\mid k\ \text{is an integer and}\ a - bk \geq 0 \right\}$.
If $0 \in S$, then $b$ divides $a$ and we may obtain the desired result with $q = a/b$ and $r = 0$.
Now assume $0 \notin S$.
Since $S$ is nonempty [if $a>0$, $a-b\cdot 0 \in S$; if $a < 0$, $a - b(2a) = a(1 - 2b) \in S$; $a \neq 0$ since $0 \notin S$], we may apply the Well Ordering Principle to conclude that $S$ has a smallest member, say $r - a - bq$.
Then $a = bq + r$ and $r \geq 0$, so all that remains to be proved is $r < b$.

If $r \geq b$, then $a - b(q + 1) = a - bq - b = r - b \geq 0$, so that $a - b(q + 1) \in S$.
But $a - b(q + 1) < a - bq$, and $a - bq$ is the \emph{smallest} member of $S$.
So $r < b$.

To establish the uniqueness of $q$ and $r$, let us suppose that there are integers, $q$, $\vprime{q}$, $r$, and $\vprime{r}$ such that
\begin{align*}
a = bq + r,\ 0\leq r < b\ \text{and}\ a = b\vprime{q} + \vprime{r},\ 0 \leq \vprime{r} < b
\end{align*}
For convenience, we may also suppose that $\vprime{r} \geq r$.
Then $bq + r = b\vprime{q} + \vprime{r}$ and $b(q - \vprime{q}) = \vprime{r} - r$.
So, $b$ divides $\vprime{r} - r$ and $0 \leq \vprime{r} - r \leq \vprime{r} < b$.
It follows that $\vprime{r} - r = 0$, and therefore $\vprime{r} = r$ and $\vprime{q} = q$.
\end{proof}

The integer $q$ in the division algorithm is called the \emph{quotient} upon dividing $a$ by $b$; the integer $r$ is called the \emph{remainder} upon dividing $a$ by $b$.

\begin{definition}{Greatest Common Divisor}
The greatest common divisor of two nonzero integers $a$ and $b$ is the largest of all common divisors of $a$ and $b$.
Denote this integer by $\gcd(a, b)$
\end{definition}

\begin{definition}{Relatively Prime}
When two nonzero integers $a$ and $b$ have $\gcd(a,b)=1$ they are relatively prime.
\end{definition}

\begin{theorem}[GCD is a Linear Combination]
For any nonzero integers $a$ and $b$, there exist integers $s$ and $t$ such that $\gcd(a,b) = as+bt$.
Moreover, $\gcd(a, b)$ is the smallest positive integer of the form $as + bt$.
\end{theorem}

\begin{proof}
Consider the set $S = \left\{am + bn\mid m,n\ \text{are integers and}\ am + bn > 0 \right\}$.
Since $S$ is obviously nonempty (if some choice of $m$ and $n$ makes $am + bn < 0$, then replace $m$ and $n$ by $-m$ and $-n$), the Well Ordering Principle asserts that $S$ has a smallest member, say, $d = as + bt$.
\end{proof}

\begin{corollary}
If $a$ and $b$ are relatively prime, then there exist integers $s$ and $t$ such that $as+bt=1$.
\end{corollary}

\begin{lemma}[Euclid's Lemma]
If $p$ is a prime that divides $ab$ then $p\mid a$ or $p\mid b$.
\end{lemma}

\begin{proof}
Suppose $p$ is a prime that divides $ab$ but does not divide $a$.
We must show that $p$ divides $b$.
Since $p$ does not divide $a$, there are integers $s$ and $t$ such that $1 = as + pt$.
Then $b = abs + ptb$, and since $p$ divides the right-hand side of this equation, $p$ also divides $b$.
\end{proof}

\digsubsubsec{Mathematical Induction}{6}{18}{18}{mathematicalinduction}

\begin{theorem}[First Principle of Mathematical Induction]
Let $S$ be a set of integers containing $a$.
Suppose $S$ has the property that whenever some integer $n\geq a$ belongs to $S$, then the integer $n+1$ also belongs to $S$.
Then, $S$ contains every integer greater than or equal to $a$.
\end{theorem}

To use induction to prove a statement involving positive integers is true for every positive integer, we must first verify that the statement is true for the integer 1.
We then \emph{assume} the statement is true for the integer $n$ and use this assumption to prove that the statement is true for the integer $n+1$.

\begin{theorem}[Second Principle of Mathematical Induction]
Let $S$ be a set of integers containing $a$.
Suppose $S$ has the property that $n$ belongs to $S$ whenever every integer less than $n$ or greater than or equal to $a$ belongs to $S$.
Then, $S$ contains every integer greater than or equal to $a$.
\end{theorem}

In this form of induction, we first show that the statement is true for the integer $a$.
We then \emph{assume} that the statement is true for \emph{all} integers that are greater than or equal to $a$ and less than $n$, and use this assumption to prove that the statement is true for $n$.

\digsubsubsec{Equivalence Relations}{6}{18}{18}{equivrelations}

\begin{definition}{Equivalence Relation}
An \emph{equivalence relation} on a set $S$ is a set $R$ of ordered pairs of elements in $S$ such that
\begin{enumerate}\mathlist
    \item $(\forall a\in S)((a,a) \in R)$ (reflexive property)
    \item $(a,b) \in R \rightarrow (b,a) \in R$ (symmetric property)
    \item $(a,b) \in R \wedge (b,c) \in R \rightarrow (a,c) \in R$ (transitive property)
\end{enumerate}
\end{definition}

When $R$ is an equivalence relation on a set $S$, it is customary to write $aRb$ instead of $(a,b) \in R$.
Also, since an equivalence relation is just a generalization of equality, a suggestive symbol such as $=$, $\sim$, or $\approx$ is usually used to denote the relation.
Using $\sim$ as a notation, the three conditions for an equivalence relation become $a\sim a$; $a\sim b$ implies $b\sim a$; and $a\sim b$ and $b\sim c$ imply $a\sim c$.
If $\sim$ is an equivalence relation on a set $S$ and $a \in S$, then the set $[a] = \{ x \in S \mid x\sim a\}$ is called the \emph{equivalence class of $S$ containing $a$}.

\begin{definition}{Partition}
A \emph{partition} of a set $S$ is a collection of nonempty disjoint subsets of $S$ whose union is $S$.
\end{definition}

\begin{theorem}[Equivalence Classes Partition]
The equivalence classes of an equivalence relation on a set $S$ constitute a partition of $S$.
Conversely, for any partition $P$ of $S$, there is an equivalence relation on $S$ whose equivalence classes are the elements of $P$.
\end{theorem}

\begin{proof}
Let $\sim$ be an equivalence relation on a set $S$.
For any $a \in S$, the reflexive property shows that $a \in [a]$.
So, $[a]$ is nonempty and the union of all equivalence classes.
We must show that $[a] \cap [b] = \varnothing$.
On the contrary, assume $c \in [a] \cap [b]$.
We will show that $[a] \subseteq [b]$.
To this end, let $a \in [a]$.
We then have $c \sim a$, $c \sim b$, and $x \sim a$.
By the symmetric property, we also have $a \sim c$.
Thus, by transitivity, $x \sim c$, and by transitivity again, $x \sim b$.
This proves $[a] \subseteq [b]$.
Analogously, $[b] \subseteq [a]$.
This $[a] = [b]$, in contradiction to our assumption that $[a]$ and $[b]$ are distinct equivalence classes.

To prove the converse, let $P$ be a collection of nonempty disjoint subsets of $S$ whose union is $S$.
Define $a \sim b$ if $a$ and $b$ belong to the same subset in the collection.
\end{proof}

\digsubsubsec{Functions (Mapping)}{6}{18}{18}{functionsmapping}

\begin{definition}{Function (Mapping)}
A \emph{function} (or \emph{mapping}) $\phi$ from a set $A$ to a set $B$ is a rule that assigns to each element $a \in A$ exactly one element $b \in B$.
The set $A$ is called the \emph{domain of $\phi$} , and $B$ is called the \emph{range of $\phi$}.
If $\phi$ assigned $b$ to $a$, then $b$ is called the \emph{image of a under $\phi$}.
The subset of $B$ comprising all the images of $A$ is called the \emph{image of $A$ under $\phi$}.
\end{definition}

Use the shorthand $\phi: A\rightarrow B$ to mean that $\phi$ is a mapping from $A$ to $B$.
We will write $\phi(a) = b$ or $\phi : a \rightarrow b$ to indicate that $\phi$ carries $a$ to $b$.

\begin{definition}{Composition of Functions}
Let $\phi : A\rightarrow B$ and $\psi : B\rightarrow C$.
The composition $\psi\phi$ is the mapping from $A$ to $C$ defined by $(\psi\phi)(a) = \psi(\phi(a))$ for all $a \in A$.
\end{definition}

\begin{definition}{One-to-One Function}
A function $\phi$ from a set $A$ is called \emph{one-to-one} if for every $a_{1},a_{2} \in A$, $\phi(a_{1}) = \phi(a_{2})$ implies $a_{1} = a_{2}$.
\end{definition}

\begin{definition}{Function from $A$ onto $B$}
A function $\phi$ from a set $A$ to a set $B$ is said to be \emph{onto} $B$ is each element of $B$ is the image of at least one element of $A$.
In symbols, $\phi : A\rightarrow B$ is onto if for each $b$ in $B$ there is at least one $a$ in $A$ such that $\phi(a) = b$.
\end{definition}

\begin{theorem}[Properties of Functions]
Given function $\alpha:A\rightarrow B$, $\beta:B\rightarrow C$, and $\gamma:C\rightarrow D$, then
\begin{enumerate}\mathlist
    \item $\gamma(\beta\alpha) = (\gamma\beta)\alpha$ (associativity).
    \item If $\alpha$ and $\beta$ are one-to-one, then $\beta\alpha$ is one-to-one.
    \item If $\alpha$ and $\beta$ are onto, then $\beta\alpha$ is onto.
    \item If $\alpha$ is one-to-one and onto, then there is a function $\alpha^{-1}$ from $B$ onto $A$ such that $(\alpha^{-1}\alpha)(a) = a$ for all $a$ in $A$ and $(\alpha\alpha^{-1})(b) = b$ for all $b$ in $B$.
\end{enumerate}
\end{theorem}

\digsubsec{Introduction to Groups}{6}{19}{18}{groups}
The term \emph{group} was used by Galois in \~{}1830 to describe sets of one-to-one functions on finite sets that could be grouped together to form a set closed under composition.

\digsubsubsec{Definition and Examples of Groups}{6}{19}{18}{defngroups}

\begin{definition}{Binary Operation}
Let $G$ be a set.
A \emph{binary operation} on $G$ is a function that assigns each ordered pair of element of $G$ and element of $G$.
\end{definition}

A binary operation on a set $G$ is simply a method (or formula) by which the members of an ordered pair from $G$ combine to yield a new member of $G$.
This condition is called \emph{closure}.

\begin{definition}{Group}
Let $G$ be a set together with a binary operation (usually called multiplication) that assigns to each ordered pair $(a,b)$ of elements of $G$ an element in $G$ denoted by $ab$.
We say $G$ is a \emph{group} under this operation if the following three properties are satisfied.
\begin{enumerate}\mathlist
    \item \emph{Associativity}. The operation is associative; that is, $(ab)c = a(bc)$ for all $a,b,c \in G$.
    \item \emph{Identity}. There is an element $e$ (called the \emph{identity}) in $G$ such that $ae=ea=a$ for all $a \in G$.
    \item \emph{Inverses}. For each element $a \in G$, there is an element $b \in G$ (called an \emph{inverse} of $a$) such that $ab=ba=e$.
\end{enumerate}
\end{definition}

If a group has the property that $ab = ba$ for every pair of elements $a$ and $b$, we say the group is \emph{Abelian}.
A group is \emph{non-Abelian} is there is some pair of elements $a$ and $b$ for which $ab\neq ba$.

\digsubsubsec{Elementary Properties of Groups}{6}{19}{18}{propgroups}

\begin{theorem}[Uniqueness of the Identity]
In a group $G$, there is only one identity element.
\end{theorem}

\begin{proof}
Suppose both $e$ and $\vprime{e}$ are identities of $G$.
Then,
\begin{enumerate}\mathlist
    \item $ae = a$ for all $a \in G$, and
    \item $\vprime{e}a = a$ for all $a \in G$
\end{enumerate}
The choices of $a = \vprime{e}$ and $a = e$ yield $\vprime{e}e = \vprime{e}$ and $\vprime{e}e = e$.
Thus, $e$ and $\vprime{e}$ are both equal to $\vprime{e}e$ and so are equal to each other.
\end{proof}

\begin{theorem}[Cancellation]
In a group $G$, the right and left cancellation laws hold; that is, $ba = ca$ implies $b = c$, and $ab = ac$ implies $b = c$.
\end{theorem}

\begin{proof}
Suppose $ba = ca$.
Let $\vprime{a}$ be an inverse of $a$.
Then multiplying on the right by $\vprime{a}$ yields $(ba)\vprime{a} = (ca)\vprime{a}$.
Associativity yields $b(a\vp{a}) = c(a\vp{a})$.
Then $be = ce$ and, therefore, $b = c$ as desired.
Similarly, one can prove that $ab = ac$ implies $b = c$ by multiplying by $\vp{a}$ on the left.
\end{proof}

\begin{theorem}[Uniqueness of Inverses]
For each element $a$ in a group $G$, there is a unique element $b$ in $G$ such that $ab=ba=e$.
\end{theorem}

\begin{proof}
Suppose that $b$ and $c$ are both inverses of $a$.
Then $ab = c$ and $ac = e$, so that $ab = ac$.
Canceling the $a$ on both sides gives $b = c$, as desired.
\end{proof}

\begin{theorem}[Socks-Shoes Property]
For group elements $a$ and $b$, $(ab)^{-1} = b^{-1}a^{-1}$.
\end{theorem}

\begin{proof}
Since $(ab)(ab)^{-1} = e$ and $(ab)(b^{-1}a^{-1}) = a(bb^{-1})a^{-1} = aea^{-1} = aa^{-1} = e$, we have $(ab)^{-1} = b^{-1}a^{-1}$.
\end{proof}

\digsubsec{Finite Groups; Subgroups}{6}{19}{18}{finitegroups}

\begin{definition}{Order of a Group}
The number of elements of a group (finite or infinite) is called its \emph{order}.
We will use $|G|$ to denote the order of $G$.
\end{definition}

\begin{definition}{Order of an Element}
The \emph{order} of an element $g$ in a group $G$ is the smallest positive integer $n$ such that $g^{n} = e$.
(In additive notation, this would be $ng = 0$.)
If no such integer exists, we say that $g$ has \emph{infinite order}.
The order of an element $g$ is denoted $|g|$.
\end{definition}

To find the order of a group element $g$, compute the sequence of products $g, g^{2}, g^{3},\ldots,$ until you reach the identity for the first time.
The exponent of this product (or coefficient if the operation is addition) is the order of $g$.
If the identity never appears in the sequence, then $g$ has infinite order.

\begin{definition}{Subgroup}
If a subset $H$ of a group $G$ is itself a group under the operation of $G$, we say that $H$ is a \emph{subgroup} of $G$.
\end{definition}

We use the notation $H \leq G$ to mean that $H$ is a subgroup of $G$.
If we want to indicate that $H$ is a subgroup of $G$ but is not equal to $G$ itself, we write $H < G$.
Such a subgroup is called a \emph{proper subgroup}.
The subgroup $\{e\}$ is called the \emph{trivial subgroup} of $G$; a subgroup that is not $\{e\}$ is called a \emph{nontrivial subgroup} of $G$.

\digsubsubsec{Subgroup Tests}{6}{19}{18}{subgrouptests}
When determining whether or not a subset $H$ of a group $G$ is a subgroup of $G$, one need not directly verify the group axioms.
The next three results provide simple tests that suffice to shaw that a subset of a group is a subgroup.

\begin{theorem}{One-Step Subgroup Test}
Let $G$ be a group and $H$ a nonempty subset of $G$.
If $ab^{-1}$ is in $H$ whenever $a$ and $b$ are in $H$, then $H$ is a subgroup of $G$.
(In additive notation, if $a - b$ is in $H$ whenever $a$ and $b$ are in $H$, then $H$ is a subgroup of $G$.)
\end{theorem}

\begin{proof}
Since the operation of $H$ is the same as that of $G$, it is clear that this operation is associative.
Next we show that $e$ is in $H$.
Since $H$ is nonempty, we may pick some $x \in H$.
Then, letting $a = x$ and $b = x$ in the hypothesis, we have $e = xx^{-1} = ab^{-1}$ is in $H$.
To verify that $x^{-1}$ is in $H$ whenever $x$ is in $H$, all we need to do is choose $a = e$ and $b = x$ in the statement of the theorem.
Finally, the proof will be complete when we show that $H$ is closed; that is, if $x$, $y$ belong to $H$, we must show that $xy \in H$.
We have already shown that $y^{-1}$ is in $H$ whenever $y$ is; so, letting $a = x$ and $b = y^{-1}$, we have $xy = x(y^{-1})^{-1} = ab^{-1}$ is in $H$.
\end{proof}

There are actually four steps in applying the previous theorem, although the first three will become routine.
Note the similarity between the last three steps and the three steps involved in the Second Principle of Mathematical Induction.
\begin{enumerate}\mathlist
    \item Identify the property $P$ that distinguishes the elements of $H$; that is, identify a defining condition.
    \item Prove that the identity has property $P$. (This verifies that $H$ is nonempty)
    \item \emph{Assume} that two elements $a$ and $b$ have property $P$.
    \item Use the assumption that $a$ and $b$ have property $P$ to show that $ab^{-1}$ has property $P$.
\end{enumerate}

\begin{theorem}[Two-Step Subgroup Test]
Let $G$ be a group and let $H$ be a nonempty subset of $G$.
If $ab$ is in $H$ whenever $a$ and $b$ are in $H$ ($H$ is closed under the operation), and $a^{-1}$ is in $H$ whenever $a$ is in $H$ ($H$ is closed under taking inverses), then $H$ is a subgroup of $G$.
\end{theorem}

\begin{proof}
By the One-Step Subgroup Test, it suffices to show that $a,b \in H$ implies $ab^{-1} \in H$.
So, we suppose that $a,b \in H$.
Since $H$ is closed under taking inverses, we also have $b^{-1} \in H$.
Thus, $ab^{-1} \in H$ by closure under multiplication.
\end{proof}

How do you prove that a subset of a group is \emph{not} a subgroup?
Here are three possible ways, any one of which guarantees that the subset is not a subgroup.
\begin{enumerate}\mathlist
    \item Show that the identity is not in the set.
    \item Exhibit an element of the set whose inverse is not in the set.
    \item Exhibit two elements of the set whose product is not in the set.
\end{enumerate}

When dealing with finite groups, it is easier to use the following subgroup test.

\begin{theorem}[Finite Subgroup Test]
Let $H$ be a nonempty finite subset of a group $G$.
If $H$ is closed under the operation of $G$, then $H$ is a subgroup of $G$.
\end{theorem}

\begin{proof}
In view of the Two-Step Subgroup Test, we need only prove that $a^{-1} \in H$ whenever $a \in H$.
If $a = e$, then $a^{-1} = a$ and we're done.
If $a \neq e$, consider the sequence $a,a^{2},\ldots$.
By closure, all of these elements belong to $H$.
Since $H$ is finite, not all of these elements are distinct.
Say $a^{i} = a^{j}$ and $i > j$.
Then $a^{i-j} = e$; and since $a \neq e$, $i -j > 1$.
Thus, $aa^{i-j-1} = a^{i-j} = e$ and therefore, $a^{i-j-1} = a^{-1}$.
But $i - j - 1 \geq 1$ implies $a^{i-j-1} \in H$ and we are done.
\end{proof}

\digsubsubsec{Examples of Subgroups}{6}{19}{18}{examplesofsubgroups}
For any element $a$ from a group, let $\langle a \rangle$ denote the set $\{a^{n} \mid n \in \Z \}$.
Note that the exponents of $a$ include all negative integers as well as $0$ and the positive integers ($a^{0}$ is defined to be the identity).

\begin{theorem}[$\bm{\langle a \rangle}$ is a Subgroup]
Let $G$ be a group, and let $a$ be any element of $G$.
Then $\langle a \rangle$ is a subgroup of $G$.
\end{theorem}

\begin{proof}
Since $a \in \langle a \rangle$, $\langle a \rangle$ is not empty.
Let $a^{n}, a^{m} \in \langle a \rangle$.
Then, $a^{n}(a^{m})^{-1} = a^{n - m} \in \langle a \rangle$; so by the One-Step Subgroup Test, $\langle a \rangle$ is a subgroup of $G$.
\end{proof}

The subgroup $\langle a \rangle$ is called the \emph{cyclic subgroup of $G$ generated by $a$}.
In the case that $G = \langle a \rangle$, we say that $G$ is \emph{cyclic} and $a$ is a \emph{generator of $G$}.
(A cyclic group may have many generators.)
Note that although the list $\ldots,a^{-2},a^{-1},a^{0},a^{1},a^{1},\ldots$ has infinitely many entries, the set $\{ a^{n} \mid n\in\Z\}$ might have only finitely many elements.
Also note that, since $a^{i}a^{j} = a^{i+j} = a^{j}a^{i}$, every cyclic group is Abelian.

\end{document}
