\documentclass[../butidigress.tex]{subfiles}

\begin{document}

\chapter{Just Random Bullshit\ldots}
\newpage

This appendix serves two purposes.
The first is as a area for my thoughts on subjects that come to my mind.
Notes made regarding said subjects will live here until I can figure out how to integrate them into the main matter of the book.
The second is for stuff that really will never serve any purpose.
For example, the random formal definitions of the pumping lemmas.

\section{Random Thoughts on Speech}\margintodo{try to incorporate this into \say{Knowledge} (\ref{chap:knowledge})}
Our (humanity's) ability to describe is, as far as is currently known, unique amongst animals, or at least, we are by far the best at descriptive language.
Other species, most or all of them, have the ability to speak, some form of language exists.
However, they deal purely within the scope of the prescriptive or the immediate, e.g., \say{look out for that lion!} or \say{there are some ripe-ass berries over there}.
They don't recall.
They don't tell they're friends what they did the preceding day.

We (\textit{Homo sapiens}) do.

Not only do we flap our gums (and twirl our pens and poke our keyboards) regarding things that have happened, we communicate ideas and concepts that have no basis in reality (hot take, religion\ftnote{Slightly less hot take, nations, but I'll get into that a bit later.}). In \book{Sapiens}, it is asserted that this ability is our most fundamental property, what renders \emph{Homo sapiens} alone alongside all other animals\autocite{sapiens}.
It has been found, independently (I think, haven't checked) that \emph{Homo sapiens'} superlative pattern recognition (superior pattern processing) is a foundational element of language.

An anecdote, from Barthes: Karl von Frisch, sought to prove the commonly held hypothesis that bees had some form of language.
He succeeded in this pursuit, but found that their language contained no description, it was all prescriptive; they had to coordinate in order to get their food, they didn't talk about what they were doing or had done, if such information was not imminently necessary for survival.
Basically, we're the only animals that talk about shit that doesn't really matter/exist.

\margindate{6}{14}{18}
Benedict Anderson wrote at some length about this concept, albeit in an decidedly less biological way, in his seminal work: \book{Imagined Communities}.
Originally published in 1983, \book{Imagined Communities} explores nationalism, its rise and its effects.
While Anderson didn't really come up with the concept, it's generally acknowledge that he coined the term.
What it boils down to is the fact that a \emph{nation} doesn't actually exist, it's just that everybody agrees that it's a thing\autocite{imaginedcommunities}.
Other animals have packs, but those max out at like, fifty, human beings are unique in our ability to construct communities of millions.

\margindate{6}{15}{18}
Going down to a deeper level, that of the brain specifically, we can attribute this uniquely human ability to our ability to recognize and process patterns\autocite{spphumans}.
On a certain level, this superior pattern processing (SPP) is the \emph{key} trait that differentiates humans from all other animals.
Well, referring to the first paragraph, it's a pretty simple syllogism\ftnote{If not entirely logically sound.}: our ability to describe separates us from other animals; our ability to describe comes from SPP; SPP separates us from other animals.

\digsec{Google Memo Thoughts}{6}{14}{18}{googlememo}
(I don't really know where to put this in the main text, or if I should at all, so it's going here for now)\margintodo{probably can fit this into \say{Empathy} (\ref{chap:empathy})}

I fairly recently (not that recently at all) read that stupid memo by that big dumb idiot that used to work for Google.
By my tone, I'm assuming one can guess my reaction to said memo\ftnote{It was, bar none, the closest I've come to putting my fist through my computer screen; my hand was only stayed by my knowledge of the cost, and the futility of the act.}; it was pretty much a whole load of bullshit and, frankly, it was borderline insulting, not just to over 50\% of the population, but to me.

First, it's important to say that my feelings on this matter are a bit irrelevant, I'm not disadvantaged by gender-biased hiring practices.
But, I am commenting, so here we go.

As a sort-of left-brained\ftnote{I'm not actually all that logical in my life, but I understand logic and can think through problems and spot fallacies, it's pretty much my training as an engineer.} person, there's not much more infuriating than the flawed appeal to \textit{logos}.
There are so many ridiculous attempts at a \say{logical} explanation for why women don't have jobs in tech and the holes are big enough to fly through with a C-130.
For example, Damore points out, that many cultures throughout world history have had similarly gendered divisions in society.
So fucking what?
A shit-ton of societies also had slaves, was that cool too?
Like, there's a bunch of stuff like that, where he points out the fact that \say{everybody does it, so it's okay,} a point which is obviously asinine.

Okay, to be fair\ftnote{Not that this dude deserves fairness.}, the schoolyard justification isn't stated explicitly.
It's introduced as evidence for the \say{biological} difference between men and women, but that doesn't make the point any less fallacious\ftnote{Side note, horrible word, just sounds way too close to another, way worse to say around mixed company word.}.
There are, many, practices spanning across cultures that are not because of \say{genetics.}
If I've gotten anything out of all my reading on semiotics it's the power and pervasiveness of ingrained, social behavior.

\newpage
\digsec{Toxic (Male) Nerd Culture}{6}{16}{18}{toxicnerddfa}
Here's a graph that I made that tracks the construction of Toxic Nerd Culture, which is, in by construction, inherently male.

\begin{figure*}[htp]
\centering
\begin{tikzpicture}[->,>=stealth',shorten >=1pt,auto,node distance=4cm,semithick]
\tikzstyle{every state}=[fill=white,text=black]
\node[initial,state]   (B)              {\scriptsize Barrier to Entry};
\node[state]           (H) [below of=B] {\scriptsize Homogeneity};
\node[state] (T) [below of=H] {\scriptsize Toxic Nerd Culture};
\node[state]           (P) [left of=H]  {\scriptsize Prior Exclusion};

\path (B) edge                    node {}                (H)
      (H) edge                    node {}                (T)
      (P) edge [left,bend right]  node {reactionary}     (T)
      (T) edge [right,bend right] node {structural bias} (H);
\end{tikzpicture}
\end{figure*}

\section{Formal Statements}
\subsection{The Pumping Lemma for Regular Languages}
\newcommand{\tab}{\hspace*{2em}}
\begin{align*}
&(\forall L\subseteq \Sigma^{\ast}) \\
&\tab (regular(L) \Rightarrow \\
&\tab ((\exists p\geq 1)((\forall w\in L)((|w| \geq p) \Rightarrow \\
&\tab ((\exists x,y,z \in \Sigma^{\ast}) \\
&\tab (w=xyz \wedge (|y| > 0 \wedge |xy|\leq p \wedge (\forall i \geq 0)(xy^{i}z\in L))))))))
\end{align*}

\subsection{The Pumping Lemma for Context Free Languages}
\begin{align*}
&(\forall L\subseteq \Sigma^{\ast}) \\
&\tab (context\ free(L) \Rightarrow \\
&\tab ((\exists p\geq 1)((\forall w\in L)((|w| \geq p) \Rightarrow \\
&\tab ((\exists u,v,x,y,z \in \Sigma^{\ast})(w=uvxyz \wedge (|vy| > 0 \wedge |vxy|\leq p\ \wedge \\
&\tab\tab (\forall i \geq 0)(uv^{i}xy^{i}z\in L))))))))
\end{align*}


\newtheorem{theorem}{Theorem}[subsection]
\newtheorem*{corollary}{Corollary}
\newtheorem{lemma}[theorem]{Lemma}

\theoremstyle{definition}
\newtheorem{definition}{Definition}[subsection]


\digsec{Abstract Algebra Notes}{6}{16}{18}{abstractalgebra}
This section is basically just me copying \book{Contemporary Abstract Algebra}.

\begin{definition}{Well Ordering Principle}
The well ordering principle states that every nonempty set of positive integers contains a smallest number.
\end{definition}

A nonzero integer $t$ is a \emph{divisor} of an integer $s$ if there is an integer $u$ such that $s=tu$.
Write $t\mid s$, \say{$t$ divides $s$.}

When $t$ is not a divisor of $s$, write $t\nmid s$.

A \emph{prime} is a positive integer greater than $1$ whose only positive divisors are $1$ and itself.

An integer $s$ is a \emph{multiple} of an integer $t$ if there is an integer $u$ such that $s=tu$ or, equivalently, if $t\mid s$.

\begin{theorem}[Division Algorithm]
Let $a$ and $b$ be integers with $b>0$.
Then there exist unique integers $q$ and $r$ with the property that $a=bq+r$, where $0\leq r<b$.
\end{theorem}

\begin{definition}{Greatest Common Divisor}
The greatest common divisor of two nonzero integers $a$ and $b$ is the largest of all common divisors of $a$ and $b$.
Denote this integer by $\gcd(a, b)$
\end{definition}

\begin{definition}{Relatively Prime}
When two nonzero integers $a$ and $b$ have $\gcd(a,b)=1$ they are relatively prime.
\end{definition}

\begin{theorem}[GCD is a Linear Combination]
For any nonzero integers $a$ and $b$, there exist integers $s$ and $t$ such that $\gcd(a,b) = as+bt$.
Moreover, $\gcd(a, b)$ is the smallest positive integer of the form $as + bt$.
\end{theorem}

\begin{corollary}
If $a$ and $b$ are relatively prime, then there exist integers $s$ and $t$ such that $as+bt=1$.
\end{corollary}

\begin{lemma}[Euclid's Lemma]
If $p$ is a prime that divides $ab$ then $p\mid a$ or $p\mid b$.
\end{lemma}

\end{document}
