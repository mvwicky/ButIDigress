%!TEX program = lualatex
%!TEX root = ./butidigress.tex
\documentclass[10pt,fleqn,titlepage]{book}
\setlength{\paperheight}{9in}
\setlength{\paperwidth}{6in}
\usepackage{marginnote}
\usepackage[top=0.75in,bottom=1.3in,outer=1.1in,inner=0.7in,marginparwidth=2cm,marginparsep=3mm]{geometry}

\usepackage[T1]{fontenc}

\usepackage{iftex}
\ifLuaTeX
\usepackage{fontspec}
\setmainfont{Roboto}[
    Extension=.ttf,
    Path=./font/Roboto/,
    UprightFont=*-Regular,
    BoldFont=*-Black,
    ItalicFont=*-Italic,
    BoldItalicFont=*-BlackItalic
]
\setmonofont{RobotoMono}[
    Extension=.ttf,
    Path=./font/Roboto_Mono/,
    UprightFont=*-Regular,
    BoldFont=*-Bold,
    ItalicFont=*-Italic,
    BoldItalicFont=*-BoldItalic
]
\else
\usepackage[utf8]{inputenc}
\usepackage{lmodern}
\fi

\usepackage{titling}

\usepackage[english]{babel}
\usepackage{csquotes}

\usepackage[backend=biber,citestyle=numeric,bibstyle=numeric]{biblatex}
\addbibresource{philo.bib}

\usepackage{indentfirst}
\usepackage{float}
\usepackage[symbol*]{footmisc}
\usepackage{amsmath,amsfonts,amssymb}
\usepackage{bm}
\usepackage[dvipsnames,svgnames]{xcolor}
\usepackage{mathrsfs,mathtools}
\usepackage{multicol}
\usepackage[super,negative]{nth}
\usepackage{setspace}
\usepackage{textcomp}
\usepackage{graphicx}
\usepackage{fancyvrb,fancyhdr}
\usepackage[plain]{fancyref}
\usepackage{titlesec}
\usepackage{epigraph}
\usepackage{subfiles}

\usepackage{verse}
\usepackage{dramatist}

\usepackage[unicode]{hyperref}
\usepackage{bookmark}
\hypersetup{colorlinks=false,linkbordercolor=white,pdfborder=0 0 0}

\graphicspath{ {images/} }

\everymath{\displaystyle}

\DeclareMathOperator{\sinc}{sinc}
\newcommand{\boph}[1]{\emph{\textbf{#1}}} % bold and emph
\newcommand{\defItem}{\itemsep1pt \parsep0pt \parskip0pt}
\newcommand{\tbul}{\textbullet}

\newcommand{\minus}{\scalebox{0.8}{$-$}}
\newcommand{\plus}{\scalebox{0.6}{$+$}}

\newcommand{\oneover}[1]{\frac{1}{#1}}

\newcommand{\br}[1]{\ensuremath{\left\{ #1 \right\}}}
\newcommand{\adpar}[1]{\ensuremath{\left( #1 \right)}}

\renewcommand{\topfraction}{0.9}
\renewcommand{\bottomfraction}{0.9}

\setcounter{topnumber}{2}
\setcounter{bottomnumber}{2}
\setcounter{totalnumber}{4}
\setcounter{dbltopnumber}{2}

\renewcommand{\dbltopfraction}{0.9}
\renewcommand{\textfraction}{0.07}

\renewcommand{\floatpagefraction}{0.7}
\renewcommand{\dblfloatpagefraction}{0.7}

\renewcommand{\thesubsection}{\thesection.\arabic{subsection}}
\renewcommand{\thesubsubsection}{\thesection.\thesubsection.\roman{subsubsection}}

\newcommand{\impword}[1]{\textcolor{red}{\textbf{#1}}}
\newcommand{\movie}[1]{{\itshape #1}}
\newcommand{\tvshow}[1]{{\itshape #1}}

\newcommand{\unnumchapter}[1]{\chapter*{#1}\markright{#1}\addcontentsline{toc}{chapter}{#1}}
\newcommand{\unnumsection}[1]{\section*{#1}\markright{#1}\addcontentsline{toc}{section}{#1}}
\newcommand{\unnumsubsection}[1]{\subsection*{#1}\markright{#1}\addcontentsline{toc}{subsection}{#1}}


\setlength{\fboxsep}{5pt}
\newcommand{\textbox}[1]{\vspace{0em}\noindent\hfil\fbox{\parbox[c][1.1\height][c]{0.9\textwidth}{\small\scshape {#1}}}\vspace{1em}}

% Division Formatting/Styling
\addtolength{\headheight}{\baselineskip}

\renewcommand*{\marginfont}{\scriptsize}

\renewcommand{\chaptermark}[1]{\markboth{#1}{}}
\renewcommand{\sectionmark}[1]{\markright{#1}}
\renewcommand{\subsectionmark}[1]{\markright{#1}}

% Some Epigraph Stuff
\renewcommand{\epigraphflush}{center}
\renewcommand{\epigraphsize}{\small}
\setlength{\epigraphwidth}{0.8\textwidth}
\newcommand{\attrib}[3]{\textsc{#1}, \textit{#2} (#3)}

% Space between title elements
\newlength{\titlevspace}
\setlength{\titlevspace}{1cm}

\newcommand*{\customtitlepage}{%
\begin{center}\scshape
\begin{spacing}{2.0}
    \vspace{3cm}
    {\huge But I Digress \ldots} \\
    \vspace*{\titlevspace}
    {\large An Ethical Codification for Exactly One Person (Me)} \\
\end{spacing}
{\vspace*{\fill}}
{\large\bfseries Michael Van Wickle} \\
\end{center}}

\newcommand*{\posttitlepage}{%
\vspace*{0.15\paperheight}
{\hfil\footnotesize\ttfamily \underline{Last Compiled: \today}\hfil}
\vspace*{\fill}}

\newcommand{\poemauthor}[1]{\nopagebreak{\centering\small\textsc{#1}\par}}

\newcommand{\margincol}[2][red]{\marginnote{\textcolor{#1}{#2}}}
\newcommand{\marginred}[1]{\marginnote{\textcolor{red}{#1}}}

% Commentary on this particular section
\newcommand{\margincomm}[1]{\marginnote{\scshape\textcolor{MidnightBlue}{#1}}}
% TODO note
\newcommand{\margintodo}[1]{\marginnote{\bfseries\textcolor{red}{TODO: #1}}}
% A last modifed date
\newcommand{\margindate}[1]{\reversemarginpar\marginnote{\textcolor{Olive}{#1}}\normalmarginpar}
% A random thought
\newcommand{\marginthought}[1]{\marginnote{\itshape\textcolor{Goldenrod}{#1}}}

\newcommand{\wrt}{w.r.t.}


\pagecolor{WhiteSmoke}

\title{But I Digress \ldots}
\author{Michael J. Van Wickle}
\date{Spring 2018}

\begin{document}
\frontmatter
\pagestyle{empty}

% Title Page
\begin{titlepage}
\customtitlepage
\end{titlepage}

% Post-Title Page
\posttitlepage
\newpage

\setfnsymbol{wiley}
\renewcommand{\thefootnote}{\fnsymbol{footnote}}
\pagestyle{fancy}
\fancyhf{}
\renewcommand{\chaptermark}[1]{\markboth{#1}{}}
\renewcommand{\sectionmark}[1]{\markright{#1}}
\fancyhead[ER]{\nouppercase{\rightmark}}
\fancyhead[OL]{\nouppercase{\leftmark}}
\fancyfoot[EL,OR]{\thepage}

% \setcounter{page}{0}
\setlength{\parskip}{\baselineskip}
\newcommand{\editorsnote}{Disclaimer}
{\centering\bfseries\Large \editorsnote}
\markright{\editorsnote}\addcontentsline{toc}{chapter}{\editorsnote}

\vspace*{\fill}
{\normalsize Some additional disclaimers will certainly be necessary overall, but a blanket one is clearly necessary up front.
The first issue I'd grapple with here is the fairly obvious, rather damning influence on this work by one David Foster Wallace.
His influence on this work is a testament to complacency and hypocrisy.}

{\normalsize There will be more of these later on, but the editors feel the need to have a blanket one up front.
This life philosophy has been much influenced by David Foster Wallace, acclaimed author, we find this problematic.
The author, personally, finds his general outline of irony in media and its potential toxicity compelling and worthy of thought and adoption.
He feels that media\footnote{movies, TV shows, books, etc.} which exhibits the traits outline by Mr. Wallace in such writings as \textit{Infinite Jest}\autocite{infjest}, \textit{E Unibus Pluram}, and \textit{This is Water} should be shown adulation.

We would like to acknowledge here the widespread adoption of Wallace's school of thought by the less-than morally upright or well-meaning.\autocite{elonwallace}\margincol{the actual problem is his treatment of woman and our collective acceptance}
Such people, generally being of the faux-intelligentsia, faux-deep, God's-gift-to-the-world-generally type, are not whom the author wished to associate himself.
Despite this, we would like to simply float the possibility that the author is actually one of those people who is in deep denial of his belonging to that category, because who truly knows his-or-her-self.

We would also, while we have your attention, to point our the absurdity of this document existing at all.
Why on goddamn earth, for what reason in the freakin world, would a person who is twenty-three, just graduated college (not even yet), need to write what is effectively a memoir?
I, oops, we, mean, what else to call this?
It's a fuckin' memoir, like, conceited much?\margincol{}}

\noindent\hspace*{2em} ---The Editors
\vspace*{\fill}

\setlength{\parskip}{0em}

\tableofcontents
\newpage
\titleformat{\chapter}[frame]{\Large}{\filright\enspace\thechapter\enspace}{2.5cm}{\Large\bfseries\filcenter}
\setlength{\parskip}{0.25em}

\setcounter{section}{1}

\subfile{sections/introduction}

\mainmatter
\renewcommand{\thefootnote}{\arabic{footnote}}

\subfile{sections/knowledge}

\subfile{sections/sincerity}

\subfile{sections/empathy}

\subfile{sections/immutable}

\subfile{sections/thevoid}

\part*{Appendices}
\markright{Appendices}\addcontentsline{toc}{part}{Appendices}
\appendix
\subfile{sections/appendix}

\newpage
\backmatter

\titleformat{\chapter}[hang]{\Large}{\filright\enspace\thechapter\enspace}{2.5cm}{\Large\bfseries\filcenter}
\markright{Bibliography}
\addcontentsline{toc}{chapter}{Bibliography}
\printbibliography

\newpage
\markright{Typeface}
\vspace*{0.15\paperheight}
\begin{center}\bfseries
About the Typeface (From Google)
\end{center}
\noindent\hfil\parbox[c][1.1\height][c]{0.5\textwidth}{\small\scshape
Roboto has a dual nature.
It has a mechanical skeleton and the forms are largely geometric.
At the same time, the font features friendly and open curves.
While some grotesks distort their letterforms to force a rigid rhythm, Roboto doesn’t compromise, allowing letters to be settled into their natural width.
This makes for a more natural reading rhythm more commonly found in humanist and serif types.}
\vspace*{\fill}

\end{document}