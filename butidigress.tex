%!TEX program = lualatex
%!TEX root = ./butidigress.tex
\documentclass[10pt,fleqn,titlepage]{book}
\setlength{\paperheight}{9in}
\setlength{\paperwidth}{6in}
\usepackage{marginnote}
\usepackage[top=0.75in,bottom=1.3in,outer=1.1in,inner=0.7in,marginparwidth=2cm,marginparsep=3mm]{geometry}

\usepackage[T1]{fontenc}

\usepackage{iftex}
\ifLuaTeX
\usepackage{fontspec}
\setmainfont{Roboto}[
    Extension=.ttf,
    Path=./font/Roboto/,
    UprightFont=*-Regular,
    BoldFont=*-Black,
    ItalicFont=*-Italic,
    BoldItalicFont=*-BlackItalic
]
\setmonofont{RobotoMono}[
    Extension=.ttf,
    Path=./font/Roboto_Mono/,
    UprightFont=*-Regular,
    BoldFont=*-Bold,
    ItalicFont=*-Italic,
    BoldItalicFont=*-BoldItalic
]
\else
\usepackage[utf8]{inputenc}
\usepackage{lmodern}
\fi

\usepackage{titling}

\usepackage[english]{babel}
\usepackage{csquotes}

\usepackage[backend=biber,citestyle=numeric,bibstyle=numeric]{biblatex}
\addbibresource{philo.bib}

\usepackage{indentfirst}
\usepackage{float}
\usepackage[symbol*]{footmisc}
\usepackage{amsmath,amsfonts,amssymb}
\usepackage{bm}
\usepackage[dvipsnames,svgnames]{xcolor}
\usepackage{mathrsfs,mathtools}
\usepackage{multicol}
\usepackage[super,negative]{nth}
\usepackage{setspace}
\usepackage{textcomp}
\usepackage{graphicx}
\usepackage{fancyvrb,fancyhdr}
\usepackage[plain]{fancyref}
\usepackage{titlesec}
\usepackage{epigraph}
\usepackage{subfiles}

\usepackage{verse}
\usepackage{dramatist}

\usepackage[unicode]{hyperref}
\usepackage{bookmark}
\hypersetup{colorlinks=false,linkbordercolor=white,pdfborder=0 0 0}

\graphicspath{ {images/} }

\everymath{\displaystyle}

\DeclareMathOperator{\sinc}{sinc}
\newcommand{\boph}[1]{\emph{\textbf{#1}}} % bold and emph
\newcommand{\defItem}{\itemsep1pt \parsep0pt \parskip0pt}
\newcommand{\tbul}{\textbullet}

\newcommand{\minus}{\scalebox{0.8}{$-$}}
\newcommand{\plus}{\scalebox{0.6}{$+$}}

\newcommand{\oneover}[1]{\frac{1}{#1}}

\newcommand{\br}[1]{\ensuremath{\left\{ #1 \right\}}}
\newcommand{\adpar}[1]{\ensuremath{\left( #1 \right)}}

\renewcommand{\topfraction}{0.9}
\renewcommand{\bottomfraction}{0.9}

\setcounter{topnumber}{2}
\setcounter{bottomnumber}{2}
\setcounter{totalnumber}{4}
\setcounter{dbltopnumber}{2}

\renewcommand{\dbltopfraction}{0.9}
\renewcommand{\textfraction}{0.07}

\renewcommand{\floatpagefraction}{0.7}
\renewcommand{\dblfloatpagefraction}{0.7}

\renewcommand{\thesubsection}{\thesection.\arabic{subsection}}
\renewcommand{\thesubsubsection}{\thesection.\thesubsection.\roman{subsubsection}}

\newcommand{\impword}[1]{\textcolor{red}{\textbf{#1}}}
\newcommand{\movie}[1]{{\itshape #1}}
\newcommand{\tvshow}[1]{{\itshape #1}}

\newcommand{\unnumchapter}[1]{\chapter*{#1}\markright{#1}\addcontentsline{toc}{chapter}{#1}}
\newcommand{\unnumsection}[1]{\section*{#1}\markright{#1}\addcontentsline{toc}{section}{#1}}
\newcommand{\unnumsubsection}[1]{\subsection*{#1}\markright{#1}\addcontentsline{toc}{subsection}{#1}}


\setlength{\fboxsep}{5pt}
\newcommand{\textbox}[1]{\vspace{0em}\noindent\hfil\fbox{\parbox[c][1.1\height][c]{0.9\textwidth}{\small\scshape {#1}}}\vspace{1em}}

% Division Formatting/Styling
\addtolength{\headheight}{\baselineskip}

\renewcommand*{\marginfont}{\scriptsize}

\renewcommand{\chaptermark}[1]{\markboth{#1}{}}
\renewcommand{\sectionmark}[1]{\markright{#1}}
\renewcommand{\subsectionmark}[1]{\markright{#1}}

% Some Epigraph Stuff
\renewcommand{\epigraphflush}{center}
\renewcommand{\epigraphsize}{\small}
\setlength{\epigraphwidth}{0.8\textwidth}
\newcommand{\attrib}[3]{\textsc{#1}, \textit{#2} (#3)}

% Space between title elements
\newlength{\titlevspace}
\setlength{\titlevspace}{1cm}

\newcommand*{\customtitlepage}{%
\begin{center}\scshape
\begin{spacing}{2.0}
    \vspace{3cm}
    {\huge But I Digress \ldots} \\
    \vspace*{\titlevspace}
    {\large An Ethical Codification for Exactly One Person (Me)} \\
\end{spacing}
{\vspace*{\fill}}
{\large\bfseries Michael Van Wickle} \\
\end{center}}

\newcommand*{\posttitlepage}{%
\vspace*{0.15\paperheight}
{\hfil\footnotesize\ttfamily \underline{Last Compiled: \today}\hfil}
\vspace*{\fill}}

\newcommand{\poemauthor}[1]{\nopagebreak{\centering\small\textsc{#1}\par}}

\newcommand{\margincol}[2][red]{\marginnote{\textcolor{#1}{#2}}}
\newcommand{\marginred}[1]{\marginnote{\textcolor{red}{#1}}}

% Commentary on this particular section
\newcommand{\margincomm}[1]{\marginnote{\scshape\textcolor{MidnightBlue}{#1}}}
% TODO note
\newcommand{\margintodo}[1]{\marginnote{\bfseries\textcolor{red}{TODO: #1}}}
% A last modifed date
\newcommand{\margindate}[1]{\reversemarginpar\marginnote{\textcolor{Olive}{#1}}\normalmarginpar}
% A random thought
\newcommand{\marginthought}[1]{\marginnote{\itshape\textcolor{Goldenrod}{#1}}}

\newcommand{\wrt}{w.r.t.}


\pagecolor{WhiteSmoke}

\title{But I Digress\ldots}
\author{Michael J. Van Wickle}
\date{Spring 2018}

\begin{document}
\frontmatter
\pagestyle{empty}

% Title Page
\begin{titlepage}
\customtitlepage
\end{titlepage}

% Post-Title Page
\posttitlepage
\newpage

\DefineFNsymbols{custom}{{\normalsize *}\maltese\textdollar{\normalsize\bfseries\textdagger}\textparagraph\pounds\#\ddag{\normalsize **}\S{\textdollar\textdollar}{\textdagger\textdagger}{\ddag\ddag}{\maltese\maltese}{\pounds\pounds}{\textparagraph\textparagraph}{\#\#}{\normalsize ***}{\S\S}{\textdollar\textdollar\textdollar}{\scshape a}{\scshape b}}
\setfnsymbol{custom}
\renewcommand{\thefootnote}{\fnsymbol{footnote}}

\pagestyle{fancy}
\fancyhf{}
\renewcommand{\headrule}{}
\renewcommand{\chaptermark}[1]{\markboth{#1}{}}
\renewcommand{\sectionmark}[1]{\markright{#1}}
\fancyhead[C]{\nouppercase{\footnotesize\itshape\rightmark}}
\fancyfoot[ER,OL]{\nouppercase{\itshape\leftmark}}
\fancyfoot[EL,OR]{\thepage}

\setlength{\parskip}{0.5\baselineskip}
\newcommand{\editorsnote}{Disclaimer}
{\centering\bfseries\Large \editorsnote}
\markright{\editorsnote}\addcontentsline{toc}{chapter}{\editorsnote}\label{chap:disclaimer}

\vspace*{\fill}
{\normalsize Some additional disclaimers will certainly be necessary overall, but a blanket one is clearly necessary up front.
The first issue I'd grapple with here is the fairly obvious, rather damning influence on this work by one David Foster Wallace.
His influence on this work is a testament to complacency and hypocrisy.
Complacency in choosing the easy way out; hypocrisy in criticizing those who have done the same.
(See \fref{chap:immutable} on \say{The Immutable.})

I'll deal with this later on in the introduction (\fref[vario]{sec:influences}), but I'd like to get ahead of it here.
As of this writing (\today), I'm still struggling internally with the implications of my life being influenced by a man who's behavior I have rejected without equivocation in other men.
I can't be entirely sure when I first heard the allegations\ftnote{the first inkling of something untoward around David Foster Wallace was almost certainly in that Jesse Eisenberg/Jason Segel movie about David Foster Wallace}---which incidentally are not even really allegations at this point, there's pretty much consensus on this---I've known about them and have chosen to practice the intentional ignorance all too common in our current society.\ftnote{further, I have read at least one article on the problems that come from messianic figures\autocite{elonwallace}}
Ignorance, intentional and otherwise, provides sanctuary to those privileged enough to escape atrocity.
Being a cis-het, partially-white male, much shelter is available to me.
The ideal version of myself would not avail myself of this shelter, instead seeking to help others, but alas I failed in this instance.

During my drive down to Florida, I many a car-hour thinking about my moral quandary.\ftnote{additionally struggling with the fact that I even have the ability to consider this}
I have come to the---likely fallacious---conclusion that the philosophy I have derived\marginthought{Some would say that you stole it directly} from Wallace's various writing is, to a hopefully acceptable extent, separable from the man.
This is something that has proven difficult to logically square with my fundamental belief in the inescapable interconnection between artist and art.
For an elucidation of that struggle see the influences section of the introduction (page \pageref{sec:influences}).

I'd also, while I have your attention, to point our the absurdity of this document existing at all.
(Apologies for the tonal shift all the way over to flip.)
Why on goddamn earth, for what reason in the freakin world, would a person who is twenty-three, just graduated college (not even yet), need to write what is effectively a memoir?
I mean, what else to call this?
Calling it a \say{Personal Philosophy} or a \say{Set of Moral Codes} is borderline (or far) worse.

In all seriousness, I truly do want to write down, to codify the lessons of the last half-decade plus years of my life.
But there isn't really any way to do that without feeling a bit like a self-indulgent doucher.
It's a fuckin' memoir, like, conceited much?}

\noindent\hspace*{2em} ---Michael J. Van Wickle
\vspace*{\fill}
\newpage

\setcounter{footnote}{0}
\newcommand{\disctwo}{Comments on the Tone (Disclaimer No. 2)}
{\centering\bfseries\Large \disctwo}
\markright{\disctwo}\addcontentsline{toc}{chapter}{\disctwo}\label{chap:disctwo}

{\normalsize I'd imagine that you thought the book would be starting right about now, or like, maybe some front matter stuff would be here, instead I've decided to include a second disclaimer.
From here, I'd like to just say some words regarding the tone of this thing, because it has been a real struggle figuring out where to go with it\ftnote{or if I should put, frankly, any consideration into tone, I mean the title suggests a level of rambling that would be consistent with a meandering tone}.

There are moments, a great deal of moments\ftnote{for instance, right now}, where I believe the language used and my attitude to be overly flip.
Such irreverence has its place, but within a deeply personal document outlining some serious life events, it may be inappropriate.}

\noindent\hspace*{2em} ---Once again,

\noindent\hspace*{2em} Michael J. Van Wickle
\vspace*{\fill}

\setlength{\parskip}{0in}

\newpage

\tableofcontents
\newpage
\titleformat{\chapter}[frame]{\Large}{\filright\enspace{\Huge\thechapter}\enspace}{2.5cm}{\Large\bfseries\filcenter}

\titleformat{\section}[display]{\Large}{\thesection\filright}{0cm}{\bfseries\filcenter}
\titleformat{\subsection}[display]{\large}{\thesubsection}{0cm}{\bfseries\filcenter}
\titleformat{\subsubsection}[display]{\small}{\thesubsubsection}{0cm}{\bfseries\filcenter}
\setlength{\parskip}{0.4em}

\setcounter{footnote}{0}
\setcounter{section}{1}

\subfile{introduction}

\mainmatter
\renewcommand{\thefootnote}{\arabic{footnote}}

\subfile{sections/empathy}

\subfile{sections/immutable}

\subfile{sections/knowledge}

\subfile{sections/sincerity}

\subfile{sections/thevoid}

\part*{Appendices}
\markright{Appendices}\addcontentsline{toc}{part}{Appendices}
\appendix
\subfile{sections/appendix}

\newpage
\backmatter

\titleformat{\chapter}[hang]{\Large}{\filright\enspace\thechapter\enspace}{2.5cm}{\Large\bfseries\filcenter}
\markright{Bibliography}
\addcontentsline{toc}{chapter}{Bibliography}
\printbibliography

\newpage
\markright{Typeface}
\vspace*{0.15\paperheight}
\begin{center}\bfseries
About the Typeface\ftnote{from Google}
\end{center}
\noindent\hfil\parbox[c][1.1\height][c]{0.5\textwidth}{\small\scshape
Spectral is a new and versatile serif face available in seven weights of roman and italic, with small caps.
Spectral offers an efficient, beautiful design that’s intended primarily for text-rich, screen-first environments and long-form reading.}
\vspace*{\fill}

\newpage
\titleformat{\chapter}[frame]{\Large}{\filright\enspace\thechapter\enspace}{1.5cm}{\Large\bfseries\filcenter}
\chapter{Random Stats}
Some totals of the various notes we've seen throughout this book.
Really just added them for fun.
\begin{itemize}\scshape\large
    \item Commentary Notes: \thecommcounter
    \item Todo Notes: \thetodocounter
    \item Thought Notes: \thethoughtcounter
    \item Date Notes: \thedatecounter
    \item Footnotes: \thefootcounter
    \item Quotes (main text): \theepicounter
\end{itemize}

\end{document}