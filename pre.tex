\documentclass[10pt,titlepage]{book}
\usepackage[protrusion=true,expansion]{microtype}
\usepackage{calc}
% The dimensions of `Brandeis' by Melvin I. Urofsky (2012)
\setlength{\paperheight}{9.2in}
\setlength{\paperwidth}{6.3in}
\usepackage{marginnote}
\usepackage[
top=0.7in,
bottom=1.0in,
outer=0.6in,
inner=0.8in,
marginparwidth=0.5in,
marginparsep=0.2in,
showframe]{geometry}

\usepackage[T1]{fontenc}

\usepackage{iftex}
\ifLuaTeX%
\usepackage{polyglossia}
\setmainlanguage[variant=american]{english}
\usepackage{fontspec}
\setmainfont{Spectral}[
    Extension=.ttf,
    Path=./font/Spectral/,
    UprightFont=*-Regular,
    BoldFont=*-Bold,
    ItalicFont=*-Italic,
    BoldItalicFont=*-BoldItalic,
    Ligatures=TeX,
    Numbers={Proportional}
]
\setmonofont{Hack}[
    Extension=.ttf,
    Path=./font/Hack/,
    UprightFont=*-Regular,
    BoldFont=*-Bold,
    ItalicFont=*-Italic,
    BoldItalicFont=*-BoldItalic,
    Scale=MatchLowercase,
    Ligatures=TeX
]
\else
\usepackage[english]{babel}
\usepackage[utf8]{inputenc}
\usepackage{lmodern}
\fi

\usepackage{lettrine}

\usepackage{siunitx}

\usepackage{titling}

% Provides, inter alia, `\LuaLaTeX` command
\usepackage{metalogo}

% Provides the displayquote environment and other quotation tools
% TODO: replace uses of `\say` with `\enquote`
\usepackage{csquotes}

% Puts the Table of Contents in the ToC
\usepackage{tocbibind}

% Bibiliography Support
\usepackage[
backend=biber,
block=nbpar,
sorting=ynt,
citestyle=verbose-trad1,
ibidpage=true,
backref=true,
bibstyle=numeric]{biblatex}
\addbibresource{butidigress.bib}

% Makes the first paragraph indented
\usepackage{indentfirst}
\usepackage{float}
\usepackage{setspace}
% Footnote utilities
% TODO: Figure out why `multiple' doesn't work
\usepackage[%
stable,     % deals with footnotes in titles
multiple,   % separates footnotes on the same character
bottom,     % forces footnotes to the bottom of the page
flushmargin % puts the footnote marker flush with, but just inside, the margin
]{footmisc}

% Math Packages
% ams*: standard American Mathematical Society packages
% bm: bold text math
% mathrsfs: provides `\mathscr` command
% mathtools: misc. math utilities
\usepackage{amsmath,amsfonts,amssymb,amsthm,bm,mathrsfs,mathtools}

% Provides a bunch of color utilities
\usepackage[dvipsnames,svgnames]{xcolor}
% Multicolumn environments
\usepackage{multicol}
% Gives the `\nth` command for superscript ordinals
\usepackage[super,negative]{nth}
% ?
\usepackage{textcomp}
% Image handling (not used as of now)
\usepackage{graphicx}
\usepackage{fancyvrb,fancyhdr}
\usepackage[plain]{fancyref}
% clearempty makes automatic even pages empty
% nobottomtitles* moves sections close to the bottom of the page to the next page
% calcwidth adds the \textwidth length
\usepackage[clearempty,nobottomtitles*,calcwidth]{titlesec}
\usepackage[dotinlabels]{titletoc}
% Package for formatting epigraphs
\usepackage{epigraph}
% Package for splitting doc into files
\usepackage{subfiles}

\usepackage{tikz}
\usetikzlibrary{arrows,automata,graphdrawing,graphs,quotes}
\usegdlibrary{phylogenetics,layered,trees}

% Packages for typesetting poems and excerpts from screenplays
\usepackage{verse,dramatist}

% Package for customizing lists
\usepackage{enumitem}
% Package allowing for underlining
\usepackage[normalem]{ulem}
\usepackage{tabu}

% \usepackage[unicode]{hyperref}
\usepackage{bookmark}
