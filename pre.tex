\documentclass[10pt,titlepage]{book}
\usepackage[protrusion=true,expansion]{microtype}
\usepackage{calc}
\setlength{\paperheight}{9.8in}
\setlength{\paperwidth}{6.9in}
\usepackage{marginnote}
\usepackage[
top=0.75in,
bottom=1.0in,
outer=1.0in,
inner=0.7in,
marginparwidth=0.9in,
marginparsep=0.1in]{geometry}

\usepackage[T1]{fontenc}

\usepackage{iftex}
\ifLuaTeX%
\usepackage{polyglossia}
\setmainlanguage[variant=american]{english}
\usepackage{fontspec}
\setmainfont{Spectral}[
    Extension=.ttf,
    Path=./font/Spectral/,
    UprightFont=*-Regular,
    BoldFont=*-Bold,
    ItalicFont=*-Italic,
    BoldItalicFont=*-BoldItalic,
    Ligatures=TeX,
    Numbers={Proportional}
]
\setmonofont{Hack}[
    Extension=.ttf,
    Path=./font/Hack/,
    UprightFont=*-Regular,
    BoldFont=*-Bold,
    ItalicFont=*-Italic,
    BoldItalicFont=*-BoldItalic,
    Scale=MatchLowercase,
    Ligatures=TeX
]


\else
\usepackage[english]{babel}
\usepackage[utf8]{inputenc}
\usepackage{lmodern}
\fi

\usepackage{titling}

\usepackage{metalogo}

% Provides the displayquote environment
\usepackage{csquotes}
% Provides the \say command
% \usepackage{dirtytalk}

% Puts the Table of Contents in the ToC
\usepackage{tocbibind}

\usepackage[backend=biber,citestyle=verbose-ibid,bibstyle=numeric]{biblatex}
\addbibresource{butidigress.bib}

% Makes the first paragraph indented
\usepackage{indentfirst}
\usepackage{float}
\usepackage[stable,multiple,bottom]{footmisc}
\usepackage{amsmath,amsfonts,amssymb,amsthm}
% Bold in math
\usepackage{bm}
\usepackage[dvipsnames,svgnames]{xcolor}
\usepackage{mathrsfs,mathtools}
\usepackage{multicol}
\usepackage[super,negative]{nth}
\usepackage{setspace}
\usepackage{textcomp}
\usepackage{graphicx}
\usepackage{fancyvrb,fancyhdr}
\usepackage[plain]{fancyref}
% clearempty makes automatic even pages empty
% nobottomtitles* moves sections close to the bottom of the page to the next page
% calcwidth adds the \textwidth length
\usepackage[clearempty,nobottomtitles*,calcwidth]{titlesec}
\usepackage[dotinlabels]{titletoc}
% Package for formatting epigraphs
\usepackage{epigraph}
% Package for splitting doc into files
\usepackage{subfiles}
\usepackage{ellipsis}

\newcommand{\say}[1]{\enquote{#1}}

\renewcommand{\ellipsisgap}{0.2em}

\newcommand{\lips}{\kern\ellipsisgap .\kern\ellipsisgap .\kern\ellipsisgap .\kern\ellipsisgap}

\usepackage{tikz}
\usetikzlibrary{arrows,automata,graphdrawing,graphs,quotes}
\usegdlibrary{phylogenetics,layered,trees}

% Packages for typesetting poems and excerpts from screenplays
\usepackage{verse,dramatist}

% Package for customizing lists
\usepackage{enumitem}
% Package allowing for underlining
\usepackage[normalem]{ulem}

\usepackage[unicode]{hyperref}
\usepackage{bookmark}
\hypersetup{%
colorlinks=false,
linkbordercolor=white,
pdfborder=0 0 0,
bookmarksnumbered=true,
bookmarksopen=true,
bookmarksopenlevel=2,
pdftitle={But I Digress\ldots},
pdfsubject={philosophy},
pdfauthor={Michael Van Wickle},
pdfkeywords={philosophy,life,empathy,knowledge,absurdism,sincerity}}

\graphicspath{{images/}}

\everymath{\displaystyle}

\newlength{\defparskip}
\setlength{\defparskip}{0.4em}

\DeclareMathOperator{\sinc}{sinc}
\def\defItem{\itemsep1pt \parsep0pt \parskip0pt}
\newcommand{\tbul}{\textbullet}

\newcommand{\minus}{\scalebox{0.8}{$-$}}
\newcommand{\plus}{\scalebox{0.6}{$+$}}

\newcommand{\oneover}[1]{\frac{1}{#1}}

\newcommand{\br}[1]{\ensuremath{\left\{ #1 \right\}}}
\newcommand{\adpar}[1]{\ensuremath{\left( #1 \right)}}

% Float Setup
\renewcommand{\topfraction}{0.9}
\renewcommand{\bottomfraction}{0.9}

\setcounter{topnumber}{2}
\setcounter{bottomnumber}{2}
\setcounter{totalnumber}{4}
\setcounter{dbltopnumber}{2}

\renewcommand{\dbltopfraction}{0.9}
\renewcommand{\textfraction}{0.07}

\renewcommand{\floatpagefraction}{0.7}
\renewcommand{\dblfloatpagefraction}{0.7}

% Semantic commands for different media
\newcommand{\impword}[1]{\textcolor{red}{\textbf{#1}}}
\newcommand{\movie}[1]{\textit{#1}}
\newcommand{\tvshow}[1]{\textit{#1}}
\newcommand{\tv}[1]{\tvshow{#1}}
\newcommand{\book}[1]{\textit{#1}}
\newcommand{\podcast}[1]{\textit{#1}}

% Convenience commands for unnumbered divisions
\newcommand{\unnumchapter}[1]{\cleardoublepage\phantomsection\chapter*{#1}\markright{#1}\addcontentsline{toc}{chapter}{#1}}
\newcommand{\unnumsection}[1]{\section*{#1}\phantomsection\addcontentsline{toc}{section}{#1}\markright{#1}}
\newcommand{\unnumsubsection}[1]{\subsection*{#1}\phantomsection\addcontentsline{toc}{subsection}{#1}\markright{#1}}

\setlength{\fboxsep}{5pt}
% Provides a block of text with a box around it
\newcommand{\textbox}[1]{\vspace{0em}\noindent\hfil\fbox{\parbox[c][1.1\height][c]{0.9\textwidth}{\small\textsc{#1}}}\vspace{1em}}

% Set margin note size
\renewcommand*{\marginfont}{\scriptsize}

\renewcommand{\chaptermark}[1]{\markboth{#1}{}}
\renewcommand{\sectionmark}[1]{\markright{#1}}
\renewcommand{\subsectionmark}[1]{\markright{#1}}

% Some Epigraph Stuff
% Center epigraphs (default is flushright)
\renewcommand{\epigraphflush}{center}
% Make epigraphs smal
\renewcommand{\epigraphsize}{\small}
\setlength{\epigraphwidth}{0.8\textwidth}
% Convenience command for attribution of quotes
\NewDocumentCommand\attrib{mmm}{\textsc{#1}, \textit{#2} (#3)}

% Count all the epigraphs (for the Random Stats page)
\newcounter{epicounter}
\NewDocumentCommand\epi{mm}{\epigraph{#1}{#2}\stepcounter{epicounter}}

% Space between title elements
\newlength{\titlevspace}
\setlength{\titlevspace}{1cm}

\newcommand*{\customtitlepage}{%
\begin{center}\scshape
\begin{spacing}{2.0}
    \vspace{3cm}
    {\huge But I Digress\ldots} \\
    \vspace*{\titlevspace}
    {\large An Ethical Codification for Exactly One Person (Me)} \\
\end{spacing}
{\vspace*{\fill}}
{\large\bfseries Michael Van Wickle} \\
\end{center}}

\newcommand*{\posttitlepage}{%
\begin{center}\small
\begin{spacing}{2.0}
\vspace*{\stretch{1}}
\underline{\texttt{Last Compiled: \today}} \\
\underline{\texttt{Compiled With} \LuaLaTeX}
\vspace{\stretch{3}}\end{spacing}
\end{center}}


\newcommand{\poemauthor}[1]{\nopagebreak{\centering\small\textsc{#1}\par}}

% Counter for footnotes
\newcounter{footcounter}
% Footnote command
\newcommand{\ftnote}[1]{\footnote{#1}\stepcounter{footcounter}}
% Step footcounter on citations
\AtEveryCite{\stepcounter{footcounter}}

% Commentary on this particular section
\newcounter{commcounter}
\newcommand{\commcolor}{MidnightBlue}
\newcommand{\commstyle}{\scshape}
\newcommand{\margincomm}[1]{\marginnote{\singlespace\commstyle\textcolor{\commcolor}{#1}}\stepcounter{commcounter}}
% TODO note
\newcounter{todocounter}
\newcommand{\todocolor}{red}
\newcommand{\todostyle}{\bfseries}
\newcommand{\margintodo}[1]{\marginnote{\singlespace\todostyle\textcolor{\todocolor}{TODO:~#1}}\stepcounter{todocounter}}
% A last modifed date
\newcounter{datecounter}
\newcommand{\datecolor}{Olive}
\newcommand{\datestyle}{\ttfamily}
\NewDocumentCommand\fmtdate{mmm}{#1/#2/#3}
\NewDocumentCommand\fmttime{mmm}{#1:#2 #3}
\NewDocumentCommand\margindate{mmm}{\reversemarginpar\marginnote{\datestyle\textcolor{\datecolor}{\fmtdate{#1}{#2}{#3}}}\normalmarginpar\stepcounter{datecounter}}
\NewDocumentCommand\margintime{mmm}{\reversemarginpar\marginnote{\datestyle\textcolor{\datecolor}{\fmttime{#1}{#2}{#3}}}\normalmarginpar\stepcounter{datecounter}}
% A random thought
\newcounter{thoughtcounter}
\newcommand{\thoughtcolor}{Sepia}
\newcommand{\thoughtstyle}{\itshape}
\newcommand{\marginthought}[1]{\marginnote{\singlespace\thoughtstyle\textcolor{\thoughtcolor}{#1}}\stepcounter{thoughtcounter}}

% Inserts monospaced text inside <>
\newcommand{\todobrak}[1]{\texttt{\textcolor{\todocolor}{<{#1}>}}}

\newcommand{\wrt}{w.r.t.}

% Skip between two paragraphs that are relevant enough to be in the same section, but a thematic break is still needed
\newcommand{\entryskip}{\vspace*{\baselineskip}}

\newcommand{\notescolor}{teal}
\newenvironment{somenotes}[1]{\par\entryskip\noindent{\Large\centering\textsc{Some Notes On: #1}}\begin{itemize}[leftmargin=0pt,parsep=1pt,itemsep=1pt]\color{\notescolor}}{\end{itemize}}

\NewDocumentCommand\ddate{O{\the\day}O{\the\month}O{\the\year}}{#1/#2/#3}

% Custom sectioning commands which take a date as an input
\NewDocumentCommand\digsec{mmmmm}{\section{#1}\margindate{#2}{#3}{#4}\label{sec:#5}}
\NewDocumentCommand\digsubsec{mmmmm}{\subsection{#1}\margindate{#2}{#3}{#4}\label{subsec:#5}}
\NewDocumentCommand\digsubsubsec{mmmmm}{\subsubsection{#1}\margindate{#2}{#3}{#4}\label{subsubsec:#5}}
