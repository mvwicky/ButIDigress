%!TEX root = ../philo.tex
\documentclass[../philo.tex]{subfiles}
\begin{document}
\chapter{Introduction}
\epigraph{I have but one lamp by which my feet are guided, and that is the lamp of experience. I know of no way of judging the future but by the past.}{\attrib{Patrick Henry}{speech at the Virginia Convention}{1775}}
\newpage

% Why am I writing this?
This document is an attempt to layout the personal philosophy that I have developed over the course of my life so far.
While realizing that I have only lived 23 years, quite a short time in the scheme of things, I believe that documenting my thoughts will serve me to reflect on from whence I came and to meditate on where I am going.
I intend this to be some manner of living document, updated over time with my new insights.

This introduction exists to define some terms and as exposition regarding the development of my philosophy.
We start our journey with a discussion of ethics vs. morals and follow that up with a (mostly) chronological account of my journey to this point.
Just before getting to the meat of my elucidation I'll mention some influences.

\unnumsection{Ethics}
Personal philosophy is a bit abstract, I would, to a certain degree, prefer the term \emph{ethical code}.
What does that mean exactly?
I assume that we all have a passing familiarity with the term ethics, but this usage may be slightly opaque.
I think a precise definition of the term ethical (or ethics) would be appropriate here, I am using a definition set forth in \textit{Objectivity}, by Lorraine Daston and Peter Galison (2007).
\begin{description}
    \item [ethics] normative codes of conduct that are bound up with a way of being in the world, an ethos in the sense of the habitual disposition of an individual or group
    \item [morals] specific normative rules that may be upheld or transgressed and to which one may be held to account
\end{description}

I really doubt that these are very many people's working definitions of these terms.
It doesn't really matter in this case, because I'm just defining these terms to mean those definitions in this specific context.
Once you're done reading this (this is a sentence which will likely only ever be read by me) you can go back to whatever normal definition you have, but for now use these.

I have no real moral code (by the above definition), besides, like, the law, I guess; I'm saying that I don't believe in God or religion.
My ethics are founded purely on choice, or they try to be, and I don't think there will be any reward for following them or a punishment for breaking them.

Removing incentive, I have found (along with many behavioral psychologists), tends to make one less likely to do some thing.
The question that comes to mind is: why continue to follow?

The answer is in two parts, neither of which are likely to mollify current skeptics.
First, this list is, at least in part, descriptive, it's how I have lived my life (which has worked out pretty well for me so far).
The second part is really a rejection of the question; living this way has made my life better, more fulfilling.

\unnumsection{Development}
\epigraph{Constant and frequent questioning is the first key to wisdom\ldots\ For through doubting we a led to inquire, and by inquiry we perceive the truth.}{\attrib{Peter Abelard}{Sic et Non}{~1121}}
I started college pretty goddamn depressed; for about two years, I thought about my suicide more like a historical event than a potential future choice.
I started this document near the end of my \nth{5} year of college, I am now without suicidal thoughts (for all intents and purposes---I'm not gonna say I'm perfect).

When I first decided to write this, I figured it might be beneficial or focusing to write down how I want to ideally live my life.
After talking to people about my task, I decided to modify my purpose.
The original goal remains: a codification of a lifestyle; a new goal enters: a reflection of how I got from preordained death to a healthy(er) human being.
What this document has become for me is an accounting of how I survived, an existential \textit{Robinson Crusoe} if one wanted to be full of oneself.
The tenets enumerated below represent the codification of life changes I made in order to, in the most literal sense, save my own life.
They are a product of years of hard work; I introspected and I studied and I made friends and I sought help.

There is something, quite a lot actually, to be said for getting help if you need it.
Don't ever be too proud to get help, it's not even a matter of pride.
The stigmatization of mental health is dangerous and buying into it can be deadly.
Learning that there isn't anything wrong with me was groundbreaking and is to a large degree the reason that these rules exist.

\unnumsection{Influences}
The commencement address delivered by David Foster Wallace on May 21, 1995 at Kenyon College, commonly referred to as \textit{This is Water}, has had a great deal of influence on my life.
Despite the frequent co-opting by somewhat douchey faux-ally types of Wallace's legacy his general philosophy resonates with me.\footnote{I fully recognize the high probability that this sentence is pretty much completely ironic and represents a pretty damning failure of introspection.}
In particular, his rejection of the irony and cynicism of postmodernism in favor of his own, sincere style is something that I fully embrace.
Despite the fact that I've never actually finished reading \textit{Infinite Jest} and that really the only work of his I've finished is \textit{Consider the Lobster}, I consider him to be among my most significant influences.


Another influence for the items listed here is Albert Camus, specifically absurdism.
I don't embrace absurdism wholeheartedly; there were some strikingly problematic elements in \textit{The Myth of Sisyphus}, what I do take from Camus' philosophy is the desperate, futile struggle against the void.
This fight, that unwinnable, yet cosmically noble struggle against the unknowable appeals to me.

Not all of my influences are now deceased well-known acclaimed authors; the next influence I'll talk about is actually a duo: Freddie Wong and Matthew Arnold.
(The acclaimed authors bit isn't intended as a slight, it was just funny, so shut the fuck up or something.)
Sincerity is one of the core virtues I'll be exploring later on, and my fascination with the concept stems in large part from these two.
I've been watching freddiew video, I assume, since I first went on YouTube, a lot of his early work is pretty ubiquitous on the World Wide Web.
Wong and Arnold also produced \textit{Video Game High School}, which I enjoyed immensely and their Hulu series was a good spot of fun as well.
Their podcast, \textit{Story Break}, is almost certainly my favorite podcast, but their true contribution to my life comes in introducing me to sincerity.\footnote{I don't mean sincerity the concept, obviously I know what that is. They pointed out sincerity in media, most memorably \movie{Speed Racer}.}

I'll not get too deep into it here, as there is a whole chapter later on dedicated to it, but I, from them, realized that it was the reason I enjoyed certain works.
The first example of this is \textit{Speed Racer}, my favorite movie (well, tied for first\footnote{With \movie{Short Term 12} which---you know what, we'll get into it later.}), but I also see it in their works.
\end{document}