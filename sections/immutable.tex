%!TEX program = lualatex
%!TEX root = ../butidigress.tex
\documentclass[../butidigress.tex]{subfiles}
\begin{document}
\chapter{The Immutable}\label{chap:immutable}
\epigraph{Reality is that which, when you stop believing in it, doesn't go away.}{\attrib{Phillip K. Dick}{How to Build a Universe That Doesn't Fall Apart Two Days Later}{1978}}
\newpage
\margintodo{my thoughts on this are strained, b/c I don't necessarily accept absolutes, but also apparently I can stomach immutability? maybe this is more like a day-to-day thing, whereas the other thing is more philosophical/existential}
Complaining, expressing one's discontent, is not only natural, but essential.
This does not enjoin complaints generally, but only when they are \emph{against what one cannot affect}.

I'd just like to take a moment here to talk about the quote at the head of this chapter.\margindate{5/1/18}
The search for this quote took a long, long, long-ass time, like, a truly disturbing amount of time.
For quite a while it simply said \say{Some quote}, attributed to \say{Some person.}
Phillip K. Dick, for those who don't know, was a prolific science-fiction writer, particularly notable for the number of movie/television adaptations of his work; he wrote \textit{Total Recall} and \textit{Do Androids Dream of Electric Sheep?}, which was adapted into \movie{Blade Runner}.
His \textit{bona fides} are stated to show that he has some experience in world building, in constructing realities.
I chose this quote, I like this quote, because, obviously, it represents the spirit of this chapter; it also ties into, reminds me, of concepts from \textit{Objectivity}.

\section{Identifying Immutability}
As will be explored later on, in the chapter on knowledge (\fref{chap:knowledge} starting on page \pageref{chap:knowledge}), I don't really believe in absolute truths, i.e., absolute knowledge about anything.
So how, one may ask, can I posit that any situation is fully unchangeable?
The answer lies in scope and in consequence.

Whilst, yes, I don't subscribe to absolute truths, only \say{asymptotic truths\ftnote{see \fref{sec:asymptotic}},} that applies mostly for big picture things.
What this chapter deals with is quite a bit closer to Earth, i.e., everyday situations.

A whole lot of this actually is about dealing with members of the service industry.
Like, dealing with bureaucratic nonsense is a pretty big part of the modern lifestyle, so being able to recognize that the person in front of you really has no control over the inanity of the whole process, and that raging at them about their being a gear in the damned machine, and that generally making a scene serves no constructive purpose and will just needlessly tire you out and, more importantly, make someone's day worse, ends up being a pretty good skill.\margindate{6/3/18}
This often works in concert with empathy\ftnote{as discussed in chapter \ref{chap:empathy}}, as generally a key step in realizing immutability is recognizing the personhood of the service worker to whom you're talking.

\section{Corollary: The Mutable}
Arising naturally\ftnote{or not, I have really no idea what people ever mean when they say that} from the concepts outlined above, one should not complain (about the mutable) without \emph{attempting to effect change}.

\margintodo{talk about self esteem being a factor in not wanting to do things}I have found this hard to put into practice, as I am naturally quite averse to going outside and/or expending energy.

\end{document}