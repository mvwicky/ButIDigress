\documentclass[./butidigress.tex]{subfiles}

\begin{document}
\chapter{From Where I (We) Came}\label{chap:fromwhereicame}
\newpage

Somewhere around \num{13.799+-0.021} billion years ago,\autocite[32]{planckcollab} space and time began, with the Big Bang.\ftnote{This bout be literal (and long) a.f. (See \fref{chap:knowledge})}
Soon after,\ftnote{Like on the order of \SI{e-33}{\second} later} the four fundamental forces of the Standard Model\ftnote{Strong, Weak, Electromagnetic, and Gravitation} emerged.
Three hundred and eighty thousand years after that,\ftnote{In that \num{380000} years, the universe cooled down, allowing quarks to go from a quark-gluon plasma to forming hadrons, antimatter was annihilated, and neutrinos stopped interacting with normal matter} electrons started binding to atomic nuclei.\ftnote{Which came into existence earlier, but it was too hot for electrons to bind to nuclei.}
Over one and a half million years later, stars started forming.
From there, galaxies, including but not limited to the Milky Way, and galaxy clusters began to take shape.
Fast-forwarding a bit\lips

Some \num{4.54+-0.04} billion years ago, the Earth was formed, coalescing out of the same cloud of dust and gas (nebula) from which the Sun, the other planets, and the Solar System generally were formed.
In the center of the nebula,\ftnote{Where the angular momentum was lowest, i.e., where centripetal force was highest ($F = mr\omega^{2}$), causing rapid compression leading to nuclear fusion.} the Sun was born; in a similar, but smaller scale, process planetary bodies began forming around areas of higher density.

Earth's history is split up, currently, into four eons, which are each divided into several eras, which are divided into periods, which are divided into epochs.
The four eons are the Hadean, the Archean, the Proterozoic, and the Phanerozoic, with the Hadean starting with Earth's formation and the Phanerozoic continuing today.

\setlength{\tabcolsep}{0.5em}
\tabulinesep=0.7em
\begin{table}[h]
\centering
\begin{tabu} to 0.75\textwidth {| X[1,r,m] | X[3,l,m] |}\hline
\multicolumn{2}{|c|}{\large\textbf{Eons of Earth}}             \\ \hline \hline
\textbf{Hadean}      & \numrange{4540}{4000} million years ago \\ \hline
\textbf{Archean}     & \numrange{4000}{2500} million years ago \\ \hline
\textbf{Proterozoic} & \numrange{2500}{542} million years ago  \\ \hline
\textbf{Hadean}      & \num{541} million years ago             \\ \hline
\end{tabu}
\end{table}

\digsec{Time Divided}{7}{27}{18}{geologicaltimescale}
One may

\digsec{Hadean: Hell is Earth}{7}{20}{18}{thehadeanera}
Lasting roughly \num{500000000} years, the Hadean featured an Earth which probably looked a bit closer to how one imagines Hell looking like than to the current Earth.
In fact, the word \enquote{Hadean} comes from Hades, the Greek god of the underworld.\ftnote{I feel like he got a bad rap. Just getting dunked on by Zeus and Poseidon and having everyone hate you, just because you drew the short straw and got the downer gig, kinda sucks.}
We really don't know a whole lot about this period of time, it was pretty long ago.

Scientists think that we caught the Moon during the Hadean, based on the current theory.
The theory being that a big-ass (Mars sized) object (planet) hit Earth and the combined ejecta eventually coalesced into the Moon.\autocite{wheredidthemooncomefrom}
This object, generally referred to as Theia (formed in the same way the planets did), started out at Earth's $L_{4}$ or $L_{5}$, points which are gravitationally stable relative to Earth.\autocite[3--4]{wheredidthemooncomefrom}
Without taking into account the other bodies of the Solar System, this system would be stable and we'd have a fun sister planet, equally capable of sustaining life as Earth.
But, gravitational disturbance from other proto-planets caused Theia to impact the Earth, the debris coalescing into the Moon.\autocite[44--47]{wheredidthemooncomefrom}

Earth got hit by a bunch of stuff at this point, a period, lasting from the late Hadean to the early Archean, called the \emph{Late Heavy Bombardment}.\autocite{lunarbombardment}
Whether or not life may have started during this eon, there is scant evidence, is up for debate, but it certainly got going during the Archean.

\digsec{Archean: Water, Water, Everywhere, but no Oxygen}{7}{21}{18}{thearcheanera}
The term Archean comes from the Greek \emph{Arkh\={e}}, meaning \enquote{beginning, origin,} and the eon originally was the first, until the Hadean was split off into its own thing.\ftnote{The demarcation of different eons, eras, periods, and epochs are not exactly universally agreed upon, and as such, they are subject to change.}

While still quite hot, relative to current temperatures, the crust had cooled down by this point, leading to the formation of the first continents.
One theory describes a proto-continent, some \num{3100} million years ago, referred to as \enquote{Ur,} which consisted of what is now India, western Australia, and southern Africa,\autocite{historyofcontinents} while another posits a mass dubbed \enquote{Vaalbara,} some \num{3600} million years before present time, consisting of western Australia and southern Africa.\autocite{sequencestratigraphy}

As one might infer from the above, plate tectonics, similar to those present today, were in full swing.
The Late Heavy Bombardment continued, possibly contributing the the emergence of life.
The Moon was much closer to Earth than it is today, causing tides up to \num{1000} feet high.
Liquid water was present on Earth, but there was barely any free oxygen in the atmosphere.

An oxygen-less atmosphere may sound bad to you or me, but it represents the ideal for anaerobic life, like bacteria, or more generally, prokaryotes.\ftnote{Prokaryote, by the way, comes from the Greek words \emph{pro}, meaning \enquote{before,} and \emph{karyon}, meaning \enquote{nut or kernel,} apt seeing as they haven't nuclei, mitochondria, or any membrane-bound organelle.}
It is theorized that the last common ancestor to all life on earth lived in the Archean, around \numrange{3.5}{3.8} billion years ago.

Bacteria developed photosynthesis and started using ATP to generate energy, which survive to this day.

\digsec{Proterozoic: Life, Now Featuring Multiple Cells}{7}{21}{18}{theproterozoicera}
As we get closer to current time our knowledge of the time period gets, understandably, more concrete.
While the Proterozoic\ftnote{A word which comes from, obviously, the Greek \emph{protero-} meaning \enquote{former, earlier} and \emph{zoic-} meaning \enquote{animal, living being.}} is, on a human scale, pretty well in the past, its extant fossil record and better-preserved rock give us a peek into what it was like.

\begin{somenotes}{Proterozoic Eon}
    \item \num{2500} to \num{542} million years ago
    \item the time between the appearance of oxygen in the Earth's atmosphere and the appearance of the first complex life forms
\end{somenotes}

\digsubsec{Classism, but it's Taxonomy}{7}{21}{18}{classismtaxonomy}
From the Greek \emph{taxis}\ftnote{Meaning arrangement} and \emph{-nomia},\ftnote{Meaning method} biological taxonomy is \enquote{the science dealing with the description, identification, naming, and classification of organisms.}
The \enquote{father} of taxonomy is generally taken to be the Swedish botanist Carl Linnaeus (1707--1778).
Since Linnaeus' time, quite a bit of debate has taken place regarding the proper organization for the tree of life.\autocite{phylogenyandbeyond}
But, the generally accepted, or at least the one I will generally adhere to, is the one introduced by Carl Woese and George Fox in 1977, which features three \enquote{domains} as the highest division of life.\autocite{woeseorignial}

The three domains, according to Woese and Fox, are Archaea, Bacteria, and Eukarya (or Eykaryota).

\todobrak{explain different taxa}

\todobrak{transition into Phanerozoic}

\begin{somenotes}{Taxonomy}
    \item from the Greek \emph{taxis}, meaning \enquote{arrangement,} and \emph{-nomia}, meaning \enquote{method}
    \item Carl Linnaeus is generally regarded as its father
    \item lots of conflict and schools of thought, probably gonna stick to the Woese system\autocite{woeseorignial}
    \item Domain: highest level
    \begin{itemize}
        \item Archaea
        \item Bacteria
        \item Eukarya
    \end{itemize}
    \item Kingdom
    \item Phylum
    \item Class
    \item Order
    \item Family
    \item Genus
    \item Species
\end{somenotes}

\digsec{Phanerozoic: 2 Complex 5 Me}{7}{21}{18}{thephanerozoicera}

\begin{somenotes}{Phanerozoic Eon}
    \item the current eon, started \num{541} millions years ago
    \item began with the Cambrian period
\end{somenotes}

\digsubsec{The Evolution of Evolution}{7}{22}{18}{evolutionevolution}

\todobrak{discuss the history of the theory of evolution}\ftnote{\smartcite{zimmer2006evolution} and \smartcite{huxley2010evolution}}


\digsubsec{The Holocene Epoch}{7}{21}{18}{holoceneepoch}
The current period of time, the one we live in, is the Holocene\ftnote{From the Greek \emph{holos}, meaning \enquote{whole or entire,} and \emph{kainos}, meaning \enquote{entirely recent.}} epoch.\ftnote{And is, itself, split into smaller sections, called stages, but I'll not get into them too much. The current stage is the Meghalayan, but the concept of the \enquote{Anthropocene} has been discussed.}
The extinction event currently underway\lips

\digsec{The Point}{7}{21}{18}{thepoint}
\todobrak{the point should be clear throughout}

Then I was born, about thirteen billion years after the Big Bang.
I will resist my compulsion for analogy, for my life has been nothing like the early stages of the universe or of Earth, except in that I was once younger and now am older.

\todobrak{work in the Standard Model at some point}
\end{document}