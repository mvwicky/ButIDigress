\documentclass[./butidigress.tex]{subfiles}
\begin{document}
\unnumsection{Notes and Todo List}
\begingroup\color{MidnightBlue}
\begin{itemize}
    \item split this part into multiple files
    \item figure out analogies for all the different topics covered here, i.e., write with a purpose
    \item this is basically gonna end up kinda like \book{A Short History of Nearly Everything}, but with more of my personal views and opinions coming through
    \item \emph{But I Digress\dots} ends up being an appropriate title for my new vision of this book.
\end{itemize}

\unnumsubsection{\textcolor{black}{Topics to Cover}}
\color{MidnightBlue}
\begin{description}
    \item[Superior Pattern Processing (SPP)] Humans are the best at this, it's a big part of what makes us humans specifically, and it gives rise to things such as language and community (which stems from language)
    \item[Evolution] A history and the actual theory; original theory
    \item[The Standard Model] History and theory and competing theories
    \item[Physics Generally] Ancient philosophers to Newton to Einstein to Hawking; overview of how discoveries were made and thinking evolved; also explain things that were mentioned in the Big Bang section
    \item[Origins of Words] To be sprinkled throughout
    \item[Histories]
\end{description}

\endgroup

\newpage

\chapter{A Leap Into the Freezing Depths}\label{chap:aleap}
\epi{How does it happen that a properly endowed natural scientist comes to concern himself with epistemology? Is there no more valuable work in his specialty? I hear many of my colleagues saying, and I sense it from many more, that they feel this way. I cannot share this sentiment. When I think about the ablest students whom I have encountered in my teaching, that is, those who distinguish themselves by their independence of judgment and not merely their quick-wittedness, I can affirm that they had a vigorous interest in epistemology. They happily began discussions about the goals and methods of science, and they showed unequivocally, through their tenacity in defending their views, that the subject seemed important to them. Indeed, one should not be surprised by this.}{\attrib{Albert Einstein}{Physikalische Zeitschrift}{1916}}
\newpage
\digsec{Defining Genesis}{7}{20}{18}{whenwasgenesis}
\lettrine{W}{hen} I first conceived of writing this book\ftnote{I feel comfortable calling this a book because I actually have a firm idea of what it's about now.} my idea was of a broad codification of life's principles, from my point of view.
I wrote a pretty decent amount regarding that subject, but I came to realize, slowly, that textbook writing was not for me.
Because, that's what it was, a textbook, maybe a lively one, but still just a textbook, reference material.
What I needed was a throughline, something that would carry the book, give each chapter and section and paragraph and sentence a drive.

My first solution was a restructuring, I tried to pull a narrative out of what I had written.
To a certain extent, that's what this represents.
But along the way, I remembered \book{A Short History of Nearly Everything} by Bill Bryson.
\emph{That's} the kind of book I wanted to write: a meandering history, expounding on any and all subject that comes to mind, teach a little, preach a little.

\entryskip

My extreme narcissism leads me to still center this book around myself.
So the book will have a general skeleton, a pervasive motivation, centered around me, and my life.
I promise that it won't come up too much, except in the context of justifying what may seem like extreme turns in subject and\slash or tone.

If you can't think of where to start, start at the beginning.
I'd like to think of that as sage advice, but one just runs into another problem: where does anything even \enquote{begin?}
The canonical starting point that I've always used has been my first year at Northeastern, but that's not necessarily entirely appropriate.
If we're talking about when my depression \enquote{started,}\ftnote{To show its face at least.} then I guess it'd be my last couple years of high school.
But the seeds for my eventual downturn, and my recovery, were probably planted in junior high; I could make the argument that elementary school circumstances are partially to blame.

Following my current line of reasoning to its logical conclusion, do I start at my birth?
But the events leading to my current situation started long before that\lips do I start when my ancestors moved to America?
But events in their homelands still affect my life, clearly.
We can follow the chain of causality \textit{ad infinitum}.
So do I start at the big bang?
That certainly fills me with \textit{nauseam}.
But maybe it's a good \textit{nauseam}.

You know what?
I think will actually start there, I brought it up for the sake of hyperbole, but it represents the cleanest point of entry.

\subfile{sections/ocean/fromwherewecame}

\end{document}