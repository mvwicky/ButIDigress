%!TEX program = lualatex
%!TEX root = ../butidigress.tex
\documentclass[../butidigress.tex]{subfiles}
\begin{document}
\chapter{Empathy}\label{chap:empathy}
\epi{If it is not tempered by compassion, and empathy, reason can lead men and women into a moral void.}{\attrib{Karen Armstrong}{Twelve Steps to a Compassionate Life}{2010}}
\newpage

Well shit, I've arrived at the first chapter and everything has pretty much gone to shit.
My great plan for my great book was dashed, hoisted by my own petard---more specifically my own complacency.
Before I spiral, or, you know continue to spiral (I'll probably continue my Charybdis impression later), some explanation is necessary I guess.
It all starts in \nth{10} or \nth{11} grade\ftnote{I had the same English teacher both years and I'm not sure in which incarnation of her class I was introduced to \book{This is Water}}\ldots\margincomm{Here is the part where I deal with some undue influences (hopefully)}

Mrs. Nye, my eccentric, two-time English teacher, was the mediator for a lesson in empathy, and not being quite so sure of yourself.\margindate{5/12/18}
(Seriously though, her class was crazy, like getting into yelling arguments with her (and then the video getting posted on Facebook) and leaving in the middle of class to get donuts and returning to class.)
From that point on, with varying levels of compliance and emphasis, my life has been lived by a code that includes empathy as a core tenet.

\digsec{The Empathetic Mind}{5/12/18}{empatheticmind}
Very little in life is not the consequence of real and intensely conscious choice.
While hard, sometimes, to accept, it's true; instead of hoping you accept this fact \textit{a priori}, we'll run through a didactic example or two.\marginthought{or only one I guess}

Imagine a soldier, he is, reluctantly, straight up murdering civilians.
One of them, an old lady or a small child\ftnote{choose whichever one has more emotional weight for you, either work fine}, looks up at him/her with watering eyes and pleads for their life, asking: \say{why are you doing this?}
Voice unsteady, hands shaky, he replies \say{I have no choice. I am \texttt{<insert how he is being coerced>}.}
I don't feel overly denotative to say that he is wrong; he does have a choice, maybe the odds are stacked towards him choosing murdering people, but he chose, like, the life of his family over the people in front of him.
This is the most extreme example, besides Nazis\ftnote{although the soldier could easily be a Nazi or something} (the easiest logical extreme in any argument).

What does this have to do with empathy? good question.
The point I'm trying to make is that we have to choose to be empathetic, you are the master of your mind, although the human psyche is a stubborn beast.
It's hard work, sometimes you'll want to give up, sometimes you'll be tired, but making that choice is well worth it.

Why is it worth it?\margintodo{why is it worth it, but actually, figure that out}
While, \textit{prima facie}, asking why empathy is worth it seems at best na\"ive and at worst sociopathic, it is something worth exploring.
The answer depends on what one wants from life.

I have decided, some time ago, that I want to be happy and I want to be generally optimistic.\ftnote{as an engineer, sometimes optimism isn't exactly an option, compartmentalization is important}
Exercising empathy, consciously, helps me get there.\marginthought{part of the reason that I want to be empathetic is because I have trouble connecting with people}

Living without a higher power means that one gets to truly choose their virtues; I have chosen interpersonal connection as one of my central tenets.
Justifying morality without being able to appeal to some manner of God is, rather appropriately\ftnote{for me specifically I guess}, a Sisyphean task; there's really no getting around the fact that, cosmically, human life is meaningless and nothing one does to one's fellows matters at all.
One may find solace, or their justification, when considering the converse, that is to say, the concept of having morality without the all-powerful as a check.
It is obviously subjective whether or not this path is nobler\ftnote{I obviously have an opinion here, but what I think should not be seen as prescriptive or proscriptive}, to cling to human decency despite the reality of the abyss or to have decency forced upon oneself by fiat of a distant god\ftnote{for more in this vein, see~\ref{sec:thehundred}}.

\digsubsec{Empathy or Projection?}{5/9/18}{empathyorproj}
One must be wary of one's own brain, as it tends to conflate projection with empathy.
Such a situation is one in which I have found myself many a time, although I have likely had instances of further confusion.

The reason for confusion stems from the similarity of action, i.e., in both situations you're trying to get into another person's head, to know how they feel.
In the empathetic case, you read them, you come to an understanding of how they feel and why; in the other, you are reading the representation of the person which one's mind creates, it may feel like you are reading them, but in reality you are just pushing your thoughts into them.

Think extraction vs.\ inception in \movie{Inception}.\margincomm{if you hadn't noticed, movie references are my currency}
During the opening sequence of \movie{Inception}, Leo DiCaprio is going in the grab some memories from Ken Watanabe's brain.
He's not messing around in there, putting new thoughts or ideas in his head, he's getting to know what Ken knows.

On the flip side, when they're incepting Cillian Murphy, Leo is actively putting thoughts inside Cillian.

It ain't a perfect metaphor, but none are. I've made my feelings about metaphors emphatically in different fora\ftnote{spoiler alert, they're negative; it's a deployment problem, because every time I use a metaphor, I can see its holes and feel compelled to qualify that it (the metaphor) isn't perfect; no metaphor is perfect, if it was, it would just be the think you're talking about\ldots not sure why I thought this was salient information, or a thought that a human being should have at all}.

Projection is like if you incepted someone without knowing you were doing it, you were trying to run an extraction but accidentally incepted\ftnote{if that makes sense}.

\digsec{Identifying Empathy: \tvshow{The Office}}{5/9/18}{theoffice}
I had a realization while watching \tvshow{The Office}, re: how it expresses its empathy.
\tvshow{The Office} is truly representative of so much of what I'm trying to express in this section.
At first, I thought to myself, \say{the empathy presents itself in the character's true desire to connect with one another}, or something to that effect.
That sentiment is true, but empathy is shown in the deepest fiber of the show, specifically, in its most recognizable quality.

\tvshow{The Office} is awkward as hell.
Countless are the times I've had to pause and count to three or take a deep-ass breath\ftnote{deep ass-breath, a.k.a, butt inhale} before restarting because the cringe is so utterly overwhelming.
In so many of these situations, the cringe may be the product of a character trying to reveal their truest self in what may be an inadvisable manner or inappropriate context.
It's because they're trying, trying their damned heart out, to connect with their fellows or to represent themselves in a better light.
Since their urge is so plain, their sentiment so open, we tend to feel almost physical pain at just the possibility (in some instances, the inevitability) of rejection.

Empathetic pain comes from our innate sociability as human beings, it's in our DNA\@.
We want to be accepted by our peers, by those we admire, and the sight of someone falling on their face in the attempt is far too real.
The pain is, I feel, exacerbated by the rejection of genuine feelings exhibited in contemporary pop-culture.\ftnote{see, \book{E Unibus Pluram}}

An additional note is the way that the show handles Ryan, the B.J. Novak character.
At not point is there any attempt to show Ryan as actually cool, or like, a truth-teller type.
He's shown as he is: an asshole, cause a lot (or most) of the time, cynical dudes who think they they see through the veil are actually just fuckin' dicks.

\digsec{Sympathy}{6/7/18}{sec:empathysympathy}
\begin{description}
    \item [empathy] \textit{noun}, the ability to understand and share the feelings of another
    \item [sympathy] \textit{noun},

    \begin{enumerate}
        \item feelings of pity and sorrow for someone else's misfortune
        \item understanding between people; common feeling
    \end{enumerate}
\end{description}

\vspace{0.1cm}\noindent\hfill\rule{0.5\textwidth}{0.5pt}\hfill\vspace{0.5cm}

With all this discussion of empathy, one might wonder\ftnote{or like, not} where \emph{sympathy} is in all this.
There are issues that come up in one's peer's\ftnote{read: friend's} lives about which one truly cannot understand if one isn't experiencing them, the most obvious (and relevant) example being mental illness.

\end{document}