%!TEX root = ../butidigress.tex
\documentclass[../butidigress.tex]{subfiles}
\begin{document}
\chapter{Empathy}
\epigraph{If it is not tempered by compassion, and empathy, reason can lead men and women into a moral void.}{\attrib{Karen Armstrong}{Twelve Steps to a Compassionate Life}{2010}}
\newpage

Well shit, I've arrived at the first chapter and everything has pretty much gone to shit.
My great plan for my great book was dashed, hoisted by my own petard---more specifically my own complacency.
Before I spiral, or, you know continue to spiral (I'll probably continue my Charybdis impression later), some explanation is necessary I guess.
It all starts in \nth{10} or \nth{11} grade\footnote{I had the same English teacher both years and I'm not sure in which incarnation of her class I was introduced to \textit{This is Water}}\ldots\margincomm{Here is the part where I deal with some undue influences}

Mrs. Nye, my eccentric, two-time English teacher, was the mediator for a lesson in empathy, and not being quite so sure of yourself.\margindate{5/12/18}
(Seriously though, her class was crazy, like getting into yelling arguments with her (and then the video getting posted on Facebook) and leaving in the middle of class to get donuts and returning to class.)
From that point on, with varying levels of compliances and emphasis, my life has been lived by a code that includes empathy as a core tenet.

\section{The Empathetic Mind}\margindate{5/12/18}
Very little in life is not the consequence of real and intensely conscious choice.
While hard, sometimes, to accept, it's true; instead of hoping you accept this fact \textit{a priori}, we'll run through a didactic example or two.

Imagine a soldier, he is, reluctantly, straight up murdering civilians.
One of them, an old lady or a small child\footnote{choose whichever one has more emotional weight for you}, looks up at him/her with watering eyes and pleads for their life, asking: ``why are you doing this?''
Voice shaking, he replies ``I have no choice. I am \texttt{<insert how he is being coerced>}.''
It is not being overly denotative to say that he is wrong; he does have a choice, maybe the odds are stacked towards him choosing murdering people, but he chose, like, the life of his family over the people in front of him.
This is the most extreme example, besides Nazis\footnote{although the soldier could easily be a Nazi or something} (the easiest logical extreme in any argument).


\subsection{Empathy or Projection?}
One must be wary of one's own brain, as it tends to conflate projection with empathy.
Such a situation is one in which I have found myself many a time, although I have likely had instances of further confusion.

The reason for confusion stems from the similarity of action, i.e., in both situations you're trying to get into another person's head, to know how they feel.
In the empathetic case, you read them, you come to an understanding of how they feel and why; in the other, you are reading the representation of the person which one's mind creates, it may feel like you are reading them, but in reality you are just pushing your thoughts into them.

Think extraction vs. inception in \textit{Inception}.

\section{Aside: Recognizing Empathy}
I had a realization while watching \textit{The Office}, re: how it expresses its empathy.
\textit{The Office} is truly representative of so much of what I'm trying to express in this section.
At first, I thought to myself, ``the empathy presents itself in the character's true desire to connect with one another'', or something to that effect.
That sentiment is true, but empathy is shown in the deepest fiber of the show, specifically, in its most recognizable quality.

\textit{The Office} is awkward as hell.
Countless are the times I've had to pause and count to three or take a deep-ass breath\footnote{deep ass-breath, a.k.a, butt inhale} before restarting because the cringe is so utterly overwhelming.
In so many of these situations, the cringe may be the product of a character trying to reveal their truest self in what may be an inadvisable manner or inappropriate context.
It's because they're trying, trying their damned heart out, to connect with their fellows or represent themselves in a better light.
Since their urge is so plain, their sentiment so open, we tend to feel almost physical pain at just the possibility (in some instances, the inevitability) of rejection.

Empathetic pain comes from our innate sociability as human beings, it's in our DNA.
We want to be accepted by our peers, by those we admire, and the sight of someone falling on their face in the attempt is far too real.
The pain is, I feel, exacerbated by the rejection of real feeling exhibited in contemporary pop-culture.\footnote{see, \textit{E Unibus Pluram}}

An additional note is the way that the show handles Ryan, the B.J. Novak character.
At not point is there any attempt to show Ryan as actually cool, or like, a truth-teller type.
He's shown as he is: an asshole, cause a lot (or most) of the time, cynical dudes who think they they see through the veil are actually just fuckin' dicks.

\end{document}