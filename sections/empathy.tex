%!TEX root = ../philo.tex
\documentclass[../philo.tex]{subfiles}
\begin{document}
\chapter{Empathy}
\epigraph{The really important kind of freedom involves attention and awareness and discipline, and being able to truly care about other people and to sacrifice for them over and over in myriad petty, unsexy ways every day.}{\attrib{David Foster Wallace}{Kenyon Commencement Address}{2005}}
\newpage

\section*{Disclaimer}
\textbox{I get that the last two epigraphs were both David Foster Wallace, I'm not really trying to be a big ol' doucher here, so bear with me.
On reflection, I realize that I'm actually writing my ``personal philosophy'' so I guess I'm failing pretty hard here.
Just, I guess, please try to see me as a semi-non-douche for a little bit here.
Also, if you're reading this, then you are almost certainly me, so take it easy on yourself man.}


\section{Becoming Empathetic---The M.V.W. Story (unfinished)}
I don't relate to people very well, it's just a part of my life.
I've always been awkward, unable to converse.

\section*{Aside: Recognizing Empathy}
I had a realization while watching \textit{The Office}, re: how it expresses its empathy.
\textit{The Office} is, truly, representative of so much of what I'm trying to express in this section.
At first, I thought to myself, ``the empathy presents itself in the character's true desire to connect with one another'', or something to that effect.
That sentiment is true, but empathy is shown in the deepest fiber of the show, specifically, in its most recognizable quality.

The show is awkward as hell.
Countless times I've had to pause and count to three or take a deep-ass breath\footnote{deep ass-breath, a.k.a, butt inhale} before restarting because the cringe is so utterly overwhelming.
In so many of these situations, the cringe may be the product of a character trying to reveal their truest self in what may be an inadvisable manner or inappropriate context.
It's because they're trying, trying their damned heart out, to connect with their fellows or represent themselves in a better light.
Since their urge is so plain, their sentiment so open, we tend to feel almost physical pain at just the possibility (in some instances, the inevitability) of rejection.

Empathetic pain comes from our innate sociability as human beings, it's in our DNA.
We want to be accepted by our peers, by those we admire, and the sight of someone falling on their face in the attempt is far too real.
The pain is, I feel, exacerbated by the rejection of real feeling exhibited in contemporary pop-culture.\footnote{see, \textit{E Unibus Pluram}}

An additional note is the way that the show handles Ryan, the B.J. Novak character.
At not point is there any attempt to show Ryan as actually cool, or like, a truth-teller type.
He's shown as he is: an asshole, cause a lot (or most) of the time, cynical dudes who think they they see through the veil are actually just fuckin' dicks.

\section{Practicing Empathy}
This is the important part.

\end{document}