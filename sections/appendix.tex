%!TEX root = ../butidigress.tex
\documentclass[../butidigress.tex]{subfiles}
\begin{document}

\chapter{Road Trip 2018}
\newpage
This is a log of my 2018 road trip from Boston to Santa Barbara.
Starting on May 16th, the trip as planned will take about two weeks and will span the country.

\section*{A Note to My Friends}\marginthought{Friend represents far too banal terminology}
I used to tell people, truthfully, that I couldn't remember the last time I had been truly happy.

Instead I created a simulation of happiness, a smiling facade over anguish.

All too often I ignored, or chose to be ignored by, those who would look upon my true face unflinching.

Not with strained eyes of pity, nor cold eyes of disdain, but with warmth and compassion and empathy.

Only once I learned to simply open the door to those knocking, who wished only the simple pleasure of my company, did I experience true happiness.

\hspace*{2em} It is you all, who lent me strength when I was weak, courage when I was afraid, a smile when all I wanted to do was cry.

\hspace*{2em} For so much that I have, my life, my joy, I thank all of you.

\section*{The Night Before}\margindate{11:30PM}
I'm sitting here before my computer, nervous about the upcoming trip.
Well, not the trip necessarily, I'm excited about that, it's more the fact that I feel like I have so much to do tomorrow.

\renewcommand{\thesection}{Day \arabic{section}\ }

\section{\ May 16: Boston to Burlington}
Lying in bed.\margindate{6:45 AM}
More accurately, lying on my mattress, on the floor of my stressfully non-empty room.
Looking around I know that all evidence of me has to be completely gone by, like, noon, and I don't have a clear exit strategy for a lot of it.
For example, the current object of my repose, even a twin mattress is a fairly substantial thing of which to dispose.

In a weird way I'm not all that worried.

\vspace*{\baselineskip}

\margindate{10:00 AM}Just dropped off my last package at Fedex, now all I have to do is clean up and pack up the car.
First, however, I'm gonna take a little break (clearly, I'm writing this), sip my tea, and just wake up a little more.
It seems quite unfathomable that I'm gonna drive like four goddamn hours later today, but that's the way it is.

\vspace*{\baselineskip}

\margindate{6:39 PM}
I did end up driving like four goddamn hours.

Said four-ish hours\margincomm{More like three and a half} ended abruptly, when I tried to stop at a red light.
I heard a popping, banging sound from my car\footnote{always fun to hear}, and then when I tried to start driving again, there were some dope grinding noises.
Luckily, and seriously, this whole situation is unbelievably fortuitous, there was a repair place within walking distance, and they picked up my car, diagnosed it, and are in the process of fixing it.\footnote{it was a broken CV-joint for the record.}

While waiting for my car to get picked up, I made a reservation at a hotel (where I am currently) two miles from the repair place.
Moseying on back to the shop, I picked up my laptop, backpack, toiletries, and a change of clothes.
Being a smart person, I called an Uber to my hotel, is what I would have said if I were a smart person, instead I walked the two miles, in flip-flops, with my bags, after a long day already.

I don't know how I thought my first day of driving would go.
But not once did I envision walking across an overpass.

\section{\ May 17: Burlington to Ithaca}
\margindate{7:50 AM}It's weird how accustomed to things one's body can be.
Like, having slept in a twin bed my entire life, even the double that I slept in last night feels needlessly, luxuriously gargantuan.

Not feeling a whole lot of motivation to leave right now, but I must, must get breakfast, must get car.

Estimated drive time for today: six-ish hours\ldots

One last thought, the experience of writing everyday, or doing this actual journaling thing has been interesting.
Normally writing for me is like, synthesizing a whole mess of life experience down into something more abstract, applicable to a variety of situations.
The relative normality of this is strange, but I also feel like writing consistently, not just when the mood strikes, is the key to becoming a better writer.\footnote{which I want to become}

\vspace*{\baselineskip}

\margindate{10:21 AM}Just finished breakfast and am sitting in the same coffee shop that I went to yesterday\footnote{\textit{Uncommon Grounds}}.
Now I have to wait roughly an hour for my car to finished getting fixed.

As I was walking back from breakfast\footnote{at \textit{Handy's Lunch}}, I had a thought argument\footnote{super Stefon voice: it's that thing when you have an argument with an imaginary person in your head to help prove a point to yourself} about this, my, generation.
To be fair, I am at kinda the head of this generation, the actual group is the people just younger than me who grew up fully immersed in the World Wide Web.
Popular conceptions of this `Internet generation' seem to show it as self-centered, with heads buried in phones, and totally oblivious.
I could not disagree with the characterization more; this is honestly, an absurdly empathetic group of people.

Take, as an easy example, the rising enforcement of `PC' language; how is this anything but a profound empathetic response to the plight of the disparate?
For a cis-het (\textit{inter alia}) person to actively champion non-binary, `non-traditional' language is remarkable.

Also, about the phone thing, it's not like we're staring at our goddamn reflections, we're connecting with the world.

\vspace*{\baselineskip}

\margindate{7:20 PM}A pretty arduous day of driving done.

\vspace*{\baselineskip}

\margindate{8:26 PM}Bit of a false alarm there, took a shower and shaved.
Anyways, the day got started late, the mechanic took longer than I anticipated and there wasn't a clear time table.
So I walked around Burlington, well, I say around but I essentially walked in circles and slowly got more and more stressed.
I left around 1:30PM, which sucked; I was tired and freaking out a bit so the drive wasn't that great either.

An additional annoying element to leaving late was the inability to stop anywhere to take pictures.

My hotel in Ithaca\footnote{where I am right now} is no where near anything, not that I have the desire to go outside and do stuff right now.
Although I probably will end up running to Wegmans.


\renewcommand{\thesection}{\thechapter .\arabic{section}}

\newlength{\episkip}
\setlength{\episkip}{0.5cm}
\newcommand{\postepi}{\vspace{\episkip}\noindent\hfill\rule{0.5\textwidth}{1pt}\hfill\vspace{\episkip}}

\chapter{Quotations}
\newpage
% https://en.wikiquote.org/wiki/Truth
% https://en.wikiquote.org/wiki/Skepticism
% https://en.wikiquote.org/wiki/Knowledge
% https://en.wikiquote.org/wiki/Science
% https://en.wikiquote.org/wiki/History_of_science
% https://en.wikiquote.org/wiki/Atomic_theory
% https://en.wikiquote.org/wiki/Immanuel_Kant
% https://en.wikiquote.org/wiki/Malcolm_X
% https://en.wikiquote.org/wiki/Martin_Luther_King,_Jr.
Some quotations that I decided to not use as epigraphs (and also other quotes that I like).
\par
\epigraph{Morality is not properly the doctrine of how we may make ourselves happy, but how we may make ourselves worthy of happiness.}{\attrib{Immanuel Kant}{Critique of Practical Reason}{1788}}
\postepi
\epigraph{How does it happen that a properly endowed natural scientist comes to concern himself with epistemology? Is there no more valuable work in his specialty? I hear many of my colleagues saying, and I sense it from many more, that they feel this way. I cannot share this sentiment. When I think about the ablest students whom I have encountered in my teaching, that is, those who distinguish themselves by their independence of judgment and not merely their quick-wittedness, I can affirm that they had a vigorous interest in epistemology. They happily began discussions about the goals and methods of science, and they showed unequivocally, through their tenacity in defending their views, that the subject seemed important to them. Indeed, one should not be surprised by this.}{\attrib{Albert Einstein}{Physikalische Zeitschrift}{1916}}
\postepi
\epigraph{One must imagine Sisyphus happy.}{\attrib{Albert Camus}{The Myth of Sisyphus}{1942}}
\postepi
\epigraph{Tonight, we gather to affirm the greatness of our nation---not because of the height of our skyscrapers, or the power of our military, or the size of our economy. Our pride is based on a very simple premise, summed up in a declaration made over two hundred years ago: `We hold these truths to be self-evident, that all men are created equal, that they are endowed by their Creator with certain inalienable rights, that among these are life, liberty, and the pursuit of happiness.' That is the true genius of America---a faith in simple dreams, an insistence on small miracles.}{\attrib{Barack Obama}{DNC Keynote Speech}{2004}}
\postepi

\chapter{Top Ten Lists}
\newpage

This is gimmicky, but I felt like I should write these down somewhere.\marginthought{also, I'm a fuckin' hack.}
\section{Movies}\label{sec:moviestopten}
The top two here are rock solid, but the rest can mostly be shuffled around.

\begin{itemize}
    \item[1.] \textbf{\textit{Short Term 12}}
    \item[1.] \textbf{\textit{Speed Racer}} This movie encapsulates perfectly the idea of sincerity. Made with the usual non-subtlety of the Waichowskis, it is so true to itself that it is, in my opinion, impossible not to love.
    \item[3.]
    \item[4.]
    \item[5.] \textbf{\textit{Mad Max: Fury Road}}
    \item[6.] \textbf{\textit{No Country for Old Men}}
    \item[7.] \textbf{\textit{Detention}}
    \item[8.] \textbf{\textit{Good Night, And Good Luck}}
    \item[9.] \textbf{\textit{Columbus}}
    \item[10.] \textbf{\textit{Hunt for the Wilderpeople}}
\end{itemize}
\subsection{Honorable Mentions}

\section{TV Shows}
\begin{enumerate}\bfseries
    \item \textit{The West Wing}
\end{enumerate}

\chapter{Just Random Bullshit}
\newpage

\section{Random Thoughts on Speech}
Our (humanity's) ability to describe is, as far as is currently known, unique amongst animals.
Other species, most or all of them, have the ability to speak, some form of language exists.
However, they deal purely within the scope of the prescriptive.
They don't recall.
They don't tell they're friends what they did the preceding day.
We do.

Not only do we flap our gums (and twirl our pens) regarding things that have happened, we communicate ideas and concepts that have no basis in reality (hot take, e.g., religion). In \emph{Sapiens}, it is asserted that this ability is our most fundamental property, what renders \emph{Homo sapiens} alone alongside all other animals.
It has been found, independently (I think, haven't checked) that \emph{Homo sapiens'} superlative pattern recognition (superior pattern processing) is a foundational element of language.

An anecdote, from Barthes: Karl von Frisch, sought to prove the commonly held hypothesis that bees had some form of language.
He succeeded in this pursuit, but found that their language contained no description, it was all prescriptive; they had to coordinate in order to get their food, they didn't talk about what they were doing or had done, if such information was not imminently necessary for survival.
Basically, we're the only animals that talk about shit that doesn't really matter/exist.

\section{The Pumping Lemmas}
% \subsection*{The Pumping Lemma for Regular Languages (Formal Statement)}
% \newcommand{\tab}{\hspace*{2em}}
% \begin{align*}
% &(\forall L\subseteq \Sigma^{\ast}) \\
% &\tab (regular(L) \Rightarrow \\
% &\tab ((\exists p\geq 1)((\forall w\in L)((|w| \geq p) \Rightarrow \\
% &\tab ((\exists x,y,z \in \Sigma^{\ast})(w=xyz \wedge (|y| > 0 \wedge |xy|\leq p \wedge (\forall i \geq 0)(xy^{i}z\in L))))))))
% \end{align*}

% \subsection*{The Pumping Lemma for Context Free Languages (Formal Statement)}
% \begin{align*}
% &(\forall L\subseteq \Sigma^{\ast}) \\
% &\tab (context\ free(L) \Rightarrow \\
% &\tab ((\exists p\geq 1)((\forall w\in L)((|w| \geq p) \Rightarrow \\
% &\tab ((\exists u,v,x,y,z \in \Sigma^{\ast})(w=uvxyz \wedge (|vy| > 0 \wedge |vxy|\leq p \wedge (\forall i \geq 0)(uv^{1}xy^{i}z\in L))))))))
% \end{align*}

\chapter{Some Poems and Stuff}
\newpage

\poemtitle{The Road Not Taken}
\poemauthor{Robert Frost}
\setlength{\versewidth}{39ex}
\begin{verse}[\versewidth]
Two roads diverged in a yellow wood, \\
And sorry I could not travel both \\
And be one traveler, long I stood \\
And looked down one as far as I could \\
To where it bent in the undergrowth; \\!

Then took the other, just as fair, \\
And having perhaps the better claim, \\
Because it was grassy and wanted wear; \\
Though as for that the passing there \\
Had worn them really about the same, \\!

And both that morning equally lay \\
In leaves no step had trodden black. \\
Oh, I kept the first for another day! \\
Yet knowing how way leads on to way, \\
I doubted if I should ever come back. \\!

I shall be telling this with a sigh \\
Somewhere ages and ages hence: \\
Two road diverged in a wood, and I--- \\
I took the one less traveled by, \\
And that has made all the difference. \\!
\end{verse}

\newpage

\poemtitle{The New Colossus}
\poemauthor{Emma Lazarus}
\setlength{\versewidth}{45ex}
\begin{verse}[\versewidth]
Not like the brazen giant of Greek fame,         \\
With conquering limbs astride from land to land; \\
Here at our sea-washed, sunset gates shall stand \\
A mighty woman with a torch, whose flame         \\
Is the imprisoned lightning, and her name        \\
Mother of Exiles. From her beacon-hand           \\
Glows world-wide welcome; her mild eyes command \\
The air-bridged harbor that twin cities frame. \\
``Keep, ancient lands, your storied pomp!'' cries she \\
With silent lips. ``Give me your tired, your poor, \\
Your huddled masses yearning to breathe free, \\
The wretched refuse of your teeming shore. \\
Send these, the homeless, tempest-tost to me, \\
I lift my lamp beside the golden door!'' \\
\end{verse}

\newpage

\poemtitle{When I Consider How My Light Is Spent}
\poemauthor{John Milton}
\setlength{\versewidth}{45ex}
\begin{verse}[\versewidth]
When I consider how my light is spent \\
\vin Ere half my days in this dark world and wide, \\
\vin And that one talent which is death to hide \\
Lodged with me useless, though my soul more bent \\
To serve therewith my Maker, and present \\
\vin My true account, lest He returning chide; \\
\vin ``Doth God exact day-labor, light denied?'' \\
I fondly ask. But Patience, to prevent \\
That murmur, soon replies, ``God doth not need \\
\vin Either man’s work or His own gifts. Who best \\
\vin Bear His mild yoke, they serve Him best. His state \\
Is kingly: thousands at His bidding speed, \\
And post o’er land and ocean without rest; \\
They also serve who only stand and wait.'' \\
\end{verse}

\newpage

\poemtitle{Eulogy (\textit{Synecdoche, New York})}
\poemauthor{Charlie Kaufman}
\setlength{\versewidth}{46ex}
\begin{verse}
Everything is more complicated than you think. \\
You only see a tenth of what is true. \\
There are a million little strings attached to every choice you make; you can destroy your life every time you choose. \\
But maybe you won't know for twenty years. \\
And you'll never trace it to its source. \\
And you only get one chance to play it out. \\
Just try and figure out your own divorce. \\
And they say there is no fate, but there is: it's what you create. \\
Even though the world goes on for eons and eons, your are here for a fraction of a fraction of a second. \\
Most of your time is spent being dead or not yet born. \\
But while alive, you wait in vain, wasting years, for a phone call or a letter or a look from someone or something to make it all right. \\
And it never comes or it seems to but doesn't really. \\
And so you spend your time in vague regret or vaguer hope for something good to come along. \\
Something to make you feel connected, to make you feel whole, to make you feel loved. \\
And the truth is I'm so angry and the truth is I'm so fucking sad, and the truth is I've been so fucking hurt for so fucking long and for just as long have been pretending I'm ok, just to get along, just for, I don't know why, maybe because no one wants to hear about my misery, because they have their own, and their own is too overwhelming to allow them to listen to or care about mine. \\!

Well, fuck everybody. \\!

Amen.\\!
\end{verse}

\newpage

\Character[Professor Perlman]{PERLMAN}{perlman}
\Character[Elio Perlman]{ELIO}{elio}
\poemtitle{Professor Perlman's Monologue (\textit{Call Me By Your Name})}
\poemauthor{James Ivory}
\begin{drama} % \ttfamily
\perlmanspeaks So? Welcome home. Did Oliver enjoy the trip?
\eliospeaks I think he did.
\StageDir{\perlman takes a drag from his cigarette, then pauses a moment before speaking.}
\perlmanspeaks You two had a nice friendship.
\eliospeaks\direct{somewhat evasive} Yes.
\StageDir{Another pause, and another drag on his cigarette.}
\perlmanspeaks You're too smart not to know how rare, how special, what you two had was.
\eliospeaks Oliver was Oliver.
\perlmanspeaks \emph{``Parce-que c'etait lui, parce-que c'etait moi.''} \direct{Because he was he, because I was I}
\eliospeaks\direct{trying to avoid talking about Oliver with his father} Oliver may be very intelligent --
\perlmanspeaks\direct{interrupting his son} Intelligent? He was more than intelligent. What you two had had everything and nothing to do with intelligence. He was good, and you were both lucky to have found each other, because you too are good.
\eliospeaks I think he was better than me.
\perlmanspeaks I'm sure he'd say the same think about you, which flatters the two of you.
\StageDir{In tapping his cigarette and leaning toward the ashtray, he reaches out and touches Elio's hand. \perlman alters his tone of voice (his tone says: We don't have to speak about it, but let's not pretend we don't know what I'm saying).}
\perlmanspeaks\direct{cont'd} When you least expect it, Nature has cunning ways of finding our weakest spot. Just remember: I am here. Right now you may not want to feel anything. Perhaps you never wished to feel anything. And perhaps it's not to me that you'll want to speak about these things. But feel something obviously you did.
\StageDir{\elio looks at his father, then drops his eyes to the floor.}
\perlmanspeaks\direct{cont'd} Look---you had a beautiful friendship. Maybe more than a friendship. And I envy you. In my place, most parents would hope the whole thing goes away, to pray that their sons land on their feet. But I am not such a parent. In your place, if there is a pain, nurse it. And if there is a flame, don't snuff it out. Don't be brutal with it. We rip so much of ourselves to be cured of things faster, that we go bankrupt by the age of thirty and have less to offer each time we start with someone new. But to make yourself feel nothing so as to not feel anything---what a waste!
\StageDir{\elio is dumstruck as he tries to take all this in.}
\perlmanspeaks\direct{cont'd} Have I spoken out of turn?
\StageDir{\elio shakes his head.}
\perlmanspeaks\direct{cont'd} Then let me say one more thing. It will clear the air. I may have come close, but I never had what you two had. Something always held me back or stood in the way. How you live your life is your business. Remember, our hearts and our bodies are given to us only once. And before you know it, your heart is worn out, and, as for your body, there comes a point when no one looks at it, much less wants to come near it. Right now there's sorrow. Pain. DOn't kill it and with it the joy you've felt.
\StageDir{\perlman takes a breath.}
\perlmanspeaks\direct{cont'd} We may never speak of this again. But I hope you'll never hold it against me that we did. I will have been a terrible father if, one day, you'd want to speak to me and felt that the door was shut, or not sufficiently open.
\eliospeaks Does mother know?
\perlmanspeaks I don't think she does.
\StageDir{But the way he says this means ``Even if she did, I am sure her feelings would be no different than mine.''}
\end{drama}

\newpage

\Character[Dr. Wong]{DR. WONG}{wong}
\poemtitle{Monologue (\textit{Rick and Morty} S03E03--\textit{Pickle Rick})}
\poemauthor{Jessica Gao}
\begin{drama}
\wongspeaks Rick, the only connection between your unquestionable intelligence and the sickness destroying your family is that everyone in your family, you included, use intelligence to justify sickness. You seem to alternate between viewing your own mind as an unstoppable force and as an inescapable force, and I think it's because the only truly unapproachable concept for you is that that it's \textbf{your} mind, within \textbf{your} control. You chose to come here, you chose to talk, to belittle my vocation, just as you chose to become a pickle. You are the master of your universe and yet you are dripping with rat blood and feces, your enormous mind literally vegetating by your own hand. I have no doubt that you would be bored senseless by therapy, the same way I'm bored when I brush my teeth and wipe my ass, because the thing about repairing, maintaining, and cleaning is; it's not an adventure. There's no way to do it so wrong you might die. It's just work, the bottom line is some people are okay going to work, and some people, well, some people would rather die. Each of us gets to choose.
\end{drama}

\end{document}