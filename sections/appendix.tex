%!TEX root = ../butidigress.tex
\documentclass[../butidigress.tex]{subfiles}
\begin{document}

\chapter{Road Trip 2018}
\newpage
This is a log of my 2018 road trip from Boston to Santa Barbara.
Starting on May 16th, the trip as planned will take about two weeks and will span the country.

\section*{A Note to My Friends}\marginthought{Friend represents far too banal terminology}
I used to tell people, truthfully, that I couldn't remember the last time I had been truly happy.

Instead I created a simulation of happiness, a smiling facade over anguish.

All too often I ignored, or chose to be ignored by, those who would look upon my true face unflinching.

Not with strained eyes of pity, nor cold eyes of disdain, but with warmth and compassion and empathy.

Only once I learned to simply open the door to those knocking, who wished only the simple pleasure of my company, did I experience true happiness.

\hspace*{2em} It is you all, who lent me strength when I was weak, courage when I was afraid, a smile when all I wanted to do was cry.

\hspace*{2em} For so much that I have, my life, my joy, I thank all of you.

\setcounter{section}{-1}

\section{The Night Before}\margindate{11:30PM}
I'm sitting here before my computer, nervous about the upcoming trip.
Well, not the trip necessarily, I'm excited about that, it's more the fact that I feel like I have so much to do tomorrow.

\section{May 16: Boston to Burlington}
Lying in bed.\margindate{6:45 AM}
More accurately, lying on my mattress, on the floor of my stressfully non-empty room.
Looking around I know that all evidence of me has to be completely gone by, like, noon, and I don't have a clear exit strategy for a lot of it.
For example, the current object of my repose, even a twin mattress is a fairly substantial thing of which to dispose.

In a weird way I'm not all that worried.

\entryskip

\margindate{10:00 AM}Just dropped off my last package at Fedex, now all I have to do is clean up and pack up the car.
First, however, I'm gonna take a little break (clearly, I'm writing this), sip my tea, and just wake up a little more.
It seems quite unfathomable that I'm gonna drive like four goddamn hours later today, but that's the way it is.

\entryskip

\margindate{6:39 PM}
I did end up driving like four goddamn hours.

Said four-ish hours\margincomm{More like three and a half} ended abruptly, when I tried to stop at a red light.
I heard a popping, banging sound from my car\footnote{always fun to hear}, and then when I tried to start driving again, there were some dope grinding noises.
Luckily, and seriously, this whole situation is unbelievably fortuitous, there was a repair place within walking distance, and they picked up my car, diagnosed it, and are in the process of fixing it.\footnote{it was a broken CV-joint for the record.}

While waiting for my car to get picked up, I made a reservation at a hotel (where I am currently) two miles from the repair place.
Moseying on back to the shop, I picked up my laptop, backpack, toiletries, and a change of clothes.
Being a smart person, I called an Uber to my hotel, is what I would have said if I were a smart person, instead I walked the two miles, in flip-flops, with my bags, after a long day already.

I don't know how I thought my first day of driving would go.
But not once did I envision walking across an overpass.

\section{May 17: Burlington to Ithaca}
\margindate{7:50 AM}It's weird how accustomed to things one's body can be.
Like, having slept in a twin bed my entire life, even the double that I slept in last night feels needlessly, luxuriously gargantuan.

Not feeling a whole lot of motivation to leave right now, but I must, must get breakfast, must get car.

Estimated drive time for today: six-ish hours\ldots

One last thought, the experience of writing everyday, or doing this actual journaling thing has been interesting.
Normally writing for me is like, synthesizing a whole mess of life experience down into something more abstract, applicable to a variety of situations.
The relative normality of this is strange, but I also feel like writing consistently, not just when the mood strikes, is the key to becoming a better writer.\footnote{which I want to become}

\entryskip

\margindate{10:21 AM}Just finished breakfast and am sitting in the same coffee shop that I went to yesterday\footnote{\textit{Uncommon Grounds}}.
Now I have to wait roughly an hour for my car to finished getting fixed.

As I was walking back from breakfast\footnote{at \textit{Handy's Lunch}}, I had a thought argument\footnote{super Stefon voice: it's that thing when you have an argument with an imaginary person in your head to help prove a point to yourself} about this, my, generation.
To be fair, I am at kinda the head of this generation, the actual group is the people just younger than me who grew up fully immersed in the World Wide Web.
Popular conceptions of this `Internet generation' seem to show it as self-centered, with heads buried in phones, and totally oblivious.
I could not disagree with the characterization more; this is honestly, an absurdly empathetic group of people.

Take, as an easy example, the rising enforcement of `PC' language; how is this anything but a profound empathetic response to the plight of the disparate?
For a cis-het (\textit{inter alia}) person to actively champion non-binary, `non-traditional' language is remarkable.

Also, about the phone thing, it's not like we're staring at our goddamn reflections, we're connecting with the world.

\entryskip

\margindate{7:20 PM}A pretty arduous day of driving done.

\entryskip

\margindate{8:26 PM}Bit of a false alarm there, took a shower and shaved.
Anyways, the day got started late, the mechanic took longer than I anticipated and there wasn't a clear time table.
So I walked around Burlington, well, I say around but I essentially walked in circles and slowly got more and more stressed.
I left around 1:30PM, which sucked; I was tired and freaking out a bit so the drive wasn't that great either.

An additional annoying element to leaving late was the inability to stop anywhere to take pictures.

My hotel in Ithaca\footnote{where I am right now} is no where near anything, not that I have the desire to go outside and do stuff right now.
Although I probably will end up running to Wegmans.

\section{May 18: Ithaca to Detroit}
\margindate{8:06 AM}Not much to report, Kate McKinnon is hilarious.

\margindate{5:47 PM} I seem to have a knack for choosing non-centrally located hotels, cause this one is near nothing.\margincomm{also this one specifically is pretty clearly primarily used for extended stay}

The drive through Canada was pretty boring, a lot like a drive through the US.
Except for the road signs being in kilometers and the highway being named after a monarch.

Coming back into the US was kinda weird though.
Being me, I was awkward with the border, immigration, whatever dude, which I'm guessing perturbed him.
Kinda expounded a bit too much I think on the whole trip and was vague and mumbly about where I came from and where I was going and why I was in Canada in the first place.
So asks to look in the trunk and goes to open the back door, so I was slow on the draw there and then the dog comes over and, even though there was nothing bad in the car, you're still nervous.
Obviously I was let through, but the level of scrutiny was weird.

Grimes + Elon Musk is a topic I spent some time working on in my head during the drive.
I've expressed my displeasure in their particular relationship\footnote{obviously, she's allowed to do whatever the hell she wants and if she's happy that's great.}, most of my animus is centered around Elon.

My dislike of him is really centered around two problems, one of which is common to people who run large businesses and the other is not really of his making.
First, in management v. labor, I pretty much unequivocally come down on the side of the latter, so his unfair labor practices irk me.
As mentioned, this is common in industry, that doesn't make it okay, but it's not like he's special in that regard.

People who like Elon Musk, seem to \emph{really} like him.
I am inherently distrustful of anyone held up as a messianic, savior-like figure.
Not his fault, I know, but it still freaks me out.

\section{\ May 19: Detroit to Columbus}
\margindate{3:34 PM}Dateline Columbus: forgot entirely to write this morning which kinda sucks, but\footnote{haha, sucks butt} now I'm in the first city I was really dying to visit.\margincomm{if you don't know why I'm so excited, go ahead and watch the movie \movie{Columbus}, it's great}
Anyways, fine drive today\ldots

\entryskip

\margindate{9:15 PM}Columbus was exactly what I expected in the best way.
It really was just a mid-western sleepy town with some awesome architecture.

Actually, I liked it so much, or I do like it so much, that I'm extending my stay here by a day.
I am, however, moving to a new hotel, because this one sucks and is dirty.
Cheap tho.

\section{May 20: Columbus}
\margindate{10:29 AM}Woke up way late, but it doesn't really matter, cause I'm not going anywhere today.

\entryskip

\margindate{11:22 AM}Sitting in the courtyard of the Columbus courthouse, I feel like this is what politicians refer to when they say that liberals don't understand `real' America.
Like, this is not a typical midwest city, but I get the feeling that this place voted about 65\% for Trump.\margincomm{I looked up and Bartholomew County, i.e., this county, voted 63.7\% for Trump, so not that far off}
At the same time, I feel like I could live here, it's quiet, interesting, and oddly fascinating.
Maybe my day here will show me something new.

I sometimes can't tell if people are actually looking at me funny or if it's just a projection of my general anxiety.
But, I am out of place here, being a coastal, college educated, elite.

I've been thinking about that, the fact that I grew up on a coast, went to school on another coast, and am now moving back to the coast.
It makes me feel like this trip, while well meaning\footnote{for whatever that's worth} is a bit condescending.
(Or, you know, super condescending.)

\section{May 21: Columbus to Milwaukee}
\margindate{8:25 AM}Excited to go see some baseball today.

\entryskip

\margindate{3:19 PM}Note that the date is now in Central Time.

Just showed up in Milwaukee, unfortunately, the check in time is 4:00 PM at my hotel.
So I'm just sitting in my car for a little bit waiting for four.

I walked around Marquette University for a little bit, pretty alright looking campus.
A lot of nice looking universities/other institutions are, I imagine, a result of building in a place where the largest expenditure isn't land, like it is in Boston or San Francisco or somewhere coastal or somewhere more `desirable.'

\section{May 22: Milwaukee to Minneapolis}
\margindate{8:55 AM}Quite tired.

Anyways, now I'm gonna write some stuff down and it's gonna go really well.

That Film Crit Hulk column on \href{http://observer.com/2018/05/the-two-crucial-filmmaking-elements-causing-all-your-movie-feuds/}{`text versus texture'} has really got me thinking.
Like, the epistemological question of `why do I like this?' is pretty much constant whenever I find myself enjoying something.
For instance, I was watching \textit{iZombie} right before writing this, and now I'm just trying to parse out, what, specifically, I like about the show.

(I'm not gonna provide any explanation at this juncture.
But, know that I'm struggling.)

\entryskip

\margindate{3:59 PM}Just showered and shaved in my Minneapolis hotel room, which is like a pretty dope studio type thing.

Excited for more baseball tonight.
Looks like a pretty quick drive tomorrow, to Fargo.

Cheating a bit here\footnote{b/c formatting} and writing post-Minneapolis.

It was fun watching live baseball, Target field is pretty nice, I left early, but I was pretty okay with it.
Even if the Dodgers aren't exactly amazing right now, I still like to watch them.

\section{May 23: Minneapolis to Fargo}
\margindate{1:28 PM}Full disclosure, I'm not actually staying in Fargo, I'm actually not even staying in North Dakota.
But, since Fargo is so close to the border, I'm staying like fifteen minutes away, in Moorhead, Minnesota.\footnote{I'm gonna keep referring to my location as Fargo}

\section{\ May 24: Fargo to Teddy Roosevelt Nat'l Park}
\margindate{8:30 AM}Going to my first National Park.

\entryskip

\margindate{5:29 PM}Another time zone change, now I'm in Mountain Time.

It would be reductive and, frankly, insulting to call Theodore Roosevelt National Park `beautiful.'

\section{May 25: Teddy Roosevelt Nat'l Park to Yellowstone}
\margindate{8:23 AM}Second National Park today.
Excited for this one, not that I wasn't excited for Teddy, b/c I feel like this is one that lives up to expectations.

\entryskip

\margindate{3:45 PM}Cody, Wyoming.
Still trying to figure out if I should go see Yellowstone stuff today.
I can either do that stuff today, or tomorrow.\margincomm{this seems obvious}
I'm tired right now, but tomorrow I have a long drive.
Such a quandary.

I'm leaning towards tomorrow, because I'm not exactly itching to do a bunch of stuff in Salt Lake City, could get a late start and just show up late.

\margindate{7:52 PM}Yup, tomorrow it is.

So I've been watching \textit{Parks and Rec} and there are so many elements that I find somewhat problematic.
First, Aziz Ansari, for obvious reasons.
Additionally, in watching the \textit{Partridge} episode, there are so many damn \textit{Infinite Jest} references.

\section{May 26: Yellowstone to Salt Lake City}
\margindate{7:25 AM}

\entryskip

\margindate{8:06 AM}Waaaaaayyy too tired to write anything in that first thing.
Still tired, but excited to see some Yellowstone ish today.

This trip is really starting to wear me down.
That extra day in Columbus honestly may have saved my gd life.
Like, I want to go check out Yellowstone, but I am so beat.
It's one thing just being tired, but this is something different; driving is so annoyingly tiring.

You're sitting, for hours, and then you're exhausted, with nothing physical to show for it.

\entryskip

\margindate{6:05 PM}Ugh

\section{May 28: Salt Lake City to Grand-Staircase Escalante}

\margindate{9:33 AM}Gosh dang, \textit{The 100} is an awesome show.
What presented itself as a pretty standard young-adult show\footnote{as in like, the first episode was pretty vanilla} is actually a super intense drama.
For example, in the third (?) episode of the fourth season, I saw a child die in her parents arms, of radiation poisoning; she took her last breaths, her breathing slowed until it stopped, her parents went from comforting to breaking down.
That's not some young-adult bs, that's for real.

I do really hate the Jasper character.
He's kinda an embodiment of postmodern, existentialism not caring about the world, that resignation to the universe's lack of regard for human life and morality.
I recognize that the universe does not have a modicum of regard for us\footnote{it can't it's inanimate, but the point is we're insignificant} but that does not, in my opinion, make my life meaningless.
In fact, choosing to live my life by a moral code is a more noble choice, laughing in the face of the unknowable void is bravery.
Succumbing to hopelessness, resigning oneself to a life free of morals or restraint is cowardly.\margincomm{this whole part will probably end up in the \emph{The Void} section}

I have been a fan of the show's interplay between faith and science.
Clarke, generally, represents science, or pragmatism, she takes the practical solution, as hard as it may be.
Jaha on the other hand, represents faith, he believes in long shots, maintaining hope, leading people to salvation through his beliefs.
The show does a great job of showing that we need both.

Being pragmatic is important, but sometimes hope is the only way out.

\entryskip

\margindate{2:57 PM}

\entryskip

\margindate{3:43 PM}Thought a lot about the Id, Ego, and Superego on the drive.
Also thought a lot about explaining shit to people.
I feel like I may have a decent ability to put concepts in more understandable terms.
I also enjoy talking about things in terms of engineering/science/mathematical concepts.
Maybe I should write down my explanations\ldots

\section{May 28: Grand-Staircase Escalante to Great Basin Nat'l Park}
\margindate{4:26 PM}Couldn't really think of anything to write this morning besides more stuff about \textit{The 100}\footnote{which I wrote down in my journal, for later transcription}

Anyways, pretty quick drive today, only like three-ish hours.
I went on a hike in Great Basin National Park, it was great.
Took a bunch\footnote{for me} of pictures.

But now I'm in my \say{hotel.}
I use the quotes because it is like no hotel to which I've ever been.
Also I have no cell service, and the WiFi here is super spotty.

\entryskip

\margindate{5:33 PM}Normally I wouldn't make a skip between these two entries, but I am fickle and capricious.
So there also isn't a desk here, I'm writing this on the couch, with the laptop on my lap.

Additional heads up, not so sure about these timestamps, because they use Pacific time here, but my phone isn't updating because no cell service.
So they're adjusted to PDT.

\section{May 29: Great Basin Nat'l Park to Death Valley Nat'l Park}
\margindate{8:00 AM}Second to last, or last really, leg of this trip.
As kinda a terrible person once said: \say{Don't cry because it's over, smile because it happened.}

\section{May 30: Death Valley Nat'l Park to Santa Barbara}
\margindate{8:07 AM}Just getting ready to leave.
Excited for a short drive today.

This trip has been fun, but next time I'd like to maybe do fewer days, or stay at each location longer.
I like road trips, but this was a bit excessive, or like, too short in a weird way.

\newlength{\episkip}
\setlength{\episkip}{0.5cm}
\newcommand{\postepi}{\vspace{\episkip}\noindent\hfill\rule{0.5\textwidth}{1pt}\hfill\vspace{\episkip}}

\chapter{Quotations}
\newpage
% https://en.wikiquote.org/wiki/Truth
% https://en.wikiquote.org/wiki/Skepticism
% https://en.wikiquote.org/wiki/Knowledge
% https://en.wikiquote.org/wiki/Science
% https://en.wikiquote.org/wiki/History_of_science
% https://en.wikiquote.org/wiki/Atomic_theory
% https://en.wikiquote.org/wiki/Immanuel_Kant
% https://en.wikiquote.org/wiki/Malcolm_X
% https://en.wikiquote.org/wiki/Martin_Luther_King,_Jr.
Some quotations that I decided to not use as epigraphs (and also other quotes that I like).
Honestly, some of these may be upgraded to epigraph status; for example, I had a Camus quote from \textit{The Myth of Sisyphus}, which is now on page \pageref{sec:voidexpounding}
\par
\epigraph{Morality is not properly the doctrine of how we may make ourselves happy, but how we may make ourselves worthy of happiness.}{\attrib{Immanuel Kant}{Critique of Practical Reason}{1788}}
\postepi
\epigraph{How does it happen that a properly endowed natural scientist comes to concern himself with epistemology? Is there no more valuable work in his specialty? I hear many of my colleagues saying, and I sense it from many more, that they feel this way. I cannot share this sentiment. When I think about the ablest students whom I have encountered in my teaching, that is, those who distinguish themselves by their independence of judgment and not merely their quick-wittedness, I can affirm that they had a vigorous interest in epistemology. They happily began discussions about the goals and methods of science, and they showed unequivocally, through their tenacity in defending their views, that the subject seemed important to them. Indeed, one should not be surprised by this.}{\attrib{Albert Einstein}{Physikalische Zeitschrift}{1916}}
\postepi
\epigraph{Tonight, we gather to affirm the greatness of our nation---not because of the height of our skyscrapers, or the power of our military, or the size of our economy. Our pride is based on a very simple premise, summed up in a declaration made over two hundred years ago: `We hold these truths to be self-evident, that all men are created equal, that they are endowed by their Creator with certain inalienable rights, that among these are life, liberty, and the pursuit of happiness.' That is the true genius of America---a faith in simple dreams, an insistence on small miracles.}{\attrib{Barack Obama}{DNC Keynote Speech}{2004}}
\postepi

\chapter{Top Ten Lists}
\newpage

This is gimmicky, but I felt like I should write these down somewhere.\marginthought{also, I'm a fuckin' hack.}
\section{Movies}\label{sec:moviestopten}
\margintodo{figure out 3 and 4}The top two here are rock solid\footnote{note they're tied for first}, but the rest can mostly be shuffled around.

\begin{itemize}
    \item[1.] \textbf{\textit{Short Term 12}} --- A beautiful, intimate film\ldots
    \item[1.] \textbf{\textit{Speed Racer}} --- This movie encapsulates perfectly the idea of sincerity. Made with the usual brick-to-the-face subtlety of the Waichowskis, it is so true to itself that it is, in my opinion, impossible not to love.
    \item[3.]
    \item[4.]
    \item[5.] \textbf{\textit{Mad Max: Fury Road}} --- Technically brilliant. Manages to pack a powerful story within a striking and kinetic film. Also Charlize Theron is incredible and an inspiration.
    \item[6.] \textbf{\textit{No Country for Old Men}} --- Within the Coen Brothers' pantheon, this is at the top. A poignant exploration of what it means to grow old and the cruelty of man\ldots
    \item[7.] \textbf{\textit{Detention}} --- Just bizarre\ldots
    \item[8.] \textbf{\textit{Good Night, And Good Luck}} --- I'm a sucker for stories about journalists, and this one is great. Shot in black and white, it deals with a tumultuous period of US history.
    \item[9.] \textbf{\textit{Columbus}} --- Beautifully shot, this film is small like \movie{Short Term 12}, but like that movie, packs a powerful story. The movie with the most \say{frame-able} scenes.
    \item[10.] \textbf{\textit{Hunt for the Wilderpeople}} --- An unorthodox coming-of-age film, Taika Waititi is a comedic master, he crafts a wonderfully sincere and intimate story of connection and adventure.
\end{itemize}
\subsection{Honorable Mentions}

\section{TV Shows}
\begin{enumerate}\bfseries
    \item \textit{The West Wing}
    \item \textit{}
\end{enumerate}

\chapter{Just Random Bullshit}
\newpage

\section{Random Thoughts on Speech}
Our (humanity's) ability to describe is, as far as is currently known, unique amongst animals.
Other species, most or all of them, have the ability to speak, some form of language exists.
However, they deal purely within the scope of the prescriptive.
They don't recall.
They don't tell they're friends what they did the preceding day.
We do.

Not only do we flap our gums (and twirl our pens) regarding things that have happened, we communicate ideas and concepts that have no basis in reality (hot take, e.g., religion). In \emph{Sapiens}, it is asserted that this ability is our most fundamental property, what renders \emph{Homo sapiens} alone alongside all other animals.
It has been found, independently (I think, haven't checked) that \emph{Homo sapiens'} superlative pattern recognition (superior pattern processing) is a foundational element of language.

An anecdote, from Barthes: Karl von Frisch, sought to prove the commonly held hypothesis that bees had some form of language.
He succeeded in this pursuit, but found that their language contained no description, it was all prescriptive; they had to coordinate in order to get their food, they didn't talk about what they were doing or had done, if such information was not imminently necessary for survival.
Basically, we're the only animals that talk about shit that doesn't really matter/exist.

\section{The Pumping Lemmas}
% \subsection*{The Pumping Lemma for Regular Languages (Formal Statement)}
% \newcommand{\tab}{\hspace*{2em}}
% \begin{align*}
% &(\forall L\subseteq \Sigma^{\ast}) \\
% &\tab (regular(L) \Rightarrow \\
% &\tab ((\exists p\geq 1)((\forall w\in L)((|w| \geq p) \Rightarrow \\
% &\tab ((\exists x,y,z \in \Sigma^{\ast})(w=xyz \wedge (|y| > 0 \wedge |xy|\leq p \wedge (\forall i \geq 0)(xy^{i}z\in L))))))))
% \end{align*}

% \subsection*{The Pumping Lemma for Context Free Languages (Formal Statement)}
% \begin{align*}
% &(\forall L\subseteq \Sigma^{\ast}) \\
% &\tab (context\ free(L) \Rightarrow \\
% &\tab ((\exists p\geq 1)((\forall w\in L)((|w| \geq p) \Rightarrow \\
% &\tab ((\exists u,v,x,y,z \in \Sigma^{\ast})(w=uvxyz \wedge (|vy| > 0 \wedge |vxy|\leq p \wedge (\forall i \geq 0)(uv^{1}xy^{i}z\in L))))))))
% \end{align*}

\chapter{Some Poems and Stuff}
\newpage

\poemtitle{The Road Not Taken}
\poemauthor{Robert Frost}
\setlength{\versewidth}{39ex}
\begin{verse}[\versewidth]
Two roads diverged in a yellow wood, \\
And sorry I could not travel both \\
And be one traveler, long I stood \\
And looked down one as far as I could \\
To where it bent in the undergrowth; \\!

Then took the other, just as fair, \\
And having perhaps the better claim, \\
Because it was grassy and wanted wear; \\
Though as for that the passing there \\
Had worn them really about the same, \\!

And both that morning equally lay \\
In leaves no step had trodden black. \\
Oh, I kept the first for another day! \\
Yet knowing how way leads on to way, \\
I doubted if I should ever come back. \\!

I shall be telling this with a sigh \\
Somewhere ages and ages hence: \\
Two road diverged in a wood, and I--- \\
I took the one less traveled by, \\
And that has made all the difference. \\!
\end{verse}

\newpage

\poemtitle{The New Colossus}
\poemauthor{Emma Lazarus}
\setlength{\versewidth}{45ex}
\begin{verse}[\versewidth]
Not like the brazen giant of Greek fame,         \\
With conquering limbs astride from land to land; \\
Here at our sea-washed, sunset gates shall stand \\
A mighty woman with a torch, whose flame         \\
Is the imprisoned lightning, and her name        \\
Mother of Exiles. From her beacon-hand           \\
Glows world-wide welcome; her mild eyes command \\
The air-bridged harbor that twin cities frame. \\
``Keep, ancient lands, your storied pomp!'' cries she \\
With silent lips. ``Give me your tired, your poor, \\
Your huddled masses yearning to breathe free, \\
The wretched refuse of your teeming shore. \\
Send these, the homeless, tempest-tost to me, \\
I lift my lamp beside the golden door!'' \\
\end{verse}

\newpage

\poemtitle{When I Consider How My Light Is Spent}
\poemauthor{John Milton}
\setlength{\versewidth}{45ex}
\begin{verse}[\versewidth]
When I consider how my light is spent \\
\vin Ere half my days in this dark world and wide, \\
\vin And that one talent which is death to hide \\
Lodged with me useless, though my soul more bent \\
To serve therewith my Maker, and present \\
\vin My true account, lest He returning chide; \\
\vin ``Doth God exact day-labor, light denied?'' \\
I fondly ask. But Patience, to prevent \\
That murmur, soon replies, ``God doth not need \\
\vin Either man’s work or His own gifts. Who best \\
\vin Bear His mild yoke, they serve Him best. His state \\
Is kingly: thousands at His bidding speed, \\
And post o’er land and ocean without rest; \\
They also serve who only stand and wait.'' \\
\end{verse}

\newpage

\poemtitle{Eulogy (\textit{Synecdoche, New York})}
\poemauthor{Charlie Kaufman}
\setlength{\versewidth}{46ex}
\begin{verse}
Everything is more complicated than you think. \\
You only see a tenth of what is true. \\
There are a million little strings attached to every choice you make; you can destroy your life every time you choose. \\
But maybe you won't know for twenty years. \\
And you'll never trace it to its source. \\
And you only get one chance to play it out. \\
Just try and figure out your own divorce. \\
And they say there is no fate, but there is: it's what you create. \\
Even though the world goes on for eons and eons, your are here for a fraction of a fraction of a second. \\
Most of your time is spent being dead or not yet born. \\
But while alive, you wait in vain, wasting years, for a phone call or a letter or a look from someone or something to make it all right. \\
And it never comes or it seems to but doesn't really. \\
And so you spend your time in vague regret or vaguer hope for something good to come along. \\
Something to make you feel connected, to make you feel whole, to make you feel loved. \\
And the truth is I'm so angry and the truth is I'm so fucking sad, and the truth is I've been so fucking hurt for so fucking long and for just as long have been pretending I'm ok, just to get along, just for, I don't know why, maybe because no one wants to hear about my misery, because they have their own, and their own is too overwhelming to allow them to listen to or care about mine. \\!

Well, fuck everybody. \\!

Amen.\\!
\end{verse}

\newpage

\Character[Professor Perlman]{PERLMAN}{perlman}
\Character[Elio Perlman]{ELIO}{elio}
\poemtitle{Professor Perlman's Monologue (\textit{Call Me By Your Name})}
\poemauthor{James Ivory}
\begin{drama} % \ttfamily
\perlmanspeaks So? Welcome home. Did Oliver enjoy the trip?
\eliospeaks I think he did.
\StageDir{\perlman takes a drag from his cigarette, then pauses a moment before speaking.}
\perlmanspeaks You two had a nice friendship.
\eliospeaks\direct{somewhat evasive} Yes.
\StageDir{Another pause, and another drag on his cigarette.}
\perlmanspeaks You're too smart not to know how rare, how special, what you two had was.
\eliospeaks Oliver was Oliver.
\perlmanspeaks \emph{``Parce-que c'etait lui, parce-que c'etait moi.''} \direct{Because he was he, because I was I}
\eliospeaks\direct{trying to avoid talking about Oliver with his father} Oliver may be very intelligent --
\perlmanspeaks\direct{interrupting his son} Intelligent? He was more than intelligent. What you two had had everything and nothing to do with intelligence. He was good, and you were both lucky to have found each other, because you too are good.
\eliospeaks I think he was better than me.
\perlmanspeaks I'm sure he'd say the same think about you, which flatters the two of you.
\StageDir{In tapping his cigarette and leaning toward the ashtray, he reaches out and touches Elio's hand. \perlman alters his tone of voice (his tone says: We don't have to speak about it, but let's not pretend we don't know what I'm saying).}
\perlmanspeaks\direct{cont'd} When you least expect it, Nature has cunning ways of finding our weakest spot. Just remember: I am here. Right now you may not want to feel anything. Perhaps you never wished to feel anything. And perhaps it's not to me that you'll want to speak about these things. But feel something obviously you did.
\StageDir{\elio looks at his father, then drops his eyes to the floor.}
\perlmanspeaks\direct{cont'd} Look---you had a beautiful friendship. Maybe more than a friendship. And I envy you. In my place, most parents would hope the whole thing goes away, to pray that their sons land on their feet. But I am not such a parent. In your place, if there is a pain, nurse it. And if there is a flame, don't snuff it out. Don't be brutal with it. We rip so much of ourselves to be cured of things faster, that we go bankrupt by the age of thirty and have less to offer each time we start with someone new. But to make yourself feel nothing so as to not feel anything---what a waste!
\StageDir{\elio is dumstruck as he tries to take all this in.}
\perlmanspeaks\direct{cont'd} Have I spoken out of turn?
\StageDir{\elio shakes his head.}
\perlmanspeaks\direct{cont'd} Then let me say one more thing. It will clear the air. I may have come close, but I never had what you two had. Something always held me back or stood in the way. How you live your life is your business. Remember, our hearts and our bodies are given to us only once. And before you know it, your heart is worn out, and, as for your body, there comes a point when no one looks at it, much less wants to come near it. Right now there's sorrow. Pain. DOn't kill it and with it the joy you've felt.
\StageDir{\perlman takes a breath.}
\perlmanspeaks\direct{cont'd} We may never speak of this again. But I hope you'll never hold it against me that we did. I will have been a terrible father if, one day, you'd want to speak to me and felt that the door was shut, or not sufficiently open.
\eliospeaks Does mother know?
\perlmanspeaks I don't think she does.
\StageDir{But the way he says this means ``Even if she did, I am sure her feelings would be no different than mine.''}
\end{drama}

\newpage

\Character[Dr. Wong]{DR. WONG}{wong}
\poemtitle{Monologue (\textit{Rick and Morty} S03E03--\textit{Pickle Rick})}
\poemauthor{Jessica Gao}
\begin{drama}
\wongspeaks Rick, the only connection between your unquestionable intelligence and the sickness destroying your family is that everyone in your family, you included, use intelligence to justify sickness. You seem to alternate between viewing your own mind as an unstoppable force and as an inescapable force, and I think it's because the only truly unapproachable concept for you is that that it's \textbf{your} mind, within \textbf{your} control. You chose to come here, you chose to talk, to belittle my vocation, just as you chose to become a pickle. You are the master of your universe and yet you are dripping with rat blood and feces, your enormous mind literally vegetating by your own hand. I have no doubt that you would be bored senseless by therapy, the same way I'm bored when I brush my teeth and wipe my ass, because the thing about repairing, maintaining, and cleaning is; it's not an adventure. There's no way to do it so wrong you might die. It's just work, the bottom line is some people are okay going to work, and some people, well, some people would rather die. Each of us gets to choose.
\end{drama}

\end{document}