%!TEX root = ../philo.tex
\documentclass[../philo.tex]{subfiles}
\begin{document}
\chapter{Knowledge}
\epigraph{Human reason has this peculiar fate that in one species of its knowledge it is burdened by questions which, as prescribed by the very nature of reason itself, it is not able to ignore, but which, as transcending all its powers, it is also not able to answer.}{\attrib{Immanuel Kant}{Critique of Pure Reason}{1781}}
\newpage

I have, on several occasions, said something to the effect of that my greatest and continuing regret in my life is that I don't know everything.\footnote{I kinda love how unverifiable this is}
Quoting oneself is ridiculous and conceited; not to mention the fact that the sentiment expressed is also ridiculous, a facially absurd notion.

Despite the true (like, literally quantum mechanical) impossibility of complete knowledge, it fits with my common practice of setting goals that are purely aspirational.
For me, these goals are in a separate class from day-to-day, banal goals, like ``finish homework'' or ``find a job''; aspirational goals are those that I recognize are unattainable.
Goals such as these are, in my mind, couched in terms of calculus: as time goes to infinity, I achieve that goal.

What I'll be referring to as ``aspirational goals'' from here on out, are, in my experience, sort of antithetical to human nature.
As humans, we generally take the path that will lead to the most immediate satisfaction and have difficulty seeing particularly far into the future.
Aspirational goals offer no prospect of satisfaction, the catharsis comes in the journey.

\section{Knowledge Acquisition}
\epigraph{He who chooses to know for the sake of knowing will choose most readily that which is most truly knowledge.}{\attrib{Aristotle}{Metaphysics}{Unknown}}

Pursuant to my insatiable desire to learn\footnote{fuck, what a douche}, I try to read books of varying subjects.
For instance, as of this writing I am reading \textit{The Subject of Semiotics}, by Kaja Silverman, and before that I read \textit{Battle Cry of Freedom}, by James McPherson, a single-volume history of the Civil War.
These two books are illustrative of two aspects of my desire for knowledge: they are both far outside my field and they are disparate from each other.
Yet, each provides value to my life.

I have never wanted to be ``just'' an engineer, it feels too confining.
Reading on a wide variety of subject matter helps me become a more well-rounded person, a trait which I find admirable.
Well-roundedness has its virtues, e.g., (hypothetically) better conversation, different perspectives, a more interesting life, but it is not without potential drawbacks.

Most obvious as a drawback is the threat of coming off as ``a mile wide but an inch deep'', that property of having passing knowledge in much, but true understanding in little.
This ends up only being a problem when one is interacting with other people, it's not really an internal problem.
I don't want to discount the potential discursive drawback to this, coming across as shallow can be problematic.
On a practical level, other people are often not knowledgeable enough to recognize that one lacks a depth of knowledge in a subject.
While this is not, in my opinion, an adequate excuse, it at least mitigates the problem somewhat.
Besides getting a pass based on general lack of knowledge, being a mile wide, regardless of depth, can provide a discursive advantage\footnote{Advantage personally, not like, trying to get one over on someone} as conversations tend to be lubricated by either interlocutor having the ability to converse on a range of subjects.

\section{Inform Oneself}
\epigraph{Democracy Dies in Darkness}{\textsc{The Washington Post}}
My sister has has considerable influence regarding the development of my political/social opinions, especially since starting college.
From my point of view, some level of political consciousness is vital.
Knowledge of current events is also important to some tenets which will be laid out in the chapter on ``The Immutable'' (\pageref{immutable})

\end{document}