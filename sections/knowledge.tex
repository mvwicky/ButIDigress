%!TEX root = ../butidigress.tex
\documentclass[../butidigress.tex]{subfiles}
\begin{document}
\chapter{Knowledge}\label{chap:knowledge}
\epigraph{Human reason has this peculiar fate that in one species of its knowledge it is burdened by questions which, as prescribed by the very nature of reason itself, it is not able to ignore, but which, as transcending all its powers, it is also not able to answer.}{\attrib{Immanuel Kant}{Critique of Pure Reason}{1781}}
\newpage

I have, on several occasions, said something to the effect of that my greatest and continuing regret in my life is that I don't know everything.\footnote{a pretty hilariously unverifiable statement}
Quoting oneself is ridiculous and conceited; not to mention the fact that the sentiment expressed is also ridiculous, a facially absurd notion.

Despite the true (like, literally quantum mechanical) impossibility of complete knowledge, it fits with my common practice of setting goals that are purely aspirational.
For me, these goals are in a separate class from day-to-day, banal goals, like \say{finish homework} or \say{find a job;} aspirational goals are those that I recognize are unattainable.
Goals such as these are, in my mind, couched in terms of calculus: as time goes to infinity, I achieve that goal.

What I'll be referring to as \say{aspirational goals} from here on out, are, in my experience, sort of antithetical to human nature.\marginthought{I feel like my thoughts on this have changed a bit, or like, I rephrase a lot of this stuff}
As humans, we generally take the path that will lead to the most immediate satisfaction and have difficulty seeing particularly far into the future.
Aspirational goals offer no prospect of satisfaction, the catharsis comes in the journey.

\section{Asymptotic Truth}\label{sec:asymptotic}\margindate{5/13/18}\margincomm{For sure came up with this idea while super tired a couple days after graduation}
Above, goals were put in the form of a limit, here, that idea will be expounded on a bit, with the concept of \emph{asymptotic truth}\footnote{not married to this term, but patent pending just in case}.
A corollary to being empathetic is the idea that, never should one be completely confident in oneself.

This ties into the idea of objective truth, those facts which are true, regardless of perspective---the eigen-truths, independent of transformation.
Having read an extensive history of the concept of objectivity, I know that the term has gone through some changes throughout the years\parencite{objectivity}.
From its inception as attempts to distill variety into \say{types}, to the idea of mechanical objectivity, to the more recent trained judgment, we observe an ever evolving definition of \emph{truth}.
What I'm getting at is that there is a deep, potentially untraversable, divide between what we (as humans, or as individuals) \emph{know} and what is \emph{true}\footnote{this idea was elucidated pretty clearly in \movie{Men in Black}, when Tommy Lee Jones (Agent K) basically says that exact phrase to Will Smith (soon to be Agent J)}.\margincomm{insofar as complete truth is even a thing}

Instead of complete certainty, we have this asymptotic behavior, which is how I resolve my commitment to uncertainty with that which is viewed as \say{objectively true.}\margintodo{Clean up the language in this section, I feel like overall I have a point, but its muddled.}
To me\footnote{and basically to most scientists, it's just never expressed in this verbose a manner or as convoluted (basically scientists are more straightforward)}, this certainty is more like the tails of a standard normal distribution.
Think of the value of $\lim_{x\to\infty}\varphi(x)$, where $\varphi$ is the PDF of a Gaussian distribution with $\mu = 0$ and $\sigma = 1$, it's zero, that's our uncertainty.
But limits exist purely in the realm of mathematics; we, as human beings don't ever get to that point, \emph{there will always be some uncertainty}.

\section{Knowledge Acquisition}
\epigraph{He who chooses to know for the sake of knowing will choose most readily that which is most truly knowledge.}{\attrib{Aristotle}{Metaphysics}{Unknown}}

Pursuant to my insatiable desire to learn\marginthought{boy, does this whole paragraph and chapter and book just make me feel conceited}, I try to read books of varying subjects.
For instance, as of this writing I am reading \textit{The Subject of Semiotics}, by Kaja Silverman, and before that I read \textit{Battle Cry of Freedom}, by James McPherson, a single-volume history of the Civil War.
These two books are illustrative of two aspects of my desire for knowledge: they are both far outside my field and they are disparate from each other.
Yet, each provides value to my life.

I have never wanted to be \say{just} an engineer, it feels too confining.\margincomm{and it feels like I'm just looking out for number one}
Reading on a wide variety of subject matter helps me become a more well-rounded person, a trait which I find admirable.
Well-roundedness has its virtues, e.g., (hypothetically) better conversation, different perspectives, a more interesting life, but it is not without potential drawbacks.

Most obvious as a drawback is the threat of coming off as \say{a mile wide but an inch deep,} that property of having passing knowledge in much, but true understanding in little.
This ends up only being a problem when one is interacting with other people, it's not really an internal problem.
I don't want to discount the potential discursive drawback to this, coming across as shallow can be problematic.
On a practical level, other people are often not knowledgeable enough to recognize that one lacks a depth of knowledge in a subject.
While this is not, in my opinion, an adequate excuse, it at least mitigates the problem somewhat.
Besides getting a pass based on general lack of knowledge, being a mile wide, regardless of depth, can provide a discursive advantage\footnote{advantage personally, not like, trying to get one over on someone} as conversations tend to be lubricated by either interlocutor having the ability to converse on a range of subjects.

\section{Unconscious Signification}\marginthought{there really wasn't a super appropriate place to put this; this chapter is a bit of a catch-all (because knowledge is kinda everything)}
A key insight gleaned from \textit{The Subject of Semiotics}\footnote{a book I intend to read again and soon} is that unconscious or unintentional symbolism (i.e., signification) is essentially unavoidable in all texts.\footnote{heads up, this is the only semiotics book I've read, so, not exactly an expert}
This realization was and is key to my firm belief that art and artist are intertwined.
My firm commitment to art and artist is, in fact, a pretty big part of my philosophy, it is partially why I unhesitatingly and without reservation abandon art made by those unmasked to be horrible people.\margincomm{with notable and shameful exceptions}

\subsection{Semiotics}
Before we continue, one may be asking: what exactly is semiotics? I'll take a second to address this.
Semiotics is, succinctly defined, the study of \say{meaning making.}
Brevity may be the soul of wit, but not necessarily that of clarity; more elucidation is probably necessary.

I have come to think of semiotics as a sort of super-field of linguistics.
All of linguistics can, theoretically, be derived from semiotics, but it also encompasses so much more.
Film criticism, as an example, is rife with intentional and unintentional invocations of semiotic concepts.

Much of what I have read has much to say about the underlying structures of the human brain, the way that human beings process and regurgitate information.
Models of signification are used to show how one's mind constructs language; the most basic form being a dyadic model\footnote{now generally deprecated by more modern post-structuralist semioticians, but still useful in explanation} by which there is a signifier, i.e., that which stands for something else, and a signified, that object or concept for which the signifier stands.
Here, a quote from Umberto Eco, an Italian writer, philosopher, and semiotician seems appropriate (emphasis in original):
\begin{displayquote}
Semiotics is concerned with everything that can be \emph{taken} as a sign.
A sign is everything which can be taken as significantly substituting for something else.\autocite{ecosemiotics}
\end{displayquote}

A key concept is that of the \say{signification chain,} which is a phenomenon by which a signified is a signifier for some other \say{thing\footnote{thing in the general sense},} which then signifies something else, and so on and so forth.
From the dyadic model we get to more complex triadic models and then to concepts beyond the scope of this section particularly.\footnote{not that they won't be revisited later on}

\subsection{Back to the Subject at Hand\ldots}

\textit{The Subject of Semiotics} presents theories based on Freudian and post-Freudian (e.g., Jungian) models of the mind.
Concepts such as the primary and secondary processes are invoked frequently and the Id, Ego, and Superego are at least alluded to.
Casting aside trepidation regarding the whole Oedipal fixation found in Freud's work, it all seems pretty solid, particularly the interaction between the unconscious, preconscious, and conscious minds.

I have decided for the purposes of this piece to pretty much conflate entirely the Id, unconscious, and primary process as well as combine the Ego and the secondary process.
This blenderization is not strictly correct and those who have studied Freud extensively, or like, at all I'm assuming, or have an interest or explicit knowledge of psychology/psychoanalytics will likely take to the streets for this transgression.
To them I say: \say{Come at me\footnote{bros} with your pitchforks and torches, I don't care, I'll conflate shit all I want! And besides, what are y'all gonna do? Kill Me? I didn't think so.}
This level of extreme confidence stems generally from a proportional conviction in the dearth of human beings who care enough about this topic.\footnote{also, the people who do care about this that much are likely quite small}

\subsection{For Real This Time}
Symbolism in media, I mostly will discuss film, but generally this applies to all media, is, in a sense, the whole point.
Nothing gets across without signification, no message, theoretically no meaning whatsoever; all aspects of the work must speak for something, in other words, be a signifier.
Since media are\footnote{until recently, and still predominantly} created by humans and consumed by humans, this, \say{universal signification,} holds true; it exists regardless of intention, is my contention.

Directors, screenwriters, artists, wardrobe, production designers, etc. all donate blood, sweat, and tears to projects, everything that they make or think up or write that ends up in the film, is a signifier.
The sign of a great director is the ability to have all departments working coherently, delivering identical semiotic signals to the audience.
However, the director is only human, no amount of diligence can give one power/knowledge of all the goings on in a complex film set; this will be important later, no amount of introspection or self-control gives on power/knowledge over/of one's own mind.

\section{Inform Oneself}
\epigraph{Democracy Dies in Darkness}{\textsc{The Washington Post}}
My sister has has considerable influence regarding the development of my political/social opinions, especially since starting college.
From my point of view, some level of political consciousness is vital.
Knowledge of current events is also important regarding ideas which are laid out in the chapter on \say{The Immutable\footnote{\cref{chap:immutable}}}

\end{document}
