%!TEX root = ../butidigress.tex
\documentclass[../butidigress.tex]{subfiles}

\newcounter{worldwarcounter}\setcounter{worldwarcounter}{1}

\begin{document}
\chapter{Sincerity}\label{chap:sincerity}
\epigraph{I think if you're too concerned with being cool or hip or liked, you can't really make good TV because sincerity and coolness are opposites}{\attrib{Michael Schur}{Interview with The AV Club}{2012}}
\newpage

The quotation on the previous page elucidates the idea behind this point more eloquently than I ever could.
Despite that, I will endeavor to clarify how this may apply to my life specifically.

To be clear, this point is not a wide-ranging proscription on lying or deceit generally.\footnote{\textbf{TO BE ADDITIONALLY CLEAR}: I'm not goddamn encouraging lying and shit}
Full disclosure, this one does not really apply to my life at all, it really applies more to the media which I consume.
In an attempt to clarify my point, I will comment my two favorite films\footnote{they're tied for first (see page \pageref{sec:moviestopten})}: \movie{Speed Racer} and \movie{Short Term 12}.
Each of these movies portray, in my opinion, sincerity in film and the powerful effect said sincerity adds to film.

\vspace{1em}\textbox{Just gonna add a little note here that I'm sitting here, at my desk in 269 Highland, and I'm trying to parse through the miasma, the rat's nest that is the way I've dealt with point of view. I didn't start this under the assumption that I'd be the only one to ever read this. In a particularly lucid moment I came to the realization that this would be the case. I just realized I don't know why I started this note\ldots}

\section{History} \margindate{5/2/2018}
We'll now embark on an adventure, an adventure of time and space and humanity and art and growth\footnote{\textsc{I assume no liability for any disappointment or material damage incurred by the following \say{adventure} not containing all the aforementioned features}}.
Tangentially related to the above topic, the history of art may\footnote{let's be honest, there's a good change there will be no illumination here, but hopefully it'll be interesting\ldots} provide some useful illumination regarding sincerity in modern media.

\margindate{6/1/2018}Modernism was an artistic and philosophical movement that started around the turn of the \nth{20} century.
Kickstarted by rapid industrialization and further motivated by World War \Roman{worldwarcounter}, modernism was, and is, characterized by the rejection of the certainty of the previous eras, most notably the Age of Enlightenment.
Intellectuals of the time, influenced by their Enlightenment peers, took certain elements of the world as \say{truths,} just as we do now.
Among the most fundamental of the time\footnote{again, as it is today, but decreasing in relevance and acceptance} was religion, the notion of a higher power, a \say{God,} was pretty much taken as \textit{a priori} true, a metaphysical certainty.

\margintodo{finish up this history lesson, give some commentary on modernism and post-modernism, viz a viz the fact the modern is an actual word}Modernism rejected such certainties; modern philosophers rejected long held certainties, like governments and science and public institutions generally.
\margindate{6/2/2018}From the start modernism was, essentially, by and for elites\autocite[2]{cambridgemodern}, who sought plenary control over who would \say{understand} their artistry.
However, this injustice, while of true import, is really neither here nor there.

Modernism, being a whole philosophical movement, was about more than art, it was about a new way of looking at the world; the underpinnings of this way of thinking can be found in the works of three intellectuals\autocite[9]{cambridgemodern}.

Karl Marx showed the world the facade of the socio-economic system, exposing how the haves exploited the have-nots.
Sigmund Freud, in investigating the unconscious mind, revealed how we cannot even be certain of what is inside our own minds.
Friedrich Nietzsche portrayed pretty much the entire \say{Western} way of thinking, from Socrates on, as a series of illusions, lambasting Christianity specifically.

\todobrak{insert a history of modernism and post-modernism and how they relate to sincerity}

There exists a modern exemplar of the virtue of sincerity; he has been intimately involved with four acclaimed television shows, all notable for their embrace of sincere emotion.\margindate{6/4/18}

\section{Michael Schur}\margintodo{figure out a transistion}
\margindate{6/2/18}\textit{The Office}, \textit{Parks and Recreations}, \textit{Brooklyn Nine-Nine}, and \textit{The Good Place}.
Four television shows that have (at least) two things in common.
The first is Michael Schur, who, writing and producing all four, created or co-created the latter three.
By my estimation, he is one of the most important/influential figures in contemporary pop-culture.
\margindate{6/3/18}His career has shown that he, so far, is a virtuoso in sincerity; he does not shy away from emotion, there is no ironic winking, no acknowledgment of the absurdity of the television environment\footnote{empathy (see \cref{chap:empathy}) also plays an important role in his shows, particularly notable in \textit{The Office}; the two (empathy and sincerity) are pretty well related}.

Some hay was made out of the sources of empathy in \tvshow{The Office} in \ref{sec:theoffice}\ and, as I've noted\footnote{in the note right about this one} sincerity goes a long way in generating empathy.
From here, we'll explore some of Schur's work\margintodo{why are we exploring?}

\section{Parks and Recreation}\margindate{6/4/18}

\section{\textit{Speed Racer}}
\epigraph{Speed, I understand that every child has to leave home. But I want you to know, that door is always open. You can always come back. 'Cause I love you.}{\attrib{From \movie{Speed Racer}}{The Wachowskis}{2008}}

Most people don't like this movie.\footnote{see., e.g., the 39\% on Rotten Tomatoes or the 37 on Metacritic}
Whenever I tell people that I love this movie, that it's my favorite film, I always feel the need to caveat my love or justify it in some way.
Something like stipulating a state of mind which must be entered in order to facilitate enjoyment; I do this for their benefit, my love is without restrictions or qualifications.
It's difficult to fully convince people I'm serious, that my favorite movie\footnote{again, 1(a) (or 1(b) (it doesn't actually matter, the point is that they aren't ordered))} is an American adaptation of a Japanese kids show that nobody likes (the movie that is, people can take it or leave it with the show I think).
But I am serious, and now that I've convinced you\footnote{me or Carly probs.} that I am indeed serious we can actually fucking talk about sincerity or whatever the fuck.\footnote{like for real, get to the darn point already}

\margintodo{this is incorrect and not why Speed Racer is dope}\movie{Speed Racer} is convinced, to an almost disconcerting degree, that race car driving is (almost) the most important activity in which one may participate; that being the best is the second most important thing in the world.
We are meant to truly believe in this primacy, besides family (this is a hella important point), racing trumps all.

\section{\textit{Short Term 12}}
\epigraph{Everything good in my life is because of you}{\attrib{From \movie{Short Term 12}}{Destin Daniel Cretton}{2013}}

As with the previous film we discussed, this is a difficult movie to introduce to people, like, its hard to tell people \say{oh, my favorite movie, it's \movie{Short Term 12}.}
The typical rejoinder to such a revelation is: \say{What the fuck is that?} and I have to explain that it's a beautiful and wonderfully small indie film that very few human beings have had the pleasure of seeing.
\end{document}
