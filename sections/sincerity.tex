%!TEX root = ../butidigress.tex
\documentclass[../butidigress.tex]{subfiles}
\begin{document}
\chapter{Sincerity}
\epigraph{What passes for hip cynical transcendence of sentiment is really some kind of fear of being really human, since to be really human \dots\ is probably to be unavoidably sentimental and na\"{i}ve and goo-prone and generally pathetic}{\attrib{David Foster Wallace}{Infinite Jest}{1996}}
\newpage

The quotation on the previous page elucidates the idea behind this point more eloquently than I ever could.
Despite that, I will endeavor to clarify how this may apply to my life specifically.

To be clear, this point is not a wide-ranging proscription on lying or deceit generally.\footnote{TO BE ADDITIONALLY CLEAR: I'm not goddamn encouraging lying and shit}
Full disclosure, this one does not really apply to my life at all, it really applies more to the media which I consume.
In an attempt to clarify my point, I will comment my two favorite films\footnote{They're tied for first}: \movie{Speed Racer} and \movie{Short Term 12}.
Each of these movies portray, in my opinion, sincerity in film and the powerful effect said sincerity adds to film.

\vspace{1em}\textbox{Just gonna add a little note here that I'm sitting here, at my desk in 269 Highland, and I'm trying to parse through the miasma, the rat's nest that is the way I've dealt with point of view. I didn't start this under the assumption that I'd be the only one to ever read this. In a particularly lucid moment I came to the realization that this would be the case. I just realized I don't know why I started this note\ldots}

\section{History} \marginnote{5/2/2018}
We'll now embark on an adventure.
Tangentially related to the above topic, the history of art may\footnote{or may not} provide some useful illumination regarding sincerity in modern media.

We


\section{\textit{Speed Racer}}
\epigraph{Speed, I understand that every child has to leave home. But I want you to know, that door is always open. You can always come back. 'Cause I love you.}{\attrib{From \movie{Speed Racer}}{The Wachowskis}{2008}}

Most people don't like this movie.\footnote{see., e.g., the 39\% on Rotten Tomatoes or the 37 on Metacritic}
Whenever I tell people that I love this movie, that it's my favorite film, I always feel the need to caveat my love or justify it in some way.
Something like stipulating a state of mind which must be entered in order to facilitate enjoyment; I do this for their benefit, my love is without restrictions or qualifications.
It's difficult to fully convince people I'm serious, that my favorite movie\footnote{Again, 1(a) (or 1(b))} is an American adaptation of a Japanese kids show that nobody likes (the movie that is, people can take it or leave it with the show I think).
But I am serious, and now that I've convinced you\footnote{me} that I am indeed serious we can actually fucking talk about sincerity or whatever the fuck.\footnote{Like for real, get to the darn point already}

\movie{Speed Racer} is convinced, to an almost disconcerting degree, that race car driving is (almost) the most important activity in which one may participate; that being the best is the second most important thing in the world.
We are meant to truly believe in this primacy, besides family (this is a hella important point), racing trumps all.

\section{\textit{Short Term 12}}
\epigraph{Everything good in my life is because of you}{\attrib{From \movie{Short Term 12}}{Destin Daniel Cretton}{2013}}

As with the previous film we discussed, this is a difficult movie to introduce to people, like, its hard to tell people ``oh, my favorite movie, it's \movie{Short Term 12}.''
A typical rejoinder to such a revelation is: ``What the fuck is that?'' and I have to explain that it's a beautiful and wonderfully small indie film that very few human beings have had the pleasure of seeing.
\end{document}