\documentclass[../butidigress.tex]{subfiles}

\newcounter{worldwarcounter}\setcounter{worldwarcounter}{1}

\begin{document}
\chapter{Sincerity}\label{chap:sincerity}
\epi{I think if you're too concerned with being cool or hip or liked, you can't really make good TV because sincerity and coolness are opposites}{\attrib{Michael Schur}{Interview with The AV Club}{2012}}
\newpage

The quotation on the previous page elucidates the idea behind this point more eloquently than I ever could.
Despite that, I will endeavor to clarify how this may apply to my life specifically.

To be clear, this point is not a wide-ranging proscription on lying or deceit generally.\ftnote{\textbf{TO BE ADDITIONALLY CLEAR}: I'm not goddamn encouraging lying and shit.}
Full disclosure, this one does not really apply to my life at all, it really applies more to the media which I consume.
In an attempt to clarify my point, I will comment my two favorite films\ftnote{They're tied for first (see page~\pageref{sec:moviestopten}).}: \movie{Speed Racer} and \movie{Short Term 12}.
Each of these movies portray, in my opinion, sincerity in film and the powerful effect said sincerity adds to film.

\vspace{1em}\textbox{Just gonna add a little note here that I'm sitting here, at my desk in 269 Highland, and I'm trying to parse through the miasma, the rat's nest that is the way I've dealt with point of view. I didn't start this under the assumption that I'd be the only one to ever read this. In a particularly lucid moment I came to the realization that this would be the case. I just realized I don't know why I started this note\ldots}

\digsec{A History Lesson}{5/2/2018}{history}
We'll now embark on an adventure, an adventure of time and space and humanity and art and growth\ftnote{\textsc{I assume no liability for any disappointment or material damage incurred in the event that the following \say{adventure} does not contain all the aforementioned features}.}.
Tangentially related to the above topic, the history of art may\ftnote{Let's be honest, there's a good change there will be no illumination here, but hopefully it'll be interesting\ldots} provide some useful illumination regarding sincerity in modern media.

\digsubsec{Modernism}{6/1/2018}{modernism}
Modernism was, and is, an artistic and philosophical movement that started around the turn of the \nth{20} century.
Kickstarted by rapid industrialization and further motivated by World War \Roman{worldwarcounter}, modernism was, and is, characterized by the rejection of the certainty of the previous eras, most notably the Age of Enlightenment.
Established intellectuals of the time, influenced by their Enlightenment predecessors, took certain elements of the world as \say{truths,} just as we do now.
Among the most fundamental of the time\ftnote{Again, as it is today, but decreasing in relevance and acceptance.} was religion, the notion of a higher power, a \say{God,} was pretty much taken as \textit{a priori} true, a metaphysical certainty.

\margintodo{finish up this history lesson, give some commentary on modernism and post-modernism, viz a viz the fact the modern is an actual word}Modernism rejected such certainties; modern philosophers rejected long held certainties, like governments and science and public institutions generally.
\margindate{6/2/2018}
From the start modernism was, essentially, by and for elites\autocite[2]{cambridgemodern}, who sought plenary control over who would \say{understand} their artistry.
However, this injustice, while of true import, is really neither here nor there.

Modernism, being a whole philosophical movement, was about more than art, it was about a new way of looking at the world; the underpinnings of this way of thinking can be found in the works of three intellectuals\autocite[9]{cambridgemodern}.

Karl Marx showed the world the facade of the socio-economic system, exposing how the haves exploited the have-nots.
Sigmund Freud, in investigating the unconscious mind, revealed how we cannot even be certain of what is inside our own minds.
Friedrich Nietzsche portrayed pretty much the entire \say{Western} way of thinking, from Socrates on, as a series of illusions, lambasting Christianity specifically.
For modernists, these epochal deconstructions of ingrained societal structures represented an opportunity to redefine the foundations of society\ftnote{With the caveat that only those who understood, i.e., those who were chosen (by modernists), had access and/or the privilege of the benefits of this new society.}.

\margindate{6/7/18}
Modernists did not reject certainty as a concept; with their rejection of, previously, integral\ftnote{Or derivative?} aspects of culture, they viewed themselves as able to create, essentially from whole cloth, equally integral replacements.
We can follow this thread pretty much directly to the concept of sincerity in contemporary media, i.e., the subject of the current chapter.
Champions of the modern art movement saw literature\ftnote{Which we'll interpret in the broadest possible sense.} as the societal replacement for religion; a new rock, upon which a new church would be built.

\begin{somenotes}{Modernism}
    \item long standing institutions were shown to be flawed/false
    \item hella epistemic questions
    \item experimentation with form
    \item kernel of future sincerity basically comes from the fact that modern (movement) writers saw literature as the new religion, i.e., the most important thing, so one had to be like, serious about it
\end{somenotes}

\setcounter{worldwarcounter}{2}
\digsubsec{PostModernism}{6/7/18}{postmodernism}
\begin{somenotes}{PostModernism}
    \item don't think of it so much as a reaction against (i.e., in contradiction to) so much as a synthesis of modernist ideas with contemporary attitudes
    \item post-modernism (as a term/idea) originated in Latin America in the 1930's; used by Federico de Onís to describe a \say{conservative reflux within modernism itself: one which sought refuge from its formidable lyrical challenge in a muted perfectionism of detail and ironic humor}\autocite[4]{originspostmodernity}
    \item Arnold Toynbee: writing \book{Study of History}, originally argued that Industrialism and Nationalism had shaped the history of the West, post-World War \Roman{worldwarcounter} he was pretty cynical regarding those two things---basically he hated civilization, Western civilization at least
    \item Lyotard: \say{I define \emph{postmodern} as incredulity toward meta-narratives}\autocite{postmodernsep}
\end{somenotes}

Cynicism and general disillusionment are integral to postmodernism as a movement.

\todobrak{a history of modernism and postmodernism and how they relate to sincerity}

\digsubsec{The Word ``Modern''}{6/7/18}{thewordmodern}\margintodo{discussion of how weird (annoying) it is that there is an art movement called modern, but modern is like, a word in common usage}

\digsec{\textit{Speed Racer}}{5/9/18}{speedracer}\margintodo{put this and the \movie{Short Term 12} section before the Mike Schur section, figure out a transition though}
\epi{Speed, I understand that every child has to leave home. But I want you to know, that door is always open. You can always come back. `Cause I love you.}{\attrib{From \movie{Speed Racer}}{The Wachowskis}{2008}}

Most people don't like this movie.\ftnote{See., e.g., the 39\% on Rotten Tomatoes or the 37 on Metacritic.}
Whenever I tell people that I love this movie, that it's my favorite film, I always feel the need to caveat my love or justify it in some way.
Something like stipulating a state of mind which must be entered in order to facilitate enjoyment; I do this for their benefit, my love is without restrictions or qualifications.
It's difficult to fully convince people I'm serious, that my favorite movie\ftnote{Again, 1(a) (or 1(b) (it doesn't actually matter, the point is that they aren't ordered)).} is an American adaptation of a Japanese kids show that nobody likes (the movie that is, people can take it or leave it with the show I think).
But I am serious, and now that I've convinced you\ftnote{Me or Carly probs.} that I am indeed serious we can actually fucking talk about sincerity or whatever the fuck.\ftnote{Like for real, get to the darn point already.}

\margintodo{this is incorrect and not why Speed Racer is dope}\movie{Speed Racer} is convinced, to an almost disconcerting degree, that race car driving is (almost) the most important activity in which one may participate; that being the best is the second most important thing in the world.
We are meant to truly believe in this primacy, besides family (this is a hella important point), racing trumps all.

\marginthought{actual reason why \movie{Speed Racer} is dope}
\todobrak{insert comments salient to sincerity}

(As an aside to the aforementioned sincerity, but not entirely irrelevant to it, \movie{Speed Racer} has incredible visual style and extremely well done and kinetic action.
Obviously attempting\ftnote{And succeeding.} to emulate the cartoon-ish look of the original \tvshow{Speed Racer}, the movie looks unlike anything I had seen until then and unlike anything I've seen since.
The action/editing delivers something consistent and coherent and thematically motivated.
Take, for example, the opening scene, it's a montage cross-cutting between timelines with crazy emotional arcs all happening in the middle of a race where, I might add, all the characters are being introduced, yet it is tremendously comprehensible.
And the one time, in the entire movie, that the action is confusing, it's with excellent purpose, because it's Speed transcending all the other racers and, in effect, the audience as well.)

\begin{somenotes}{\movie{Speed Racer}}
    \item \href{https://birthmoviesdeath.com/2013/11/20/hulks-favorite-movies-speed-racer-2008}{Hulk---2013}\autocite{hulkspeedracer2013}
    \begin{itemize}
        \item the movie is \emph{assured}
        \item the sincerity of family and feelings
        \item \say{What brings on Hulk's favoritism? Often it is spark, playful ingenuity, a belief in self. \ldots [T]hey make movies \emph{feel} new, even when the notion is an impossibility.}\autocite{hulkspeedracer2013}
        \item \say{\movie{Speed Racer} is really serious about its world. It's serious about the stakes. It's serious about the fun. It's serious about the characters. That's because it just \emph{believes in itself}.}
        \item \say{\ldots [T]he creativity of \movie{Speed Racer} is expressed in a much weirder way that easily throws us on just about every tonal level.}
        \item \say{But this isn't a lame case of simply trying to please everybody. This is a film where every single decision makes sens for the story and characters. Better yet, every decision hammers home the point of the movie itself.}
        \item \say{So when Hulk says you just have to \say{go with it} to enjoy \movie{Speed Racer}, Hulk isn't saying to turn your brain off. Hulk is actually asking you to turn your brain on. For this is as vibrant, swelling, and unapologetic a movie as they come.}
    \end{itemize}
    \item \href{https://birthmoviesdeath.com/2015/03/27/film-crit-hulk-smash-speed-racer-as-artist}{Hulk---2015}\autocite{hulkspeedracer2015}
    \begin{itemize}
        \item \say{The Wachowskis, perhaps more than any other filmmakers, suffer from the pains of being pure at heart}\autocite{hulkspeedracer2015}
        \item \say{They [the Wachowskis] make cinema that is so genuine and jaw-droppingly sincere that is can't help but skew right into most people's \say{this is corny} territory.}
        \item \say{\ldots [I]f Hulk had one creative standard of all filmmakers, it is the request that one's cinema be honest to what it really wants, what it really believes, and what it is really saying.}
    \end{itemize}
    \item \href{http://observer.com/2018/05/looking-back-at-the-wachowskis-2008-masterpiece-speed-racer/}{Hulk---2018}\autocite{hulkspeedracer2018}
\end{somenotes}

\digsec{\textit{Short Term 12}}{5/9/18}{shortterm12}
\epi{Everything good in my life is because of you}{\attrib{From \movie{Short Term 12}}{Destin Daniel Cretton}{2013}}

\begin{somenotes}{\textit{Short Term 12}}
    \item \href{https://birthmoviesdeath.com/2013/03/16/sxsw-review-why-short-term-12-is-a-masterpiece}{Hulk---2013}\autocite{hulkshortterm2015}
\end{somenotes}

As with the previous film we discussed, this is a difficult movie to introduce to people, like, its hard to tell people \say{oh, my favorite movie, it's \movie{Short Term 12}.}
The typical response to such a revelation is: \say{What the fuck is that?} and I have to explain that it's a beautiful and wonderfully small indie film that very few human beings have had the pleasure of seeing.

\margindate{6/7/19}
It is the story of a social worker; she works at a home for \say{troubled,} i.e., abused, children.
They're only supposed to stay there for a year (hence the title), but some have been there for a whole lot longer.

\entryskip

\margindate{6/4/18}
Having finished with two case studies of specific movies, we'll now turn to one man's catalog of work.
There exists a contemporary exemplar of the virtue of sincerity; he has been intimately involved with four acclaimed television shows, all notable for their embrace of sincere emotion.

\digsec{Michael Schur}{6/2/18}{mikeschur}
\epi{For storytelling purposes, there has to be conflict, but that doesn't mean the people have to be mean. I've never like mean-spirited comedy.}{\scshape Michael Schur}
\margintodo{figure out a transistion}
\tvshow{The Office}, \tvshow{Parks and Recreation}, \tvshow{Brooklyn Nine-Nine}, and \tvshow{The Good Place}.
Four television shows that have (at least) two things in common.
The first is Michael Schur, who, writing and producing all four, created or co-created the latter three.
By my estimation, he is one of the most important/influential figures in contemporary pop-culture.
\margindate{6/3/18}
His career has shown that he, so far, is a virtuoso in sincerity; he does not shy away from emotion, there is no ironic winking, no acknowledgment of the absurdity of the television environment\ftnote{Empathy (see~\cref{chap:empathy}) also plays an important role in his shows, particularly notable in \tvshow{The Office}; the two (empathy and sincerity) are pretty well related.}.

Some hay was made out of the sources of empathy in \tvshow{The Office} in~\ref{sec:theoffice}\ and, as I've noted\ftnote{In the note right above this one.} sincerity goes a long way in generating empathy.
From here, we'll explore some of Schur's work in order to ascertain what makes for sincerity in media; we'll also look at why it's important, from my opinion.

\digsubsec{Parks and Recreation}{6/4/18}{parksandrec}\margintodo{write these two, and a \tvshow{The Good Place} one}

\digsubsec{Brooklyn Nine-Nine}{6/5/18}{brooklynninenine}

\begin{somenotes}{Brooklyn 99}
    \item just watching the active shooter situation episode, and it's really good for the sincerity thing
    \item like, Holt's advice to Jake on how to help everybody in the precinct is to actually engage them and not distract them
    \item it totally show the father figure evolution of the character
    \item I like how they took Boyle from thirsty work friend, to actual best friend; also the way they changed Terry and Captain Holt to more real father figures
\end{somenotes}

\end{document}
