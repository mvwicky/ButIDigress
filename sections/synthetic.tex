\documentclass[../butidigress.tex]{subfiles}
\begin{document}

\chapter{Synthetic Topics; or, Musings}\label{chap:synthetic}
\newpage

\begin{description}
    \item [synthesis] \textit{noun}
    \begin{enumerate}
        \item the combining of the constituent elements of separate material or abstract entities into a single or unified entity (opposed to analysis), the separating of any material or abstract entity into its constituent elements.
        \item a complex whole formed by combining.
        \item \textbf{Chemistry}. the forming or building of a more complex substance or compound from elements or simpler compounds.
        \item \textbf{Philosophy}. the third stage of argument in Hegelian dialectic, which reconciles the mutually contradictory first to propositions, \emph{thesis} and \emph{antithesis}.
    \end{enumerate}
    \item [synthetic] \textit{adj.}, of, pertaining to, proceeding by, or involving synthesis (opposed to analysis).
\end{description}

As with any endeavor, there were specific moments in this writing that were more difficult than others.
But, as is so often the case, from this pain comes inspiration.
In my writing, I was quite often constrained by the format I had chosen; splitting this work into monolithic-seeming chapters limited my ability to present thoughts which stemmed from multiple concepts.
Such concepts tend to be the most important.

\end{document}