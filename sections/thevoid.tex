%!TEX root = ../butidigress.tex
\documentclass[../butidigress.tex]{subfiles}
\begin{document}
\chapter{The Void}\label{chap:thevoid}
\epigraph{Can one be a saint without God?, that's the problem, in fact the only problem I'm up against today.}{\attrib{Albert Camus}{The Plague}{1947}}
\newpage
This comes pretty clearly from my (partial) embrace of the absurd.
I struggle with Camus' absurd, caveat that I have read \textit{The Myth of Sisyphus} but once; my difficulty stems from the acknowledgment of some form of higher power and the extremely gendered terminology in \textit{Sisyphus}.
Intellectually I cannot really brook any consideration of \say{God} in any real form.
In normal conversation I tend to describe myself as agnostic, partially for the sake of expediency, partially out of my fear of certainty\ftnote{partially out of a fear of nothingness}, but in reality I am an atheist.
I'm not even really anti-religion, people are free to believe what they believe, and if it doesn't infringe on me, then more power to them.
Religion, for all its faults, has made some great things in the world, and getting a huge swath of humanity to conform to some morality is a good, even if most people don't follow exactly.\ftnote{this partial embrace of religion feels almost like a betrayal to my liberal, academic sensibility}

I originally titled this section \textit{Fight the Void}, which is not necessarily inappropriate; the current title reflects a broadening of scope.\margincomm{and I just like it better, it's cleaner}
While the fight against oblivion is important, some might say the most important, conceptually, it is necessary to understand what one is up against.
So this chapter title is meant to be about the fight, but also about the enemy.

\entryskip

Here, I'll give a brief introduction to what the fuck absurdism even is\ldots

As noted above, absurdism is most frequently associated with Albert Camus, the \nth{20} century French-Algerian author and philosopher.
The seminal work in this school of though is \textit{The Myth of Sisyphus} an essay, written by Camus in 1942, which outlines the virtues of the \say{absurd man.}
Again, this work had some seriously gendered terminology, a kind of implicit belief that only the male mind is capable of such rich philosophical sight.
This dated way of thinking is easily reconcilable, just imagine \say{people.}

What it really boils down to\ftnote{or at least, this is the succinct way of thinking about it I just thought of} is that life, existence as a whole, is about the journey, because that's all there is.\autocite{mythofsisyphus}

\section{Expounding on the Infinite}\label{sec:voidexpounding}
\epigraph{One must imagine Sisyphus happy.}{\attrib{Albert Camus}{The Myth of Sisyphus}{1942}}
As previously asserted, I don't subscribe fully to Absurdism.\margintodo{this is just rambling at this point}
Conceptually it is appealing to me.

Maybe the higher power thing is not as integral as I thought.
It's more about the nobility of the fight, of not giving in.
I'll talk more about some specifics in the section on \textit{The 100}, (\ref{sec:thehundred}) but I'll cover some generalities right now.

\margintodo{some generalities}

\section{Case Study: \textit{The 100}}\label{sec:thehundred}\margintodo{basically copy (with revisions) the stuff I wrote in the appendix}
\textit{The 100} is a television show, currently\ftnote{as of this writing} airing on \textit{The CW}.
Set in a dystopian future in which civilization, destroyed by global nuclear war, is kept alive by around 500 people aboard a space station, the Ark, orbiting Earth; the series starts around 100 years after the war.
In order to maintain discipline, all crimes are treated as capital offenses, except for those under eighteen, who are imprisoned.
There are, at the beginning of the series, exactly 100\ftnote{somehow} of said young offenders.
These reprobates\ftnote{some are actually bad people, others are locked up for more innocuous crimes or for getting caught being altruistic, like stealing medicine for sick loved ones and the like} are sent to the surface to test survivability of the radiation levels.

As one might expect, sending a group of raucous youths, unsupervised, to a, theoretically, uninhabited planet, leads to some hijinks, and a certain amount of attempted murder.
People to whom I've talked about the show tend to dismiss it as \say{just another young adult show,} and I'll admit that it does look like that upon first glance and upon viewing the first episode.
Once the series gets going, however, it gets dark, and starts becoming a compelling exploration of the dangers of tribalism, faith vs. pragmatism, and man's true nature.

\end{document}