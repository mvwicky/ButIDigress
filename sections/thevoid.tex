%!TEX program = lualatex
%!TEX root = ../butidigress.tex
\documentclass[../butidigress.tex]{subfiles}
\begin{document}
\chapter{The Void}\label{chap:thevoid}
\epi{Can one be a saint without God?, that's the problem, in fact the only problem I'm up against today.}{\attrib{Albert Camus}{The Plague}{1947}}
\newpage

\marginthought{this is the new introduction to this chapter, much of it adapted from the original}
The values and lessons presented in this chapter stem pretty directly from my partial embrace of Albert Camus' absurdism.
Taken directly from \book{The Myth of Sisyphus}, the \say{principles of the absurd man} have some problematic elements, not the least of which is the fact that the phrase \say{absurd man} is widely used throughout that essay.
That being said, I don't really have much experience with the philosophy, basically \book{Sisyphus}, a video about \movie{Synecdoche, New York}, some fragmented high school memories, and various Wikipedia/Stanford Encyclopedia readings.

I first came into contact with Camus in high school, in Mrs.~Nye's, whom I've mentioned before, sophomore year\ftnote{I think} English class.
We were reading \book{L'Étranger} (\book{The Outsider} or \book{The Stranger}), a pretty standard high school reading.
Not much attention was paid in that class, I vaguely remember her mentioning absurdism and Sisyphus and his boulder, but I didn't rediscover the philosophy until later, until college.\ftnote{I'm not exactly sure when, but somewhere around there}

\todobrak{the paragraph below is the old introduction}

This comes pretty clearly from my (partial) embrace of the absurd.\margintodo{this is crazy wrong, absurdism doesn't require the \textit{a priori} recognition of a god}
I struggle with Camus' absurd, caveat that I have read \book{The Myth of Sisyphus} but once; my difficulty stems from the acknowledgment of some form of higher power and the extremely gendered terminology in \book{Sisyphus}.
Intellectually I cannot really brook any consideration of \say{God} in any real form.
In normal conversation I tend to describe myself as agnostic, partially for the sake of expediency, partially out of my fear of certainty\ftnote{partially out of a fear of nothingness}, but in reality I am an atheist.
I'm not even really anti-religion, people are free to believe what they believe, and if it doesn't infringe on me, then more power to them.
Religion, for all its faults, has made some great things in the world, and getting a huge swath of humanity to conform to some morality is a good, even if most people don't follow exactly.\ftnote{this partial embrace of religion feels almost like a betrayal to my liberal, academic sensibility}

I originally titled this section \textit{Fight the Void}, which is not necessarily inappropriate; the current title reflects a broadening of scope.\margincomm{and I just like it better, feels cleaner}
While the fight against oblivion is important, some might say the most important, conceptually, it is necessary to understand what one is up against.
So this chapter title is meant to be about the fight, but also about the enemy.

The other reason for the title change is that I'm not trying to constrain myself to Camus' absurd, which has a relatively specific definition.
I want to discuss living in the face of the absolute knowledge of one's death, dealing with bleak circumstances\ftnote{one might say that absolute knowledge of one's death is bleak}, and perseverance through unimaginable hardship.
That's the broader theme of the chapter, the nobility of the struggle; I get that that sounds like absurdism.

\digsec{Camus' Absurd}{6/8/18}{camusabsurd}\margintodo{seems pretty self-evident; give an introduction to the \say{canonical} form of absurdism}

\begin{somenotes}{Camus' Absurd}
    \item lots of people called Camus an existentialist, he didn't call himself that
\end{somenotes}

Here I'll give a brief introduction to what the fuck absurdism even is\ldots\marginthought{I don't consider myself an \say{absurd man} in the way Camus puts forth; I do however, consider myself a man willing to yell at hurricanes, so to speak}

As I may have mentioned a couple times\ftnote{kinda not sure what's above this paragraph, lots of rewriting}, absurdism is most frequently associated with Albert Camus, the \nth{20} century French-Algerian author and philosopher.
The seminal work in this school of though is \book{The Myth of Sisyphus} an essay, written by Camus in 1942, which outlines the virtues of the \say{absurd man.}
Again, this work had some seriously gendered terminology, a kind of implicit belief that only the male mind is capable of such rich philosophical sight.
This dated way of thinking is easily reconcilable, just imagine \say{people.}

What it really boils down to\ftnote{or at least, this is the succinct way of thinking about it I just thought of} is that life, existence as a whole, is about the journey, because that's all there is.\autocite{mythofsisyphus}

\digsec{Expounding on the Infinite}{5/9/18}{voidexpounding}
\epi{One must imagine Sisyphus happy.}{\attrib{Albert Camus}{The Myth of Sisyphus}{1942}}
As previously asserted, I don't subscribe fully to Absurdism.\margintodo{this is just rambling at this point}
Conceptually it is appealing to me.

Maybe the higher power thing is not as integral as I thought.
It's more about the nobility of the fight, of not giving in.
I'll talk more about some specifics in the section on \tvshow{The 100}, (\ref{sec:thehundred}) but I'll cover some generalities right now.

\margintodo{some generalities}

In the \say{scheme of things} every single human life is entirely meaningless. Without fully unpacking what \say{scheme} this refers to we can take it for the titular \say{void}. Such a truth (i.e., the opening sentence to this paragraph) is pretty unavoidable. Logically, most everybody can come to this conclusion, yet we all continue to live.\par Why? Well the simple answer is that most people just ignore it. And why not? it's so easy and it makes life livable. I confess to intentional ignorance on most days.\margintodo{integrate this better}

\digsec{Case Study: \tvshow{The 100}}{5/9/18}{thehundred}\margintodo{basically copy (with revisions) the stuff I wrote in the appendix}
\tvshow{The~100} is a television show, currently\ftnote{as of this writing} airing on \textit{The~CW}.
Set in a dystopian future in which civilization, destroyed by global nuclear war, is kept alive by around 500 people aboard a space station, the Ark, orbiting Earth; the series starts around 100 years after the war.
In order to maintain discipline, all crimes are treated as capital offenses, except for those under eighteen, who are imprisoned.
There are, at the beginning of the series, exactly 100\ftnote{somehow\ldots} of said young offenders.
These reprobates\ftnote{some are actually bad people, others are locked up for more innocuous crimes or for getting caught being altruistic, like stealing medicine for sick loved ones and the like} are sent to the surface to test survivability of the radiation levels.

As one might expect, sending a group of raucous youths, unsupervised, to a, theoretically, uninhabited planet, leads to some hijinks, and a certain amount of attempted murder.
People to whom I've talked about the show tend to dismiss it as \say{just another young adult show,} and I'll admit that it does look like that upon first glance and upon viewing the first episode.
Once the series gets going, however, it gets dark, and starts becoming a compelling exploration of the dangers of tribalism, faith vs.\ pragmatism, and man's true nature.

Here's a little example of how gosh darn intense the show gets; this little scene is from season four, episode three\ftnote{I think, may be earlier or later}.
So there's this crazy radiation wave coming to kill everybody\ftnote{because of nuclear reactors that are breaking down; turns out the bad guy in season three genuinely thought she was saving humanity} and this family of \say{grounders}\ftnote{what the people from the Ark call the descendants those who managed to survive on Earth after the nuclear war} shows up at the space people's camp.
There's the dad (who was actually a pretty established character, which was weird) and the mom (who ends up being pretty important to the story) and a couple kids (or maybe just one, I don't remember for sure).
They're a cute family and one of the kids has pretty bad radiation poisoning.
Anyways, long story short the kid dies in her parents arms; like, she takes her last, strained and shuddering breaths, her breathing slowing until it stops.
Her parents, go from comforting their child, to the most abject grief anyone could possibly imagine.
That ain't normal, run-of-the-mill young adult fare; that's some hardcore stuff.
From here, we'll talk about some individual characters and overarching plot-lines as they relate to \say{The Void.}

One of the options available to a person who considers the universe's lack of feeling is to basically say \say{fuck it} and pretty much give up on life.
In fact, that may be the only real choice that we make in life, whether or not to live once our eyes have opened.\autocite{mythofsisyphus}
Having chosen the \say{throw it all away} route, there are additional forks; one can simply commit suicide\marginthought{I've always found the fact that we say \say{commit suicide} kinda weird, like it feels weirdly formal} or one can throw any sense of human decency to the wind and live a hedonistic, Caligula-esque life.
In \tv{The 100}, the character Jasper has taken these two paths; Earth was not friendly to him and he lost at every turn.

\begin{somenotes}{\tvshow{The 100}}
    \item (transcribed from my notebook, to be adapted into additions to the current section)
    \item What presented itself as a pretty standard young-adult show\ftnote{as in like, the first episode was pretty vanilla} is actually a super intense drama. For example, in the third (?) episode of the fourth season, I saw a child die in her parents arms, of radiation poisoning; she took her last breaths, her breathing slowed until it stopped, her parents went from comforting to breaking down. That's not some young-adult bs, that's for real. \par I do really hate the Jasper character. He's kinda an embodiment of postmodern, existentialism not caring about the world, that resignation to the universe's lack of regard for human life and morality. I recognize that the universe does not have a modicum of regard for us\ftnote{it can't it's inanimate, but the point is we're insignificant} but that does not, in my opinion, make my life meaningless. In fact, choosing to live my life by a moral code is a more noble choice, laughing in the face of the unknowable void is bravery. Succumbing to hopelessness, resigning oneself to a life free of morals or restraint is cowardly.\par I have been a fan of the show's interplay between faith and science. Clarke, generally, represents science, or pragmatism, she takes the practical solution, as hard as it may be. Jaha on the other hand, represents faith, he believes in long shots, maintaining hope, leading people to salvation through his beliefs. The show does a great job of showing that we need both.\par Being pragmatic is important, but sometimes hope is the only way out.
    \item \tvshow{The 100} contains multitudes. Jasper, while I still find him reprehensible, actually has a purpose. He is the embodiment of an unfettered Id. Obviously, that's not great in isolation. But he exists to show that there really is no point to all this survival if your humanity is lost on the way.
    \item It (\tvshow{The 100}) is, as it has ever been, a pretty brutal depiction of the conflict between humanity's base savagery and our better angels. Can we, as a species, be civil in the most trying circumstances. The bunker people (Octavia) and the prison crew (the Colonel) are the negative side of that. Clarke and the six on the Ark are the positive side.\par The fact that the \say{good} groups are small is a bit of a downer. But, the large groups have people who speak for civility (Kane and the Pilot in the bunker and with the prisoners, respectively). Obviously the season has yet to play out fully, so we don't know who wins out.
\end{somenotes}

\end{document}