%!TEX root = ../philo.tex
\documentclass[../philo.tex]{subfiles}
\begin{document}
\chapter{The Void}  % Fight the Void
\epigraph{Can one be a saint without God?, that's the problem, in fact the only problem I'm up against today.}{\attrib{Albert Camus}{The Plague}{1947}}
\newpage
This comes pretty clearly from my (partial) embrace of the absurd.
I struggle with Camus' absurd, caveat that I have read \textit{The Myth of Sisyphus} but once; my difficulty stems from the acknowledgment of some form of higher power and the extremely gendered terminology in \textit{Sisyphus}.
Intellectually I cannot really brook any consideration of god.
Generally I will describe myself as agnostic, partially out of my fear of nothingness, but in reality I am an atheist.
I'm not even really anti-religion, people are free to believe what they believe, and if it doesn't infringe on me, then more power to them.

I originally titled this section \textit{Fight the Void}, which is not necessarily inappropriate; the current title reflects a broadening of scope.

\section{Branching from Camus}
Again, I don't subscribe fully to Absurdism.

\end{document}