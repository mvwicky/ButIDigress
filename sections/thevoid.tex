%!TEX root = ../butidigress.tex
\documentclass[../butidigress.tex]{subfiles}
\begin{document}
\chapter{The Void}\label{chap:thevoid}
\epigraph{Can one be a saint without God?, that's the problem, in fact the only problem I'm up against today.}{\attrib{Albert Camus}{The Plague}{1947}}
\newpage
This comes pretty clearly from my (partial) embrace of the absurd.
I struggle with Camus' absurd, caveat that I have read \textit{The Myth of Sisyphus} but once; my difficulty stems from the acknowledgment of some form of higher power and the extremely gendered terminology in \textit{Sisyphus}.
Intellectually I cannot really brook any consideration of \say{God} in any real form.
In normal conversation I tend to describe myself as agnostic, partially for the sake of expediency, partially out of my fear of certainty\footnote{partially out of a fear of nothingness}, but in reality I am an atheist.
I'm not even really anti-religion, people are free to believe what they believe, and if it doesn't infringe on me, then more power to them.
Religion, for all its faults, has made some great things in the world, and getting a huge swath of humanity to conform to some morality is a good, even if most people don't follow exactly.\footnote{this partial embrace of religion feels almost like a betrayal to my liberal, academic sensibility}

I originally titled this section \textit{Fight the Void}, which is not necessarily inappropriate; the current title reflects a broadening of scope.\margincomm{and I just like it better, it's cleaner}

\entryskip

Here, I'll give a brief introduction to what the fuck absurdism even is\ldots

As noted above, absurdism is most frequently associated with Albert Camus, the \nth{20} century French-Algerian author and philosopher.
The seminal work in this school of though is \textit{The Myth of Sisyphus} an essay, written by Camus in \textcolor{red}{YEAR}, which outlines the virtues of the \say{absurd man.}
Again, this work had some seriously gendered terminology, a kind of implicit belief that only the male mind is capable of such rich philosophical sight.
This dated way of thinking is easily reconcilable, just imagine \say{people.}

What it really boils down to\footnote{or at least, this is the succinct way of thinking about it I just thought of} is that life, existence as a whole, is about the journey, because that's all there is.

\section{Expounding on the Infinite}
\epigraph{One must imagine Sisyphus happy.}{\attrib{Albert Camus}{The Myth of Sisyphus}{1947?}}
As previously asserted, I don't subscribe fully to Absurdism.\margintodo{I honestly don't know what I'm gonna write for this chapter generally, still some thought to be had.}
Conceptually it is appealing to me.

Maybe the higher power thing is not as integral as I thought.
It's more about the nobility of the fight, of not giving in.
I'll talk more about some specifics in the section on \textit{The 100}, (\ref{sec:thehundred}) but I'll cover some generalities right now.

\section{Case Study: \textit{The 100}}\label{sec:thehundred}\margintodo{basically copy (with revisions) the stuff I wrote in the appendix}

\end{document}