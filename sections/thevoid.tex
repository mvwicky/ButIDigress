%!TEX root = ../butidigress.tex
\documentclass[../butidigress.tex]{subfiles}
\begin{document}
\chapter{The Void}  % Fight the Void
\epigraph{Can one be a saint without God?, that's the problem, in fact the only problem I'm up against today.}{\attrib{Albert Camus}{The Plague}{1947}}
\newpage
This comes pretty clearly from my (partial) embrace of the absurd.
I struggle with Camus' absurd, caveat that I have read \textit{The Myth of Sisyphus} but once; my difficulty stems from the acknowledgment of some form of higher power and the extremely gendered terminology in \textit{Sisyphus}.
Intellectually I cannot really brook any consideration of `God' in any real form.
In normal conversation I tend to describe myself as agnostic, partially for the sake of expediency, partially out of my fear of nothingness, but in reality I am an atheist.
I'm not even really anti-religion, people are free to believe what they believe, and if it doesn't infringe on me, then more power to them.
Religion, for all its faults, has made some great things in the world, and getting a huge swath of humanity to conform to some morality is a good, even if most people don't follow exactly.

I originally titled this section \textit{Fight the Void}, which is not necessarily inappropriate; the current title reflects a broadening of scope.\margincomm{and I just like it better, it's cleaner}

\section{Branching from Camus}
Again, I don't subscribe fully to Absurdism.\margintodo{I honestly don't know what I'm gonna write for this chapter generally, still some thought to be had.}
Conceptually it is appealing to me.

\end{document}